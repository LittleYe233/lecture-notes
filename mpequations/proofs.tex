\section{部分推导过程及说明} \label{部分推导过程及说明}
\subsection{行波法与积分变换法} \label{proofs 行波法与积分变换法}
\subsubsection{三维波动方程的球对称解} \label{proofs 3 三维波动方程的球对称解}
在球坐标系下, 波动方程 (\ref{eq:3.2 1 1}) 可以写为
\begin{equation}
    \frac{1}{r^2}\frac{\partial}{\partial r}\left(r^2\frac{\partial u}{\partial r}\right)+\frac{1}{r^2\sin\theta}\frac{\partial}{\partial\theta}\left(\sin\theta\frac{\partial u}{\partial\theta}\right)+\frac{1}{r^2\sin^2\theta}\frac{\partial^2u}{\partial\varphi^2}=\frac{1}{a^2}\frac{\partial^2u}{\partial t^2}.
\end{equation}

因为此时求的是球对称解, 可以认为
\begin{equation} \label{eq:3.2 1 2}
    \frac{1}{r^2\sin\theta}\frac{\partial}{\partial\theta}\left(\sin\theta\frac{\partial u}{\partial\theta}\right)=\frac{1}{r^2\sin^2\theta}\frac{\partial^2u}{\partial\varphi^2}=0.
\end{equation}
故方程 (\ref{eq:3.2 1 2}) 可以简化为
\begin{equation}
    \begin{aligned}
                    & \frac{1}{r^2}\frac{\partial}{\partial r}\left(r^2\frac{\partial u}{\partial r}\right)=\frac{1}{a^2}\frac{\partial^2u}{\partial t^2} \\
        \Rightarrow & r\frac{\partial^2u}{\partial r^2}+2\frac{\partial u}{\partial r}=\frac{r}{a^2}\frac{\partial^2u}{\partial t^2}.
    \end{aligned}
\end{equation}

又因为
\begin{equation*}
    r\frac{\partial^2u}{\partial r^2}+2\frac{\partial u}{\partial r}=\frac{r}{a^2}\frac{\partial^2u}{\partial t^2},
\end{equation*}
所以可以得到关于 $ru$ 的一维波动方程 (\ref{eq:3.2 ru eq}) 与其通解 (\ref{eq:3.2 ru eq sol}).

\subsection{拉普拉斯方程的格林函数法} \label{proofs 拉普拉斯方程的格林函数法}
\subsubsection{调和函数的积分表达式} \label{proofs 4 调和函数的积分表达式}
设 $u$ 为调和函数, $M_0$ 为 $\Omega$ 中任意一点. 构造函数 (三维拉普拉斯方程的\textbf{基本解})
\begin{equation}
    v=\frac{1}{r}=\frac{1}{r_{MM_0}}=\frac{1}{\sqrt{(x-x_0)^2+(y-y_0)^2+(z-z_0)^2}}.
\end{equation}
显然该函数除点 $M_0$ 外处处满足拉普拉斯方程. 作以 $M_0$ 为球心, 充分小正数 $\epsilon$ 为半径的球面 $\Gamma_\epsilon$, 其形成球域 $K_\epsilon$.

由此, 新的求解域为 $\Omega-K_\epsilon$, 新的求解边界为 $\Gamma+\Gamma+\epsilon$. 利用格林第二公式 (\ref{eq:4.2 green 2}), 得到
\begin{equation}
    \iiint_{\Omega-K_\epsilon}(u\Delta v-v\Delta u)\mathrm{d}V=\iint_{\Gamma}\left(u\frac{\partial v}{\partial\boldsymbol{n}}-v\frac{\partial u}{\partial\boldsymbol{n}}\right)\mathrm{d}S+\iint_{\Gamma_\epsilon}\left(u\frac{\partial v}{\partial\boldsymbol{n}}-v\frac{\partial u}{\partial\boldsymbol{n}}\right)\mathrm{d}S.
\end{equation}

因为在 $\Omega-K_\epsilon$ 内 $\Delta u=\Delta v=0$, 且球面 $\Gamma_\epsilon$ 上
\begin{equation*}
    \frac{\partial v}{\partial\boldsymbol{n}}=\frac{\partial}{\partial(-r)}\left(\frac{1}{r}\right)=\frac{1}{r^2}=\frac{1}{\epsilon^2},
\end{equation*}
所以
\begin{equation}
    \iint_{\Gamma_\epsilon}u\frac{\partial v}{\partial\boldsymbol{n}}\mathrm{d}S=\iint_{\Gamma_\epsilon}\frac{u}{\epsilon^2}\mathrm{d}S=\frac{4\pi}{4\pi\epsilon^2}\iint_{\Gamma_\epsilon}u\mathrm{d}S=4\pi\overline{u}.
\end{equation}
其中 $\overline{u}$ 为平均值函数 (\ref{eq:3.2 ave u}). 当 $\epsilon\rightarrow 0$ 时, 上式等于 $4\pi u(M_0)$.

同时,
\begin{equation}
    \iint_{\Gamma_\epsilon}v\frac{\partial u}{\partial\boldsymbol{n}}\mathrm{d}S=\frac{1}{\epsilon}\iint_{\Gamma_\epsilon}\frac{\partial u}{\partial\boldsymbol{n}}\mathrm{d}S=4\pi\epsilon\overline{\left(\frac{\partial u}{\partial\boldsymbol{n}}\right)}.
\end{equation}
当 $\epsilon\rightarrow 0$ 时, 上式等于 $0$. 由此, 当 $\epsilon\rightarrow 0$ 时, 可以得到式 (\ref{eq:4.2 u M_0}).
