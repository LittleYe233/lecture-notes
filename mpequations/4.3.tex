\subsection{格林函数} \label{4 格林函数}
在格林第二公式 (\ref{eq:4.2 green 2}) 中, 取 $u$ 为 $\Omega$ 内的调和函数, 选取 $v$ 满足
\begin{equation}
    v|_\Gamma=\left.\frac{1}{4\pi r_{MM_0}}\right|_\Gamma,
\end{equation}
且在 $\overline{\Omega}$ 上有连续的一阶偏导数, 则得
\begin{equation} \label{eq:4.3 laplace u M_0}
    u(M_0)=-\iint_\Gamma u\frac{\partial G}{\partial\boldsymbol{n}}\mathrm{d}S,
\end{equation}
其中
\begin{equation} \label{eq:4.3 green func}
    G(M,M_0)=\frac{1}{4\pi r_{MM_0}}-v
\end{equation}
称为\textbf{格林函数}. (\ref{eq:4.3 laplace u M_0}) 为拉普拉斯方程的狄利克雷问题的解. 而对于如下的拉普拉斯方程和边值条件 (泊松方程的狄利克雷问题)
\begin{equation}
    \begin{cases}
        \nabla^2u=F, & \qquad \text{在 $\Omega$ 内}, \\
        u|_\Gamma=f,
    \end{cases}
\end{equation}
有如下的通解
\begin{equation}
    u(M_0)=-\iint_{\Gamma}f\frac{\partial G}{\partial n}\mathrm{d}S-\iiint_{\Omega}GF\mathrm{d}V.
\end{equation}

一旦求出 $G(M,M_0)$ 中的调和函数 $v$ (相当于求得了 $G(M,M_0)$ 本身), 即可求得如上拉普拉斯 (泊松) 方程的狄利克雷问题的解. 确定 $G(M,M_0)$ 归结于解如下狄利克雷问题:
\begin{equation}  \label{eq:4.3 v eq}
    \begin{cases}
        \Delta v=0, & \text{在 $\Omega$ 内} \\
        v|_\Gamma=\left.\frac{1}{4\pi r_{MM_0}}\right|_\Gamma.
    \end{cases}
\end{equation}
