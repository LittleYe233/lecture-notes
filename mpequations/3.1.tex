\section{行波法与积分变换法} \label{行波法与积分变换法}
\subsection{一维波动方程的达朗贝尔公式} \label{3 一维波动方程的达朗贝尔公式}
考虑一维\textit{无界}弦的自由振动的定解问题:
\begin{numcases}{}
    \frac{\partial^2u}{\partial t^2}=a^2\frac{\partial^2u}{\partial x^2}, & \quad \color{red} $-\infty<x<+\infty,t>0$ \label{eq:3.1 eq} \\
    u|_{t=0}=\varphi(x),u_t|_{t=0}=\psi(x). & \quad \color{red} $-\infty<x<+\infty$ \label{eq:3.1 cond}
\end{numcases}

作如下变换:
\begin{equation} \label{eq:3.1 xi eta}
    \begin{cases}
        \xi=x+at, \\
        \eta=x-at.
    \end{cases}
\end{equation}

将 (\ref{eq:3.1 xi eta}) 代入 (\ref{eq:3.1 eq}), 得
\begin{align}
     & \begin{aligned}
           \frac{\partial u}{\partial x}
            & =\frac{\partial u}{\partial\xi}\frac{\partial\xi}{\partial x}+\frac{\partial u}{\partial\eta}\frac{\partial\eta}{\partial x} \\
            & =\frac{\partial u}{\partial\xi}+\frac{\partial u}{\partial\eta};
       \end{aligned}                                                                                                                                                                               \\
     & \begin{aligned}
           \frac{\partial u}{\partial t}
            & =\frac{\partial u}{\partial\xi}\frac{\partial\xi}{\partial t}+\frac{\partial u}{\partial\eta}\frac{\partial\eta}{\partial t} \\
            & =a\left(\frac{\partial u}{\partial\xi}-\frac{\partial u}{\partial\eta}\right);
       \end{aligned}                                                                                                                                                                               \\
     & \begin{aligned}
           \frac{\partial^2u}{\partial x^2}
            & =\frac{\partial}{\partial\xi}\left(\frac{\partial u}{\partial\xi}+\frac{\partial u}{\partial\eta}\right)\frac{\partial\xi}{\partial x}+\frac{\partial}{\partial\eta}\left(\frac{\partial u}{\partial\xi}+\frac{\partial u}{\partial\eta}\right)\frac{\partial\eta}{\partial x} \\
            & =\frac{\partial^2u}{\partial\xi^2}+2\frac{\partial^2u}{\partial\xi\partial\eta}+\frac{\partial^2u}{\partial\eta^2};
       \end{aligned} \label{eq:3.1 u_xx}                             \\
     & \begin{aligned}
           \frac{\partial^2u}{\partial t^2}
            & =\frac{\partial}{\partial\xi}\left[a\left(\frac{\partial u}{\partial\xi}-\frac{\partial u}{\partial\eta}\right)\right]\frac{\partial\xi}{\partial t}+\frac{\partial}{\partial\eta}\left[a\left(\frac{\partial u}{\partial\xi}-\frac{\partial u}{\partial\eta}\right)\right]\frac{\partial\eta}{\partial t} \\
            & =a^2\left[\frac{\partial}{\partial\xi}\left(\frac{\partial u}{\partial\xi}-\frac{\partial u}{\partial\eta}\right)-\frac{\partial}{\partial\eta}\left(\frac{\partial u}{\partial\xi}-\frac{\partial u}{\partial\eta}\right)\right]                                                                          \\
            & =a^2\left(\frac{\partial^2u}{\partial\xi^2}-2\frac{\partial^2u}{\partial\xi\partial\eta}+\frac{\partial^2u}{\partial\eta^2}\right).
       \end{aligned} \label{eq:3.1 u_tt}
\end{align}
上述推导利用了 (\ref{eq:3.1 xi eta}):
\begin{equation}
    \begin{cases}
        \frac{\partial\xi}{\partial x}=\frac{\partial\xi}{\partial y}=1, \\
        \frac{\partial\xi}{\partial t}=a,                                \\
        \frac{\partial\eta}{\partial t}=-a.
    \end{cases}
\end{equation}

结合 (\ref{eq:3.1 u_xx}) 和 (\ref{eq:3.1 u_tt}), 得
\begin{equation}
    \frac{\partial^2 u}{\partial\xi\partial\eta}=0.
\end{equation}
上式依次对 $\eta$, $\xi$ 积分:
\begin{gather}
    \frac{\partial u}{\partial\xi}=f_1'(\xi); \\
    u=\int f_1'(\xi)\mathrm{d}\xi+f_2(y)=f_1(\xi)+f_2(\eta). \label{eq:3.1 u}
\end{gather}

将 (\ref{eq:3.1 xi eta}) 代入 (\ref{eq:3.1 u}), 得
\begin{equation} \label{eq:3.1 u x,t}
    u(x,t)=f_1(x+at)+f_2(x-at).
\end{equation}
再将 (\ref{eq:3.1 cond}) 代入上式, 得
\begin{gather}
    f_1(x)+f_2(x)=\varphi(x); \label{eq:3.1 f_1 f_2} \\
    af_1'(x)-af_2'(x)=\psi(x). \label{eq:3.1 af_1 af_2}
\end{gather}

对 (\ref{eq:3.1 af_1 af_2}) 积分:
\begin{equation}
    f_1(x)-f_2(x)=\frac{1}{a}\int_{0}^{x}\psi(x)\mathrm{d}x+C.
\end{equation}
结合上式和 (\ref{eq:3.1 f_1 f_2}), 得
\begin{equation}
    \begin{cases}
        f_1(x)=\frac{1}{2}\left[\varphi(x)+\frac{1}{a}\int_{0}^{x}\psi(x)\mathrm{d}x+C\right], \\
        f_2(x)=\frac{1}{2}\left[\varphi(x)-\frac{1}{a}\int_{0}^{x}\psi(x)\mathrm{d}x-C\right].
    \end{cases}
\end{equation}
上式代入 (\ref{eq:3.1 u x,t}), 得
\begin{equation} \label{eq:3.1 d'Alembert}
    u(x,t)=\frac{1}{2}[\varphi(x+at)+\varphi(x-at)]+\frac{1}{2a}\int_{x-at}^{x+at}\psi(x)\mathrm{d}x.
\end{equation}

式 (\ref{eq:3.1 d'Alembert}) 被称为无限长弦自由振动的达朗贝尔公式.

\subsubsection{达朗贝尔公式的物理意义}
\textbf{左行波}\quad $f_1(x+at)$.

\textbf{右行波}\quad $f_2(x-at)$.

\textbf{依赖区间}\quad $[x-at,x+at]$. 解在 $(x,t)$ 的数值仅依赖该区间内的初值条件, 而与该区间之外的无关.

\textbf{决定区域}\quad 对于初始轴 $t=0$ 上的一个区间 $[x_1,x_2]$, 绘出一个等腰三角形区域 (两腰直线方程 $x=x_1+at$, $x=x_2-at$), 该区域内任何一点 $(x,t)$ 的依赖区间都落在 $[x_1,x_2]$ 的内部, 因此解在此三角形区域中的数值完全由 $[x_1,x_2]$ 上的初值条件决定. 换言之, 给定 $[x_1,x_2]$ 的初值条件, 即可推得决定区域中初值问题的解.

\textbf{影响区域}\quad $\{(x,t)|x_1-at\leq x\leq x_2+at\}$ ($t>0$). 经过 $t$ 时间后受到区间 $[x_1,x_2]$ 初始扰动影响的区域. 在此区域之外的波动不受 $[x_1,x_2]$ 初始扰动影响.

\textbf{特征线}\quad $x\pm at=C$ (两族直线).

\subsubsection{数理方程的分类}
考虑如下的一般二阶线性偏微分方程
\begin{equation}
    A\frac{\partial^2u}{\partial x^2}+2B\frac{\partial^2u}{\partial x\partial y}+C\frac{\partial^2u}{\partial y^2}+D\frac{\partial u}{\partial x}+E\frac{\partial u}{\partial y}+Fu=0.
\end{equation}
其\textbf{特征方程}为
\begin{equation}
    A(\mathrm{d}y)^2{\color{red} -}2B\mathrm{d}x\mathrm{d}y+C(\mathrm{d}x)^2=0.
\end{equation}

该偏微分方程的特征线即上述特征方程的积分曲线, 而其仅与偏微分方程的二阶导数项的系数有关. 令 $\Delta=B^2-AC$. 具体的微分方程系数和特征线的关系见表 \ref{tab:3.1 eq feat}.
\begin{table}[H]
    \caption{二阶偏微分方程的其中三种类型与其特征线的关系} \label{tab:3.1 eq feat}
    \centering
    \begin{tabular}{ccc} \hline
        \bfseries 微分方程类型 & \bfseries 过某 $xt$ 区域内的 $\Delta$ & \bfseries 过该区域每一点的实特征线数量 \\ \hline
        椭圆型              & $\Delta<0$                      & 不存在                      \\ \hline
        抛物型              & $\Delta=0$                      & 仅有一条                     \\ \hline
        双曲型              & $\Delta>0$                      & 两条相异                     \\ \hline
    \end{tabular}
\end{table}

具体来说:
\begin{itemize}
    \item 拉普拉斯方程与泊松方程属于椭圆型;
    \item 热传导方程属于抛物型;
    \item 波动方程属于双曲型 (\textit{可以使用行波法}).
\end{itemize}

对于特征线方程, 将 $\mathrm{d}y$, $\mathrm{d}x$ 分别替换为 $y$, $x$, 再对其因式分解即可得到特征线方程; 注意得到的方程常数端需要对应替换为待定常数 $C_1$, $C_2$.

对于双曲型方程的通解, 其可以表为 $u(x,t)=f_1(px+qy)+f_2(ux+wy)$, 其中 $\begin{cases}
        px+qy=C_1, \\
        ux+wy=C_2
    \end{cases}$ 为特征线方程.

\subsubsection{非齐次方程的情形}
若需要解如下定解问题:
\begin{numcases}{}
    \frac{\partial^2u}{\partial t^2}=a^2\frac{\partial^2u}{\partial x^2}{\color{red} +f(x,t)}, & \qquad $-\infty<x<+\infty,t>0$ \label{eq:3.1 non 1} \\
    u|_{t=0}=\varphi(x),u_t|_{t=0}=\psi(x). & \qquad $-\infty<x<+\infty$
\end{numcases}

将方程 (\ref{eq:3.1 non 1}) 中 $u(x,t)$ 分解为 $v(x,t)+w(x,t)$. 其中自由振动 $v(x,t)$ 满足
\begin{equation}
    \begin{cases}
        \frac{\partial^2v}{\partial t^2}=a^2\frac{\partial^2v}{\partial x^2}, \\
        v|_{t=0}=\varphi(x),                                                  \\
        v_t|_{t=0}=\psi(x).
    \end{cases}
\end{equation}
强迫振动 $w(x,t)$ 满足
\begin{numcases}{}
    \frac{\partial^2w}{\partial t^2}=a^2\frac{\partial^2w}{\partial x^2}+f(x,t), \label{eq:3.1 non 2} \\
    w|_{t=0}=w_t|_{t=0}=0.
\end{numcases}

显然 $v(x,t)$ 满足达朗贝尔公式. 对于 $w(x,t)$, 在 $[\tau,\tau+\mathrm{d}\tau]$ 时间范围内, (\ref{eq:3.1 non 2}) 可以分别写成
\begin{gather}
    \frac{\partial^2w(\tau)}{\partial t^2}=a^2\frac{\partial^2w(\tau)}{\partial x^2}+f(x,\tau), \\
    \frac{\partial^2w(\tau+\mathrm{d}\tau)}{\partial t^2}=a^2\frac{\partial^2w(\tau+\mathrm{d}\tau)}{\partial x^2}+f(x,\tau+\mathrm{d}\tau).
\end{gather}

因为 $f(x,\tau+\mathrm{d}\tau)=f(x,\tau)$, 所以, 上两式相减得
\begin{equation}
    \frac{\partial^2}{\partial t^2}\left[\frac{w(\tau+\mathrm{d}\tau)-w(\tau)}{\mathrm{d}\tau}\right]=a^2\frac{\partial^2}{\partial x^2}\left[\frac{w(\tau+\mathrm{d}\tau)-w(\tau)}{\mathrm{d}\tau}\right].
\end{equation}

令 $\overline{w}=\dfrac{\partial w}{\partial t}$, 则上式可以写成
\begin{equation}
    \frac{\partial^2\overline{w}}{\partial t^2}=a^2\frac{\partial^2\overline{w}}{\partial x^2}.
\end{equation}

在 $[\tau,\tau+\mathrm{d}\tau]$ 时间范围内, 位移变化应为 0, 也即
\begin{equation}
    \begin{aligned}
                    & w(\tau+\mathrm{d}\tau)-w(\tau)=0                \\
        \Rightarrow & \overline{w}|_{t=\tau}=\frac{0}{\mathrm{d}t}=0.
    \end{aligned}
\end{equation}
而速度应满足\textit{冲量定理}, 也即
\begin{equation}
    \begin{aligned}
                    & \frac{\partial w(\tau+\mathrm{d}\tau)}{\partial t}-\frac{\partial w(\tau)}{\partial t}=f(x,\tau)\mathrm{d}\tau \\
        \Rightarrow & \overline{w}_{t}|_{t=\tau}=f(x,\tau).
    \end{aligned}
\end{equation}

此时, 联立以上三式, 即得关于 $\overline{w}$ 的定解问题 (取值范围略)
\begin{equation}
    \begin{cases}
        \frac{\partial^2\overline{w}}{\partial t^2}=a^2\frac{\partial^2\overline{w}}{\partial x^2}, \\
        \overline{w}|_{t=\tau}=0,\overline{w}_{t}|_{t=\tau}=f(x,\tau).
    \end{cases}
\end{equation}
对上述定解问题, 令 $T=t-\tau$, 对该定解问题则可以使用达朗贝尔公式, 解得
\begin{equation}
    \overline{w}=\frac{1}{2a}\int_{x-aT}^{x+aT}f(x,t)\mathrm{d}x.
\end{equation}

又因为 $\overline{w}=\dfrac{\partial w}{\partial t}$, 所以
\begin{equation}
    w(x,t)=\int_{0}^{t}\frac{1}{2a}\int_{x-a(t-\tau)}^{x+a(t-\tau)}f(x,t)\mathrm{d}x\mathrm{d}t.
\end{equation}

由此, 即可解得 $u(x,t)$ 的解.
