\section*{课程信息} \label{课程信息}
\addcontentsline{toc}{section}{课程信息}
\subsection{基本内容}
\begin{center}
    \begin{tabularx}{.5\textwidth}[ht]{>{\bfseries}c>{\centering\arraybackslash}X} \hline
        课程代码   & MATH1007                       \\
        课程中文名称 & 数理方程                           \\
        课程英文名称 & Mathematical Physics Equations \\
        课程类别   & 其他                             \\
        总学分    & 2                              \\
        考核方式   & 考查                             \\
        总学时    & 32                             \\
        \hline
    \end{tabularx}
\end{center}

\subsection{课程中文简介}

本课程主要是求解在物理, 力学以及电子工程技术中常见的一些偏微分方程. 通过本课程的教学使学生初步了解和掌握数理方程这一学科的基本概念及理论, 培养学生综合应用所学的数学知识解决其相关专业学科中所可能遇到的一些偏微分方程的定解问题. 通过学习三类典型方程的归纳, 求解方法及适应性理论, 能够具备将一些实际问题归纳为数学问题的初步能力, 初步掌握解决偏微分方程的各种方法, 最终为从事相关专业的教学和研究打下一定的数学理论基础.

\subsection{分数构成}
