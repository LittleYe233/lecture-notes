\subsection{两种特殊区域的格林函数及狄利克雷问题的解} \label{4 两种特殊区域的格林函数及狄利克雷问题的解}
特殊区域的格林函数, 即定解问题 (\ref{eq:4.3 v eq}) 可以使用\textbf{电像法}求得.

\textbf{电像法}

在区域 $\Omega$ 外找出 $M_0$ 关于边界 $\Gamma$ 的像点 $M_1$, 然后在这个像点放置适当的负电荷, 由它产生的负电位与点 $M_0$ 处单位正电荷所产生的正电势在曲面 $\Gamma$ 上互相抵消.可以证明 $M_1$ 处的负电荷形成的电场的电势在 $\Gamma$ 内部是调和函数 $v$ (满足 $\Delta v=0$), 且有 $v|_\Gamma\left.=\dfrac{1}{4\pi r_{MM_0}}\right|_\Gamma$.

因为 $M_1$ 的电荷电性为负, 而 $v$ 显然表示电势的绝对值 (因为 $\dfrac{1}{4\pi r_{MM_0}}>0$), 因此由 $M_0$ 和 $M_1$ 处的电荷所形成电场在 $\Omega$ 内的电势为 $\dfrac{1}{4\pi r_{MM_0}}-v$, 这即是式 (\ref{eq:4.3 green func}) 的格林函数 $G(M,M_0)$.

\subsubsection{半空间的格林函数}
求解拉普拉斯方程在半空间 $z\geq 0$ 内的狄利克雷问题, 就是求函数 $u(x,y,z)$ 使适合
\begin{equation} \label{eq:4.4 half space eq}
    \begin{cases}
        \nabla^2u=0,     & z>0                 \\
        u|_{z=0}=f(x,y). & -\infty<x,y<+\infty
    \end{cases}
\end{equation}

显然 $z>0$ 内某点 $M_0(x_0,y_0,z_0)$ 处的单位正电荷的像电荷位于 $M_1(x_0,y_0,-z_0)$, 为单位负电荷. 因此, 格林函数为
\begin{equation}
    G(M,M_0)=\frac{1}{4\pi}\left(\frac{1}{r_{MM_0}}-\frac{1}{r_{MM_1}}\right).
\end{equation}

因为区域 $z\geq 0$ 在边界 $z=0$ 上的外法线方向是 $Oz$ 轴负向, 所以有
\begin{equation}
    \begin{aligned}
        \left.\frac{\partial G}{\partial\boldsymbol{n}}\right|_{z=0}
         & =-\left.\frac{\partial G}{\partial z}\right|_{z=0}                                          \\
         & =\frac{1}{4\pi}\left\{\frac{z-z_0}{[(x-x_0)^2+(y-y_0)^2+(z-z_0)^2]^{\frac{3}{2}}}\right.    \\
         & \phantom{=}\left.-\frac{z+z_0}{[(x-x_0)^2+(y-y_0)^2+(z+z_0)^2]^{\frac{3}{2}}}\right\}_{z=0} \\
         & =-\frac{z_0}{2\pi[(x-x_0)^2+(y-y_0)^2+z_0^2]^\frac{3}{2}}.
    \end{aligned}
\end{equation}

对于边界 $\Gamma:z=0$, $\mathrm{d}S=\mathrm{d}x\mathrm{d}y$. 因此, 问题 (\ref{eq:4.4 half space eq}) 的解为
\begin{equation}
    u(M_0)=\frac{1}{2\pi}\int_{-\infty}^{+\infty}\int_{-\infty}^{+\infty}\frac{z_0f(x,y)}{[(x-x_0)^2+(y-y_0)^2+z_0^2]^\frac{3}{2}}\mathrm{d}x\mathrm{d}y.
\end{equation}

\subsubsection{球域的格林函数}
