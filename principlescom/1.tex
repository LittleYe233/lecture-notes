\section{绪论}
\subsection{基本概念与信息的度量}
\textbf{消息}\quad 信息的载体, 具体的事物.

\textbf{信号}\quad 消息的传输载体, 具体的事物.

\textbf{信息}\quad 消息中所含的\textit{有效}信息, 与概率有关.
\begin{itemize}
    \item 与概率相关: $I=I[P(x)]$;
    \item 严格单调递减: $I[P(x)=1]=0$, $I[P(x)=0]=1$;
    \item 可加性: $I[P(x_1)P(x_2)P(x_3)\cdots]=I[P(x_1)]+I[P(x_2)]+I[P(x_3)]+\cdots$.
\end{itemize}

\textbf{信息量}
\begin{equation}
    I=-\log_aP(x).
\end{equation}
\begin{itemize}
    \item $a=2$, 单位为比特 (bit, b);
    \item $a=e$, 单位为奈特 (nat);
    \item $a=10$, 单位为哈特莱 (Hartley).
\end{itemize}

\textbf{信源的熵}\quad 每个符号所含信息量的统计平均.
\begin{gather}
    H(x)=-\sum_{i=1}^{M}P(x_i)\log_2P(x_i), \\
    H(x)=-\int_{-\infty}^{\infty}f(x)\log_af(x)\mathrm{d}x.
\end{gather}
单位为 b/符号.

当信源传输的 $M$ 个波形是等概率的 ($P(x_i)=\dfrac{1}{M}$), 信源熵\textit{最大}, 即 $\log_2M$.

\subsection{数字通信系统}
\textbf{信源编码与译码}
\begin{itemize}
    \item 提高信息传输的有效性 (压缩编码, 降低码元速率);
    \item 模/数转换.
\end{itemize}

\textbf{信道编码与译码}\quad 差错控制 (抗干扰编码).

\textbf{加密与解密}\quad 添加密码.

\subsection{通信系统的主要性能指标}
\subsubsection{有效性}
\textbf{频带利用率}\quad 单位带宽内的传输速率.
\begin{gather}
    \eta=\frac{R_B}{B}\ (\unit{Baud/Hz}), \\
    \eta_b=\frac{R_b}{B}\ (\unit{b/(s.Hz)}).
\end{gather}

\textbf{码元传输速率 (波特率)}\quad 单位时间传输码元 (符号) 数目.
\begin{equation}
    R_B=\frac{1}{T_B\ (\unit{s})}\ (\unit{Baud}).
\end{equation}

\textbf{信息传输速率 (比特率)}\quad 单位时间传输平均信息量.
\begin{equation}
    R_b=R_B\log_2M\ (\unit{b/s}).
\end{equation}

\subsubsection{可靠性}
\textbf{误码率}
\begin{equation}
    P_e=\frac{\textrm{错误码元数}}{\textrm{传输总码元数}}.
\end{equation}

\textbf{误信率}
\begin{equation}
    P_b=\frac{\textrm{错误比特数}}{\textrm{传输总比特数}}.
\end{equation}

对于二进制信息, $P_b=P_e$.
