\documentclass{notes}

\title{Lecture Notes on Radio Frequency Electronic Circuits}
\author{\href{https://github.com/LittleYe233}{LittleYe233} \textit{a.k.a.} Zhifan Ye <\href{mailto:littleye233@gmail.com}{littleye233@gmail.com}>}
% See: https://tex.stackexchange.com/a/142973
\datemodified{\DTMdate{2023-2-26}}
\version{1.2.0}

\tikzstyle{normal} = [
    rectangle,
    rounded corners,
    minimum width=2cm,
    minimum height=1cm,
    text centered,
    draw=black
]

\begin{document}

\maketitle
\tableofcontents
\newpage

\section*{课程信息} \label{课程信息}
\addcontentsline{toc}{section}{课程信息}

\subsection{地点}
\textbf{教室:} T5204 \par
\textbf{座位:} 6 排 4 列

\subsection{授课教师}
王凯旭 <\href{mailto:wangkaixu@hit.edu.cn}{wangkaixu@hit.edu.cn}>

\subsection{分数构成}
\begin{itemize}
    \item 平时成绩 (30\%)
          \begin{itemize}
              \item 作业 (5\%)
              \item 课堂表现 (5\%)
              \item 随堂作业或测验 (20\%)
          \end{itemize}
    \item 期末考试 (70\%)
\end{itemize}

\section{绪论} \label{绪论}
\subsection{无线电信号传输原理}
\subsubsection{传输信号的基本方法}
\begin{figure}[H]
    \centering
    \begin{tikzpicture}
        \node (1) [normal] {信号源};
        \node (2) [normal, right=of 1] {发送设备};
        \node (3) [normal, right=of 2] {传输信道};
        \node (4) [normal, right=of 3] {接收设备};
        \node (5) [normal, right=of 4] {收信装置};

        \draw [-latex] (1) edge (2)
        (2) edge (3)
        (3) edge (4)
        (4) edge (5);
    \end{tikzpicture}
    \caption{通信系统组成框图}
\end{figure}

\begin{itemize}
    \item \textbf{发送设备}\quad 转换为适合传输的信号.
    \item \textbf{收信装置}\quad 扬声器, 等.
\end{itemize}

\subsubsection{无线电信号的产生与发射}
\begin{figure}[H]
    \centering
    \begin{tikzpicture}
        \node (1) [normal] {高频振荡};
        \node (2) [normal, right=of 1] {倍频};
        \node (3) [normal, right=of 2] {高频放大};
        \node (4) [normal, right=of 3] {调制};
        \node (5) [normal, below=of 2] {音频源};
        \node (6) [normal, right=of 5] {音频放大};
        \node (7) [right=1.5cm of 4] {};

        \draw [-latex] (1) -- (2) node [midway, above] {缓冲};
        \draw [-latex] (2) edge (3)
        (3) edge (4)
        (5) edge (6)
        (6) -| (4);
        \draw [-latex] (4) -- (7) node [midway, above] {传输线};
    \end{tikzpicture}
    \caption{无线电发射机方框图}
\end{figure}

\textbf{调制}\quad 将声音电流加在高频电流上.
\begin{itemize}
    \item \textbf{分类}\quad 调幅 (AM, 高频振幅变化所形成的包络波形即为原信号波形), 调频 (FM), 调相 (PM).
    \item \textbf{原因}
          \begin{itemize}
              \item 传输低频信号需要更庞大的天线;
              \item 减少干扰, 提高信道复用率 (将信号放大到不同的频段).
          \end{itemize}
\end{itemize}

\subsubsection{无线电信号的接收}
\textbf{接收设备}要求尽可能减少失真.

\begin{figure}[H]
    \centering
    \begin{tikzpicture}
        \node (1) [] {};
        \node (2) [normal, right=1.5cm of 1] {选择性电路};
        \node (3) [normal, right=of 2] {检波};
        \node (4) [normal, right=of 3] {发声装置};

        \draw [-latex] (1) -- (2) node [midway, above] {传输线};
        \draw [-latex] (2) edge (3)
        (3) edge (4);
    \end{tikzpicture}
    \caption{最简单的接收机方框图}
\end{figure}

\begin{figure}[H]
    \centering
    \begin{tikzpicture}
        \node (1) [] {};
        \node (2) [normal, right=1.5cm of 1] {高频放大};
        \node (3) [normal, right=of 2] {混频};
        \node (4) [normal, right=of 3] {中频放大};
        \node (5) [normal, right=of 4] {检波};
        \node (6) [normal, right=of 5] {低频放大};
        \node (7) [normal, below=of 3] {本地振荡};

        \draw [-latex] (1) -- (2) node [midway, above] {传输线};
        \draw [-latex] (2) edge (3)
        (3) edge (4)
        (4) edge (5)
        (5) edge (6)
        (7) edge (3);
    \end{tikzpicture}
    \caption{超外差式接收机方框图}
\end{figure}

\subsection{通信的传输媒质}
\begin{itemize}
    \item 有线通信
          \begin{itemize}
              \item 双线对电缆 (双绞线; 频率低)
              \item 同轴电缆
              \item 光纤 (频率高, 衰减小)
          \end{itemize}
    \item 无线通信
          \begin{itemize}
              \item 地波
                    \begin{itemize}
                        \item 地面波 (电磁波沿地面绕射传输; 传播距离近, 频率低)
                        \item 空间波 (由直射波, 反射波组成; 视距范围内)
                    \end{itemize}
              \item 天波 (通过空中电离层的折射反射传播, 穿透电离层, 更高频率会携带水蒸气)
          \end{itemize}
\end{itemize}

\section{选频网络}
\begin{itemize}
    \item 振荡回路 (由 L, C 组成)
          \begin{itemize}
              \item 单振荡回路
              \item 耦合振荡回路
          \end{itemize}
    \item 滤波器
          \begin{itemize}
              \item LC 集中滤波器
              \item 石英晶体滤波器
              \item 陶瓷滤波器
              \item 声表面波滤波器
          \end{itemize}
\end{itemize}

\subsection{串联谐振回路}
\subsubsection{基本原理}
\begin{figure}[H]
    \centering
    \begin{tikzpicture}[circuit ee IEC, x=3cm, y=2cm, semithick, every info/.style={font=\normalsize}, large circuit symbols]
        \draw (0, 0) to [voltage source={info={$\dot{V}_s$}}] (0, 1);
        \draw (0, 1) to [inductor={info={$L$}}] (1, 1);
        \draw (1, 1) to [capacitor={info={$C$}}] (1, 0);
        \draw (1, 0) to [resistor={info={$R$}}] (0, 0);
    \end{tikzpicture}
    \caption{串联谐振回路}
\end{figure}

\textbf{阻抗}

\begin{equation}
    Z=R+\mathrm{j}\left(\omega L-\frac{1}{\omega C}\right)=|Z|e^{\mathrm{j}\varphi}.
\end{equation}

其中

\begin{equation}
    X=\left(\omega L-\frac{1}{\omega C}\right).
\end{equation}

\begin{equation}
    |Z|=\sqrt{R^2+X^2}=\sqrt{R^2+\left(\omega L-\frac{1}{\omega C}\right)^2}.
\end{equation}

\begin{equation}
    \varphi=\arctan\frac{X}{R}=\arctan\frac{\omega L-\dfrac{1}{\omega C}}{R}.
\end{equation}

\textbf{谐振条件}\quad $\omega=\omega_0=\dfrac{1}{\sqrt{LC}}$. $f_0=\dfrac{1}{2\pi\sqrt{LC}}$.

\begin{exampleprob}
    有一串联回路在频段 535kHz\textasciitilde 1605kHz 内工作. 现有两个可变电容器 A: 12pF\textasciitilde 100pF, B: 15pF\textasciitilde 450pF. 应该选用何种电容器?

    \begin{solution}
        电容器的可调电容值范围应尽可能宽, 使得回路在指定频段内可以达到串联谐振. 因为 $f=\dfrac{1}{2\pi\sqrt{LC}}$, 所以

        \begin{equation*}
            k=\frac{C_m}{C_n}=\left(\frac{f_m}{f_n}\right)^2=9.
        \end{equation*}

        A 电容器的比值 $k\approx 8.33$, B 电容器的比值 $k=30$, 故应该选用 B 电容器.
    \end{solution}
\end{exampleprob}

\textbf{特性阻抗}\quad 谐振时的感抗或容抗. $\rho=\sqrt{\dfrac{L}{C}}$.

\textbf{非谐振特性}

当 $\omega\neq\omega_0$ 时, $|Z|>R$.

\begin{itemize}
    \item $\omega>\omega_0$, $X>0$, 呈感性; $|\dot{V}_L|<|\dot{V}_C|$; $\varphi>0$, 电流滞后电压;
    \item $\omega<\omega_0$, $X<0$, 呈容性; $|\dot{V}_L|>|\dot{V}_C|$; $\varphi<0$, 电流超前电压.
\end{itemize}

\textbf{谐振特性}

\begin{itemize}
    \item $X=0$, $Z=R$, 呈纯阻性, $|Z|$ 达到最小值;
    \item $\varphi=0$, 电流与电压同相位;
    \item L 和 C 上的电压 $\dot{V}_{L0}$ 和 $\dot{V}_{C0}$ 大小相等, 相位相差 180°, $\dot{V}_s=\dot{V}_R$;
    \item $\dot{V}_{L0}=\mathrm{j}Q\dot{V}_s$, $\dot{V}_{C0}=-\mathrm{j}Q\dot{V}_s$, 其中 $Q=\dfrac{\omega_0L}{R}=\dfrac{1}{\omega_0CR}=\dfrac{\rho}{R}=\dfrac{1}{R}\sqrt{\dfrac{L}{C}}$ 为回路的\textbf{品质因数} ($Q$ 越高, 回路选择性越好).
\end{itemize}

\begin{figure}[H]
    \centering
    \begin{tikzpicture}
        % See: https://tex.stackexchange.com/a/569760
        \begin{axis}[
                domain=0:6.5,
                xmin=0, xmax=7,
                ymin=0, ymax=7,
                ticks=none,
                xlabel={$\omega$},
                ylabel={$I$}]
            \addplot [color=flatblue] {3/(sqrt(0.5^2+(\x-1/\x)^2))};
            \addplot [color=tea] {3/(sqrt(1^2+(\x-1/\x)^2))};
            \addplot [color=copper] {3/(sqrt(2^2+(\x-1/\x)^2))};
            \addlegendentry{$Q=4Q^*$};
            \addlegendentry{$Q=2Q^*$};
            \addlegendentry{$Q=Q^*$};
        \end{axis};
    \end{tikzpicture}
    \caption{外加电压为常数时, $Q$ (或 $R$) 对 $I$-$\omega$ 曲线的影响 (谐振曲线)}
\end{figure}

当 $\dot{V}_s$ 保持不变时, $|\dot{I}|$ 达到最大值.

\subsubsection{能量关系} \label{能量关系}

串联单振荡回路中, 只有 R \textit{消耗}外加电动势的能量; 电路进入稳定状态后, L 和 C 只\textit{储存和交换}能量.

\textbf{电阻}

\begin{equation}
    \overline{P}_R=\frac{1}{2}RI_{0m}^2.
\end{equation}

单个周期:

\begin{equation}
    W_R=\overline{P}_T=\frac{1}{2}\frac{RI_{0m}^2}{f_0}.
\end{equation}

\textbf{电容和电感}

设谐振时 $i=I_{0m}\sin\omega t$,

\begin{equation*}
    \begin{gathered}
        v_C=\frac{1}{C}\int i\mathrm{d}t=\frac{I_{0m}}{\omega C}\sin(\omega t-90^\circ)=-V_{Cm}\cos\omega t, \\
        v_L=L\frac{\mathrm{d}i}{\mathrm{d}t}=LI_{0m}\omega\cos\omega t=V_{Lm}\cos\omega t.\qquad\textrm{(显然}\ V_{Cm}=V_{Lm}\textrm{.)}
    \end{gathered}
\end{equation*}

\begin{equation}
    W_C=\frac{1}{2}Cv_C^2=\frac{1}{2}CV_{Cm}^2\cos^2\omega t.
\end{equation}

\begin{equation}
    W_L=\frac{1}{2}Li_L^2=\frac{1}{2}LI_{0m}^2\sin^2\omega t.
\end{equation}

因为 $Q=\dfrac{1}{R}\sqrt{\dfrac{L}{C}}$, 所以 $CV_{Cm}^2=LI_{0m}^2$.

\begin{equation}
    W=W_L+W_C=\frac{1}{2}LI_{0m}^2.
\end{equation}

\begin{equation}
    \frac{W_C+W_L}{W_R}=\frac{f_0L}{R}=\frac{Q}{2\pi}.
\end{equation}

\subsubsection{串联振荡回路的谐振曲线和通频带} \label{串联振荡回路的谐振曲线和通频带}

\begin{equation}
    \frac{\dot{I}}{\dot{I}_0}=\frac{R}{R+\mathrm{j}\left(\dfrac{\omega}{\omega_0}-\dfrac{\omega_0}{\omega}\right)}=\frac{1}{1+\mathrm{j}Q\left(\dfrac{\omega}{\omega_0}-\dfrac{\omega_0}{\omega}\right)}.
\end{equation}

\begin{equation} \label{eq:2.1 I/I_0}
    \frac{I}{I_0}=\frac{1}{\sqrt{1+\left(\dfrac{X}{R}\right)^2}}=\frac{1}{\sqrt{1+Q^2\left(\dfrac{\omega}{\omega_0}-\dfrac{\omega_0}{\omega}\right)^2}}.
\end{equation}

\textbf{失谐 (失调)}\quad $\Delta\omega=\omega-\omega_0$.

当 $\omega\approx\omega_0$ 时, 利用

\begin{equation*}
    \frac{\omega}{\omega_0}-\frac{\omega_0}{\omega}=\frac{\omega^2-\omega_0^2}{\omega_0\omega}=\left(\frac{\omega+\omega_0}{\omega}\right)\left(\frac{\omega-\omega_0}{\omega_0}\right)\approx 2\frac{\Delta\omega}{\omega_0},
\end{equation*}

\noindent 将 (\ref{eq:2.1 I/I_0}) 改写为\textit{小量失谐}情况下通用形式的\textbf{谐振特性方程式}

\begin{equation} \label{eq:2.1 I/I_0 xi}
    \frac{I}{I_0}\approx\frac{1}{\sqrt{1+\left(Q\dfrac{2\Delta\omega}{\omega_0}\right)^2}}=\frac{1}{\sqrt{1+\xi^2}}.
\end{equation}

\textbf{广义失谐 (一般失谐)}\quad $\xi=\dfrac{X}{R}=Q\dfrac{2\Delta\omega}{\omega_0}$.

\textbf{边界角频率 (频率)}\quad $\dfrac{I}{I_0}=\dfrac{1}{\sqrt{2}}$ 时的 $\omega_{1,2}$ ($f_{1,2}$).

此时回路损耗功率为谐振功率的一半 (半功率点). $\xi=\pm 1$.

\textbf{通频带}\quad $B_\omega=2\Delta\omega_{0.7}=\omega_2-\omega_1=\dfrac{\omega_0}{Q}$, $B=2\Delta f_{0.7}=f_2-f_1=\dfrac{f_0}{Q}$.

$Q$ 越高, 通频带越窄.

\textbf{有载品质因数 (考虑其他内阻)}

\begin{equation}
    Q_L=\frac{\omega_0L}{R+R_s+R_L}.
\end{equation}

若信号源变为恒流源 ($R_s$ 和 $V_s$ 可趋于无穷大), $Q_L$ 降为零.

\begin{figure}[H]
    \centering
    \begin{tikzpicture}
        % See: https://tex.stackexchange.com/a/569760
        \begin{axis}[
                domain=0:6.5,
                xmin=0, xmax=7,
                ymin=0, ymax=7,
                ticks=none,
                xlabel={$\omega$},
                ylabel={$I$}]
            \addplot [color=flatblue] {3/(sqrt(0.5^2+(\x-1/\x)^2))};
            \addplot [color=tea] {3/(sqrt(0.75^2+(\x-1/\x)^2))};
            \addplot [color=copper] {3/(sqrt(1^2+(\x-1/\x)^2))};
            \addplot [color=lavender] {1};
            \addlegendentry{$R_s=0$};
            \addlegendentry{$R_s=0.5R$};
            \addlegendentry{$R_s=R$};
            \addlegendentry{恒流源};
        \end{axis};
    \end{tikzpicture}
    \caption{信号源内阻对谐振曲线的影响}
\end{figure}

串联谐振回路通常适用于\textit{信号源内阻很小}, \textit{负载电阻也不大}的情况.

\subsubsection{串联振荡回路的相位特性曲线} \label{串联振荡回路的相位特性曲线}

回路电流的\textbf{相位特性曲线表达式} (认为 $\dot{V}_s$ 的相位为零):

\begin{equation} \label{eq:2.1 psi}
    \psi=-\varphi=-\arctan Q\left(\frac{\omega}{\omega_0}-\frac{\omega_0}{\omega}\right)\approx-\arctan Q\frac{2\Delta\omega}{\omega_0}=-\arctan\xi.
\end{equation}

\begin{figure}[H]
    % See: https://tex.stackexchange.com/a/272262/290833
    \begin{minipage}{.48\textwidth}
        \centering
        \begin{tikzpicture}
            \begin{axis}[%
                    domain=10:30,
                    xmin=10, xmax=30,
                    ymin=-2, ymax=2,
                    xlabel={$\omega$}, ylabel={$\psi$},
                    % See: https://tex.stackexchange.com/a/406697/290833
                    x tick label style={xshift=-0.6em},
                    extra x ticks={20},
                    extra x tick label={$\omega_0$},
                    xtick=\empty,
                    % See: https://tex.stackexchange.com/a/34958/290833
                    ytick={-0.7853981633974483, -1.5707963267948966, 0.7853981633974483, 1.5707963267948966},
                    yticklabels={$-\dfrac{\pi}{2}$, $-\dfrac{\pi}{4}$, $\dfrac{\pi}{4}$, $\dfrac{\pi}{2}$},
                    extra y ticks={0},
                    extra y tick label={0},
                    % See: https://tex.stackexchange.com/a/134086/290833
                    legend style={at={(1, 0.85)}}]
                \addplot[color=flatblue]{-rad(atan(10*(\x/20-20/\x)))};
                \addplot[color=copper]{-rad(atan(2.5*(\x/20-20/\x)))};
                \addplot[black, dashed]{pi/2};
                \addplot[black, dashed]{-pi/2};
                \addlegendentry{$Q_1$};
                \addlegendentry{$Q_2$}
            \end{axis};
        \end{tikzpicture}
        \caption{串联振荡回路的相位特性曲线 ($Q_1>Q_2$)}
    \end{minipage}
    \begin{minipage}{.48\textwidth}
        \centering
        \begin{tikzpicture}
            \begin{axis}[%
                    domain=-8:8,
                    xmin=-8.5, xmax=8.5,
                    ymin=-2, ymax=2,
                    xlabel={$\xi$}, ylabel={$\psi$},
                    % See: https://tex.stackexchange.com/a/34958/290833
                    ytick={-0.7853981633974483, -1.5707963267948966, 0, 0.7853981633974483, 1.5707963267948966},
                    yticklabels={$-\dfrac{\pi}{2}$, $-\dfrac{\pi}{4}$, 0, $\dfrac{\pi}{4}$, $\dfrac{\pi}{2}$}]
                \addplot[black]{-rad(atan(\x))};
                \addplot[black, dashed]{pi/2};
                \addplot[black, dashed]{-pi/2};
            \end{axis};
        \end{tikzpicture}
        \caption{串联振荡回路通用相位特性}
    \end{minipage}
\end{figure}

\begin{exampleprob}
    设给定串联谐振回路的 $f_0=1$ MHz, $Q_0=50$. 若输出电流超前信号源电压相位 $45^\circ$, 试求:

    \begin{enumerate}
        \item 信号源频率 $f$; 输出电流相对于谐振时衰减多少分贝;
        \item 现要在回路中再串联一个元件使回路谐振, 该元件的种类及其参数满足的表达式.
    \end{enumerate}

    \begin{solution}[1]
        \begin{enumerate}
            \item 根据 (\ref{eq:2.1 psi}),

                  \begin{equation*}
                      45^\circ=-\arctan Q_0\left(\frac{f}{f_0}-\frac{f_0}{f}\right),
                  \end{equation*}

                  解得 $f\approx 990$ kHz.

                  根据 (\ref{eq:2.1 I/I_0 xi}),
                  \question[电流衰减]
                  \begin{equation*}
                      20\lg\left|\frac{I}{I_0}\right|=20\lg\left|\frac{1}{\sqrt{1+(-\tan 45^\circ)^2}}\right|=-3\ \mathrm{dB}.
                  \end{equation*}
            \item 显然此时回路呈容性, 需要串联一电感 $L^*$ 满足

                  \begin{equation*}
                      4\pi^2 f_0(L+L^*)=\frac{1}{f_0 C}.
                  \end{equation*}
        \end{enumerate}
    \end{solution}

    \begin{solution}[2]
        此时广义失谐为 $\xi=-\arctan 45^\circ=-1$, 说明回路实际频率即为\textit{边界频率的较低值}, 也即 $f=f_0-\Delta f$. 而根据 (\ref{eq:2.1 psi}) 可以求出 $\Delta f$, 继而可以求得 $f$.
    \end{solution}
\end{exampleprob}

\subsection{并联谐振回路} \label{并联谐振回路}
\subsubsection{基本原理及特性}

\begin{figure}[H]
    \centering
    \begin{tikzpicture}[%
            circuit ee IEC,
            x=2cm, y=4cm,
            semithick,
            every info/.style={font=\large},
            huge circuit symbols]
        \draw (0, 0) node [above right, xshift=1em] {$-$} to [current source={direction info={info={$\dot{I}_s$}}}] node [right, xshift=1em] {$\dot{V}$} (0, 1) node [below right, xshift=1em] {$+$};
        \draw (1, 1) to [capacitor={info={$C$}}] (1, 0);
        \draw (2, 1) to [%
            inductor={near start, info={$L$}},
            resistor={near end, info={$R$}}] (2, 0);
        \draw (0, 1) -- (2, 1) (2, 0) -- (0, 0);
    \end{tikzpicture}
    \caption{并联谐振回路}
\end{figure}

\begin{subequations}
    \begin{align}
                                                & Z=\frac{(R+\mathrm{j}\omega L)\dfrac{1}{\mathrm{j}\omega C}}{(R+\mathrm{j}\omega L)+\dfrac{1}{\mathrm{j}\omega C}}                                              \\
        \xRightarrow{{\color{red} \omega L>>R}} & Z\approx\frac{\dfrac{L}{C}}{R+\mathrm{j}\left(\omega L-\dfrac{1}{\omega C}\right)}=\frac{1}{\dfrac{CR}{L}+\mathrm{j}\left(\omega C-\dfrac{1}{\omega L}\right)}.
    \end{align}
\end{subequations}

\begin{equation}
    \dot{V}=\dot{I}_sZ=\frac{\dot{I}_s}{\dfrac{CR}{L}+\mathrm{j}\left(\omega C-\dfrac{1}{\omega L}\right)}.
\end{equation}

\begin{equation}
    Y=\frac{1}{Z}=G+\mathrm{j}B=\frac{CR}{L}+\mathrm{j}\left(\omega C-\frac{1}{\omega L}\right).
\end{equation}

\textbf{谐振条件}\quad $B=\omega_pC-\dfrac{1}{\omega_p L}=0$.

\begin{equation}
    \omega_p=\frac{1}{\sqrt{LC}}.
\end{equation}

不满足 $\omega L>>R$ 时, 谐振频率的精确解:

\begin{equation}
    \omega_p=\sqrt{\frac{1}{LC}-\frac{R^2}{L^2}}.
\end{equation}

\textbf{谐振特性}

\begin{itemize}
    \item $\dot{V}_0$ 与 $\dot{I}_s$ 同相, $V_0=V_m=\dfrac{I_sL}{CR}$;
    \item 谐振时, 回路阻抗为纯电阻且达到最大值;
    \item 电容支路和电感支路电流大小相等, 相位相差 180°, 各支路电流为总电流的 $Q_p$ 倍
          \begin{equation}
              \dot{I}_{Cp}=\mathrm{j}Q_p\dot{I}_s,\quad\dot{I}_{Lp}=-\mathrm{j}Q_p\dot{I}_s;
          \end{equation}
    \item 考虑 $R$ 时, 在 $Q_p$ 很大时, 可以认为 $Z_p$ 为\textit{纯电阻时恰好为最大值}; 此时 $\dot{I}_L$ 的相位落后 $\dot{V}_0$ 小于 90°.
\end{itemize}

\textbf{非谐振特性}

\begin{itemize}
    \item $\omega>\omega_p$, 呈容性;
    \item $\omega<\omega_p$, 呈感性.
\end{itemize}

\textbf{品质因数}

\begin{equation}
    Q_p=\frac{\omega_p}{L}=\frac{1}{\omega_pCR}=\frac{1}{R}\sqrt{\frac{L}{C}}.
\end{equation}

\textbf{谐振电阻}

\begin{equation}
    R_p=\frac{\dot{V}_0}{\dot{I}_s}=\frac{L}{CR}=\frac{\omega_p^2L^2}{R}=Q_p\omega_pL=Q_p\frac{1}{\omega_pC}=\frac{1}{R\omega_p^2C^2}.
\end{equation}

谐振时, 并联振荡回路的谐振电阻等于电感支路或电容支路电抗值的 $Q_p$ 倍.

\subsubsection{并联振荡回路的谐振曲线, 相位特性曲线和通频带}

类比串联振荡回路:

\begin{equation}
    \frac{\dot{V}}{\dot{V}_0}=\frac{1}{1+\mathrm{j}Q_p\left(\dfrac{\omega}{\omega_p}-\dfrac{\omega_p}{\omega}\right)}.
\end{equation}

\begin{subequations}
    \begin{equation}
        \frac{V}{V_0}=\frac{1}{\sqrt{1+Q_p^2\left(\dfrac{\omega}{\omega_p}-\dfrac{\omega_p}{\omega}\right)^2}}.
    \end{equation}
    \begin{equation}
        \frac{V}{V_0}=\frac{1}{\sqrt{1+\xi^2}}.
    \end{equation}
\end{subequations}


\begin{subequations}
    \begin{equation}
        \psi=-\arctan Q_p\left(\frac{\omega}{\omega_p}-\frac{\omega_p}{\omega}\right).
    \end{equation}
    \begin{equation}
        \psi=-\arctan\xi.
    \end{equation}
\end{subequations}

\textbf{绝对通频带}\quad $B_\omega=2\Delta\omega_{0.7}=\dfrac{\omega_p}{Q_p}$, $B=2\Delta f_{0.7}=\dfrac{f_p}{Q_p}$.

\textbf{相对通频带}\quad $\dfrac{2\Delta\omega_{0.7}}{\omega_p}=\dfrac{2\Delta f_{0.7}}{f_p}=\dfrac{1}{Q_p}$.

\subsubsection{信号源内阻和负载内阻的影响}

\question[并联振荡回路等效电路将 $R_p$ 并联在电路中]
\begin{figure}[H]
    \centering
    \begin{tikzpicture}[%
            circuit ee IEC,
            x=1.6cm, y=2.75cm,
            semithick,
            every info/.style={font=\large},
            huge circuit symbols]
        \draw (0, 0) node [above right, xshift=1em] {$-$} to [current source={direction info={info={$\dot{I}_s$}}}] node [right, xshift=1em] {$\dot{V}$} (0, 1) node [below right, xshift=1em] {$+$};
        \draw (1, 1) to [resistor={info={$R_s$}}] (1, 0);
        \draw (2, 1) to [capacitor={info={$C$}}] (2, 0);
        \draw (3, 1) to [inductor={info={$L$}}] (3, 0);
        \draw (4, 1) to [resistor={info={$R_p$}}] (4, 0);
        \draw (5, 1) to [resistor={info={$R_L$}}] (5, 0);
        \draw (0, 1) -- (5, 1) (5, 0) -- (0, 0);
    \end{tikzpicture}
    \caption{考虑 $R_s$ 和 $R_L$ 后的并联振荡回路的等效电路}
\end{figure}

\question[并联谐振的有载品质因数]

\textbf{品质因数}

\begin{subequations}
    \begin{equation}
        Q_L=\frac{1}{\omega_pL(G_p+G_s+G_L)}=\frac{1}{\omega_pL\left(\dfrac{1}{R_p}+\dfrac{1}{R_s}+\dfrac{1}{R_L}\right)}.
    \end{equation}
    \begin{equation}
        Q_L=\frac{Q_p}{1+\dfrac{R_p}{R_s}+\dfrac{R_p}{R_L}}.
    \end{equation}
\end{subequations}

并联更多的电阻, 或降低 $R_s$, $R_L$ 的阻值, 回路总电导增大, 品质因数降低, 通频带变宽, 回路选择性更差.

\begin{figure}[H]
    \begin{subfigure}{.48\textwidth}
        \centering
        \begin{tikzpicture}
            \begin{axis}[%
                    domain=0:2.5,
                    xmin=0, xmax=2.99,
                    ymin=0, ymax=1.3,
                    xlabel={$\omega$/$\mathrm{s}^{-1}$}, ylabel={$V_C$/V},
                    extra y ticks={0}, extra y tick label={$0$}]
                \addplot[black]{1/sqrt((1+0.1*0/(x^2+0.01))^2+0^2*x^2*(1-1/(x^2+0.01))^2)};
                \addplot[color=flatblue]{1/sqrt((1+0.1*2/(x^2+0.01))^2+2^2*x^2*(1-1/(x^2+0.01))^2)};
                \addplot[color=tea]{1/sqrt((1+0.1*5/(x^2+0.01))^2+5^2*x^2*(1-1/(x^2+0.01))^2)};
                \addplot[color=copper]{1/sqrt((1+0.1*10/(x^2+0.01))^2+10^2*x^2*(1-1/(x^2+0.01))^2)};
                \addplot[black, dashed] coordinates {(1, 0) (1, 1.2)};
                \addlegendentry{$R_s=0$};
                \addlegendentry{$R_s=2\Omega$};
                \addlegendentry{$R_s=5\Omega$};
                \addlegendentry{$R_s=10\Omega$};
            \end{axis};
        \end{tikzpicture}
        \caption{信号源电动势保持不变}
    \end{subfigure}
    \begin{subfigure}{.48\textwidth}
        \centering
        \begin{tikzpicture}
            \begin{axis}[%
                    domain=0:2.5,
                    xmin=0, xmax=2.99,
                    ymin=0, ymax=1.3,
                    xlabel={$\omega$/$\mathrm{s}^{-1}$}, ylabel={$V_C$/V},
                    extra y ticks={0}, extra y tick label={$0$}]
                \addplot[black]{(1+0.1*0)/sqrt((1+0.1*0/(x^2+0.01))^2+0^2*x^2*(1-1/(x^2+0.01))^2)};
                \addplot[color=flatblue]{(1+0.1*2)/sqrt((1+0.1*2/(x^2+0.01))^2+2^2*x^2*(1-1/(x^2+0.01))^2)};
                \addplot[color=tea]{(1+0.1*5)/sqrt((1+0.1*5/(x^2+0.01))^2+5^2*x^2*(1-1/(x^2+0.01))^2)};
                \addplot[color=copper]{(1+0.1*10)/sqrt((1+0.1*10/(x^2+0.01))^2+10^2*x^2*(1-1/(x^2+0.01))^2)};
                \addplot[color=lavender]{(1/10.1)/sqrt((0.1/(x^2+0.01))^2+x^2*(1-1/(x^2+0.01))^2)};
                \addplot[black, dashed] coordinates {(1, 0) (1, 1.2)};
                \addlegendentry{$R_s=0$};
                \addlegendentry{$R_s=2\Omega$};
                \addlegendentry{$R_s=5\Omega$};
                \addlegendentry{$R_s=10\Omega$};
                \addlegendentry{$I_s=\dfrac{10}{101}$A};
            \end{axis};
        \end{tikzpicture}
        \caption{调整信号源电动势使 $V_C=1\mathrm{V}$}
    \end{subfigure}
    \caption{电源内阻对并联谐振曲线的影响 ($L=1\mathrm{H},R=0.1\Omega,C=1\mathrm{F},V_s=1\mathrm{V}$)}
\end{figure}

为获得优良的选择性, 信号源内阻低时, 应采用串联振荡回路; 信号源内阻高时, 应采用并联振荡回路.

\subsubsection{低 Q 值 (Q<10) 的并联谐振回路}

假设工作频率保持不变, 若要得到谐振 (满足 $Z_p$ 最大且 $Z_p$ 为纯电阻):

\begin{itemize}
    \item 电阻集中在电感支路, 电容支路的电阻为零, 则改变 C;
    \item 电阻集中在电容支路, 电感支路的电阻为零, 则改变 L.
\end{itemize}

\newpage

\section{附录} \label{附录}
\subsection{图片索引} \label{图片索引}
% See: https://tex.stackexchange.com/a/55088
\makeatletter
\@starttoc{lof}
\makeatother

\subsection{表格索引} \label{表格索引}
\makeatletter
\@starttoc{lot}
\makeatother

\subsection{例题索引} \label{例题索引}
\listofexampleprobs

\subsection{疑问索引} \label{疑问索引}
\listofquestions

\subsection{更新日志} \label{更新日志}
\addcontentsline{toc}{section}{更新日志}
\subsubsection*{1.2.0 (2023-2-26)}
\begin{itemize}
    \item 增加 \ref{能量关系}, \ref{串联振荡回路的相位特性曲线} 至 \ref{并联谐振回路} 的部分, 及其他内容;
    \item 增加两道例题;
    \item 修改图片的位置选项为 \texttt{H};
    \item 增加 \hyperref[附录]{附录}, 并增加 \hyperref[图片索引]{图片索引}, \hyperref[表格索引]{表格索引}, \hyperref[例题索引]{例题索引}, \hyperref[疑问索引]{疑问索引} 的部分;
    \item 其他细微改动.
\end{itemize}

\subsubsection*{1.1.0 (2023-2-23)}
\begin{itemize}
    \item 增加 \ref{绪论} 至 \ref{串联振荡回路的谐振曲线和通频带} 的部分;
    \item 增加 \hyperref[更新日志]{更新日志}.
\end{itemize}

\subsubsection*{1.0.0 (2023-2-22)}
\begin{itemize}
    \item 增加 \hyperref[课程信息]{课程信息} 的部分.
\end{itemize}

\end{document}
