\subsection{系统函数 (网络函数) \texorpdfstring{$H(s)$}{H(s)}} \label{4 系统函数 (网络函数) H(s)}
系统的零状态响应的拉普拉斯变换与激励的拉普拉斯变换之比. 系统函数\textit{与系统初始状态无关}.
\begin{equation}
    H(s)=\frac{\mathcal{L}[r_{zs}(t)]}{\mathcal{L}[e(t)]}.
\end{equation}

LTI 系统中, 若激励, 零状态响应, 冲激响应分别为 $e(t)$, $r(t)$, $h(t)$. 显然有
\begin{equation}
    r(t)=e(t)*h(t).
\end{equation}
两边取拉普拉斯变换, 利用时域卷积定理 (\ref{eq:4.2 conv t}), 得到
\begin{equation*}
    R(s)=E(s)H(s)
\end{equation*}
或
\begin{equation}
    H(s)=\frac{R(s)}{E(s)}=\mathcal{L}[h(t)].
\end{equation}

\textbf{求解系统函数}
\begin{exampleprob}
    已知系统的微分方程为 $r''+3r'+2r=e'(t)+3e(t)$, 求系统函数 $H(s)$ 和单位冲激响应 $h(t)$.

    \begin{solution}
        对题设微分方程两边作拉普拉斯变换, 利用微分性质 (\ref{eq:4.2 diff t}) 且\textit{初始条件为零}, 得到
        \begin{equation*}
            (s^2+3s+2)R(s)=(s+3)E(s).
        \end{equation*}

        因此系统函数为
        \begin{equation*}
            H(s)=\frac{R(s)}{E(s)}=\frac{2}{s+1}-\frac{1}{s+2}.
        \end{equation*}
        单位冲激响应为
        \begin{equation}
            h(t)=\mathcal{L}^{-1}[H(s)]=(2e^{-t}-e^{-2t})u(t).
        \end{equation}
    \end{solution}
\end{exampleprob}

\begin{exampleprob}
    给定一系统的模拟框图, 求系统函数 $H(s)$.
    \begin{figure}[H]
        \centering
        \includegraphics[width=.8\textwidth]{./images/signals_eprob_4_5_system.pdf}
        \caption{例题 \theexampleprob 图}
    \end{figure}

    \begin{solution}
        设第一个加法器的输出为 $Q(s)$, 则对两个加法器列写方程
        \begin{gather*}
            Q(s)=X(s)-3s^{-1}Q(s)-2s^{-2}Q(s), \\
            Y(s)=3s^{-2}Q(s)+s^{-1}Q(s).
        \end{gather*}

        联立整理, 得
        \begin{equation*}
            H(s)=\frac{Y(s)}{X(s)}=\frac{s+3}{s^2+3s+2}.
        \end{equation*}
    \end{solution}
\end{exampleprob}

\textbf{利用系统函数求解网络响应}

有如下两种方法:
\begin{itemize}
    \item 取 $H(s)$ 作逆变换再与激励 $e(t)$ 卷积;
    \item 将 $R(s)=H(s)E(s)$ 部分分式展开, 再逐项求出逆变换.
\end{itemize}
