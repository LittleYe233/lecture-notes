\subsection{信号的分解}

\subsubsection{直流分量与交流分量}

\rmg
\begin{gather}
    f(t)=f_D+f_A(t). \\
    P=\frac{1}{T}\int_{0}^{T}f^2(t)\mathrm{d}t=f_D^2+\frac{1}{T}\int_{0}^{T}f_A^2(t)\mathrm{d}t.\qquad \left(\int_{0}^{T}f_Df_A(t)\mathrm{d}t=0.\right)
\end{gather}

一个信号的平均功率等于直流功率与交流功率之和.

\subsubsection{偶分量与奇分量}

\rmg
\begin{gather}
    f_e(t)=f_e(-t). \\
    f_o(t)=-f_o(-t). \\
    f(t)=f_e(t)+f_o(t)=\frac{1}{2}(f(t)+f(-t))+\frac{1}{2}(f(t)-f(-t)).
\end{gather}

一个信号的平均功率等于偶分量功率与奇分量功率之和.

\subsubsection{脉冲分量}

\textbf{分解为矩形窄脉冲分量}\quad (极限情况为\textit{冲激信号的叠加}.)
\begin{equation}
    f(t)=\int_{-\infty}^{\infty}f(\tau)\delta(t-\tau)\mathrm{d}\tau.
\end{equation}

这亦是卷积的性质之一. 参见式 (\ref{eq:2.6 f(t)*delta(t)}).

\textbf{分解为阶跃信号分量}
\begin{equation}
    f(t)=f(0)u(t)+\int_{0}^{\infty}\frac{\mathrm{d}f(\tau)}{\mathrm{d}\tau}u(t-\tau)\mathrm{d}\tau.
\end{equation}

\subsubsection{实部分量与虚部分量}

\rmg
\begin{gather}
    f_r(t)=\frac{1}{2}(f(t)+f^*(t)). \\
    f_i(t)=\frac{1}{2}(f(t)-f^*(t)). \\
    f(t)=f_r(t)+\mathrm{j}f_i(t).
\end{gather}

\subsubsection{正交函数分量}

使用\textit{正交函数集}表示信号, 例如正交的正余弦函数集 (傅里叶变换).
