\subsection{由系统函数零, 极点分布决定频响特性} \label{4 由系统函数零, 极点分布决定频响特性}
\textbf{系统的频响特性}\quad 系统在\textit{正弦信号}激励下\textit{稳态响应} (参见 \ref{4.10 线性稳定系统的瞬态响应和稳定响应} 小节) 随信号频率的变化情况 (幅度和相位).

对于稳定系统, 可以求得其在正弦激励 $E_m\sin(\omega t)$ 下的稳态响应 (去除随时间增加指数衰减的响应分量) 为
\begin{equation} \label{eq:4.7 r ss}
    r_{ss}(t)=E_m|H(\mathrm{j}\omega)|\sin(\omega t+\varphi).
\end{equation}
而 $\varphi$ 即为 $H(\mathrm{j}\omega)$ 的相位. 因此, 将系统的频率响应特性定义为
\begin{equation}
    H(s)|_{s=\mathrm{j}\omega}=H(\mathrm{j}\omega)=|H(\mathrm{j}\omega)|e^{\mathrm{j}\varphi(\omega)}.
\end{equation}
其中 $|H(\mathrm{j}\omega)|$ 是\textbf{幅频响应特性}, $\varphi(\omega)$ 是相频响应特性 (相移特性).

系统频响特性可以表示为
\begin{equation} \label{eq:4.7 H j omega}
    H(\mathrm{j}\omega)=\frac{\displaystyle K\prod_{j=1}^{m}(\mathrm{j}\omega-z_j)}{\displaystyle\prod_{i=1}^{n}(\mathrm{j}\omega-p_i)}.
\end{equation}

一般 $K$ 对系统的频响特性没有影响, 但在一些题型中会给定 $H(\infty)$ 可用以待定该系数.

\textbf{零点矢量}\quad $\mathrm{j}\omega-z_j=N_je^{\mathrm{j}\psi_j}$.

\textbf{极点矢量}\quad $\mathrm{j}\omega-p_i=M_ie^{\mathrm{j}\theta_i}$.

引入零点矢量和极点矢量后, 系统的频响特性可以表示为
\begin{equation}
    \left\{\begin{aligned}
         & |H(\mathrm{j}\omega)|=K\frac{N_1N_2\cdots N_m}{M_1M_2\cdots M_n},                  \\
         & \varphi(\omega)=(\psi_1+\psi_2+\cdots+\psi_m)-(\theta_1+\theta_2+\cdots+\theta_n).
    \end{aligned}\right.
\end{equation}
