\subsection{由系统函数零, 极点分布决定时域特性} \label{4 由系统函数零, 极点分布决定时域特性}
\subsubsection{系统函数零, 极点分布与单位冲激响应波形特征的对应}
对于 $H(s)$ 的一阶极点, 其与 $h(t)$ 的波形特征有如下的对应关系. 对于更高阶的极点, 以二阶为例, $h(t)$ 需要相应乘 $t$ 的指数幂 (二阶对应 $t$ 的一次方), 会使波形特征发生其他的变化.

\renewcommand{\arraystretch}{1.5}
\begin{longtable}{@{\extracolsep{\fill}}cc@{}}
    \caption{\texorpdfstring{$H(s)$}{H(s)} 典型一阶和二阶极点与 \texorpdfstring{$h(t)$}{h(t)} 波形特征的对应关系 (定性)}
    \label{tab:4.6 H(s) 典型一阶和二阶极点与 h(t) 波形特征的对应关系 (定性)}                      \\

    \hline \textbf{$H(s)$ 极点位置} & \textbf{$h(t)$ 波形特征}                       \\ \hline
    \endfirsthead

    \multicolumn{2}{r}{\em \footnotesize cont'd}                             \\
    \hline \textbf{$H(s)$ 极点位置} & \textbf{$h(t)$ 波形特征}                       \\ \hline
    \endhead

    \hline
    \endlastfoot

    s 平面左半平面                    & 包含指数衰减因子                                   \\
    s 平面右半平面                    & 包含指数增长因子                                   \\
    s 平面虚轴上 (不含原点)              & 等幅振荡 (一阶极点), 增长振荡 (二阶极点)                   \\
    s 平面实轴上 (不含原点)              & 包含指数因子但不振荡                                 \\
    s 平面原点                      & 与 $h(t)$ 等值 (一阶极点), 乘 $t$ 的阶数减 1 次方 (高阶极点)
\end{longtable}
\renewcommand{\arraystretch}{1}

\renewcommand{\arraystretch}{1.5}
\begin{longtable}{@{\extracolsep{3em}}cccc@{}}
    \caption{\texorpdfstring{$H(s)$}{H(s)} 典型一阶和二阶极点与 \texorpdfstring{$h(t)$}{h(t)} 波形特征的对应关系 (定量)}
    \label{tab:4.6 H(s) 典型一阶和二阶极点与 h(t) 波形特征的对应关系 (定量)}                                                                                                        \\

    \hline \textbf{$H(s)$}                                   & \textbf{$H(s)$ 极点 $p_i$}             & \textbf{$h(t)$}                   & \textbf{$h(t)$ 波形特征} \\ \hline
    \endfirsthead

    \multicolumn{4}{r}{\em \footnotesize cont'd}                                                                                                               \\
    \hline \textbf{$H(s)$}                                   & \textbf{$H(s)$ 极点 $p_i$}             & \textbf{$h(t)$}                   & \textbf{$h(t)$ 波形特征} \\ \hline
    \endhead

    \hline
    \endlastfoot

    $\dfrac{1}{s}$                                           & $p_1=0$                              & $u(t)$                            & 直流响应                 \\
    $\dfrac{1}{s+\alpha} \quad (\alpha>0)$                   & $p_1=-\alpha$                        & $e^{-\alpha t}u(t)$               & 指数衰减                 \\
    $\dfrac{1}{s+\alpha} \quad (\alpha<0)$                   & $p_1=-\alpha$                        & $e^{-\alpha t}u(t)$               & 指数增长                 \\
    $\dfrac{\omega}{s^2+\omega^2}$                           & $p_{1,2}=\pm\mathrm{j}\omega$        & $\sin(\omega t)u(t)$              & 等幅振荡                 \\
    $\dfrac{\omega}{(s+\alpha)^2+\omega^2} \quad (\alpha>0)$ & $p_{1,2}=-\alpha\pm\mathrm{j}\omega$ & $e^{-\alpha t}\sin(\omega t)u(t)$ & 指数衰减振荡               \\
    $\dfrac{\omega}{(s+\alpha)^2+\omega^2} \quad (\alpha<0)$ & $p_{1,2}=-\alpha\pm\mathrm{j}\omega$ & $e^{-\alpha t}\sin(\omega t)u(t)$ & 指数增长振荡               \\
    $\dfrac{1}{s^2}$                                         & $p_1=0$                              & $tu(t)$                           & 线性增长                 \\
    $\dfrac{1}{(s+\alpha)^2} \quad (\alpha>0)$               & $p_1=-\alpha$                        & $te^{-\alpha t}u(t)$              & 先增长后衰减               \\
    $\dfrac{2\omega s}{(s^2+\omega^2)^2}$                    & $p_{1,2}=\pm\mathrm{j}\omega$        & $t\sin(\omega t)u(t)$             & 线性增长振荡
\end{longtable}
\renewcommand{\arraystretch}{1}

\subsubsection{系统函数, 激励极点分布与自由响应, 强迫响应特征的对应}
s 域中, 系统响应 $R(s)$, 激励 $E(s)$ 与系统函数 $H(s)$ 满足如下关系:
\begin{equation*}
    R(s)=H(s)E(s).
\end{equation*}

显然 $R(s)$ 的零点和极点由 $H(s)$ 和 $E(s)$ 的零点和极点决定. 不妨设 $H(s)$ 和 $E(s)$ 的极点分别为 $\{p_i\}$ 和 $\{p_k\}$ (其中 $1\leq i\leq n$, $1\leq k\leq v$). 假设 $R(s)$ 无高阶极点, 且 $H(s)$ 和 $E(s)$ 没有相同极点, 那么对 $R(s)$ 部分分式展开, 得到
\begin{equation*}
    R(s)=\sum_{i=1}^{n}\frac{K_i}{s-p_i}+\sum_{k=1}^{v}\frac{K_k}{s-p_k}.
\end{equation*}
则系统响应的时域表示式为
\begin{equation*}
    r(t)=\sum_{i=1}^{n}K_ie^{p_it}+\sum_{k=1}^{v}K_ke_{p_kt}.
\end{equation*}

系统函数的极点所形成的分量 ($\displaystyle\sum_{i=1}^{n}K_ie^{p_it}$) 为\textbf{自由响应}, 激励函数的极点所形成的分量 ($\displaystyle\sum_{k=1}^{v}K_ke_{p_kt}$) 为\textbf{强迫响应}. 自由响应和强迫响应的\textit{时间函数的形式}只分别取决于 $H(s)$ 和 $E(s)$, 但是二者的\textit{幅度和相位}与 $H(s)$ 和 $E(s)$ 都有关.

若 $H(s)$ 和 $E(s)$ 存在零极点相消, 则响应中不会出现对应的时间分量. 一些题型会要求最终的响应不包含特定时间分量 (例如, 不包含正弦稳态分量) 从而求出特定参数; 或根据时域响应反推系统的特征根, 对照系统函数发现存在零极点相消的情况.

\begin{exampleprob}
    已知一 LTI 系统, 当输入为 $x(t)=Eu(t)$ 时, 零状态响应和零输入响应分别为
    \begin{gather*}
        y_{zs}(t)=E\left(\frac{1}{6}-\frac{1}{2}e^{-2t}+\frac{1}{3}e^{-3t}\right)u(t), \\
        y_{zi}(t)=3e^{-t}-5e^{-2t}+2e^{-3t}.
    \end{gather*}
    试求系统函数 $H(s)$ 及系统的微分方程.

    \begin{solution}
        系统零状态响应的拉普拉斯变换为
        \begin{equation*}
            Y_{zs}(s)=E\left(\frac{1}{6s}-\frac{1}{2(s+2)}+\frac{1}{3(s+3)}\right)=\frac{E}{s(s+2)(s+3)}.
        \end{equation*}

        激励的拉普拉斯变换为
        \begin{equation*}
            X(s)=\frac{E}{s}.
        \end{equation*}

        所以, 系统函数为
        \begin{equation*}
            H(s)=\frac{Y_{zs}(s)}{X(s)}=\frac{1}{s^2+5s+6}.
        \end{equation*}

        由零输入响应可知, 系统有 3 个特征根 $\alpha_{1,2,3}=-1,-2,-3$, 则系统函数应该有相同的 3 个极点 (\textit{零输入响应与自由响应的形式相同}). 但极点 $s=-1$ 在系统函数的分母中未出现, 则说明在此处发生零极点相消. 因此实际的系统函数为
        \begin{equation*}
            H(s)=\frac{\color{red} s+1}{{\color{red} (s+1)}(s+2)(s+3)}=\frac{s+1}{s^3+6s^2+11s+6}.
        \end{equation*}

        \question[系统微分方程与系统函数的关系]
        系统的三阶微分方程为
        \begin{equation*}
            y'''(t)+6y''(t)+11y'(t)+6y(t)=x'(t)+x(t).
        \end{equation*}
    \end{solution}
\end{exampleprob}
