\subsection{调制与解调} \label{5 调制与解调}
\textbf{调制}\quad 将信号频谱搬移到更高频的范围.

\textbf{频谱搬移}\quad 见公式 (\ref{eq:3.6 freq move cos}), (\ref{eq:3.6 freq move sin}).

设原始信号 $g(t)$ 的频谱 $G(\omega)$ 的范围为 $-\omega_m\sim\omega_m$. 对其施以调制信号 $\cos(\omega_0 t)$, 则调制后信号 $g(t)\cos(\omega_0 t)$ 的频谱集中在 $\pm\omega_0$ 附近, 带宽仍为 $2\omega_m$, 幅度降低一半.

\textbf{同步解调}\quad 对调制信号乘以\textit{相同频率和相位}的信号, 再使其通过低通滤波器.

继续对调制后信号调制一次, 则此时输出信号的频谱集中在 $\pm 2\omega_0$ 和 $0$ 附近, 其中前者幅度变为原信号频谱的 $\dfrac{1}{4}$, 后者变为一半. 此时将其通过截止频率 $\omega_c$ 满足 $\omega_m<\omega_c<2\omega_0-\omega_m$ 的低通滤波器即可取出原信号 $g(t)$.

同步解调的缺点是解调器需要生成与本地载波相同的信号, 不易于设计电路.

\textbf{调幅}\quad 在发射信号中叠加一定强度的载波信号 $A\cos(\omega_0 t)$, 合成出实际发射信号 $[A+g(t)]\cos(\omega_0 t)$. 若 $A$ 足够大使得全时段内 $A+g(t)>0$, 则实际发射信号的包络即为 $A+g(t)$. 因此\textit{无需本地载波信号}即可恢复发射信号.

\begin{exampleprob} \ref{}
    一倒频系统如图 (a) 所示, 激励限带信号 $f(t)$ 的频谱如图 (b) 所示. 已知两个滤波器的频率响应分别为
    \begin{equation*}
        H_1(\mathrm{j}\omega)=\begin{cases}
            K, & \quad |\omega|>\omega_c \\
            0; & \quad |\omega|<\omega_c
        \end{cases} \qquad
        H_2(\mathrm{j}\omega)=\begin{cases}
            0, & \quad |\omega|>\omega_c \\
            K. & \quad |\omega|<\omega_c
        \end{cases}
    \end{equation*}
    试绘出 $f(t)$ 通过系统时在 A, B, C, D 各点的频谱.

    \begin{solution}

    \end{solution}
\end{exampleprob}

\textbf{非线性调制} (例如调频, 调相.) 调制过程中频率或相位的变化可以看作载波角度的变化.
