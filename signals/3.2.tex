\subsection{典型周期信号的傅里叶级数} \label{3 典型周期信号的傅里叶级数}
\subsubsection{周期矩形脉冲信号}
设 $f(t)$ 脉冲宽度 $\tau$, 相邻两个脉冲中心间隔 $T$, 脉冲幅度 $E$. 则 $f(t)$ 的傅里叶级数可表示为
\begin{equation}
    f(t)=\frac{E\tau}{T}+\frac{2E\tau}{T}\sum_{n=1}^{\infty}\mathrm{Sa}\left(\frac{n\omega\tau}{2}\right)\cos(n\omega t)
\end{equation}
或
\begin{equation}
    f(t)=\frac{E\tau}{T}\sum_{-\infty}^{\infty}\mathrm{Sa}\left(\frac{n\omega\tau}{2}\right)e^{\mathrm{j}n\omega t}.
\end{equation}

在允许失真的条件下, 可以认为信号的频带范围为 $0\leq \omega\leq \dfrac{2\pi}{\tau}$.
\begin{equation} \label{eq:3.2 periodic rect signal Bf}
    B_f=\frac{1}{\tau}.
\end{equation}

\subsubsection{对称方波信号}
\srmg
\begin{equation}
    f(t)=\frac{2E}{\pi}\sum_{n=1}^{\infty}\frac{1}{n}\sin\left(\frac{n\pi}{2}\right)\cos(n\omega t).
\end{equation}

\subsubsection{周期锯齿脉冲信号}
\srmg
\begin{equation}
    f(t)=\frac{E}{\pi}\sum_{n=1}^{\infty}\frac{(-1)^{n+1}}{n}\sin(n\omega t).
\end{equation}

\subsubsection{周期三角脉冲信号}
\srmg
\begin{equation}
    f(t)=\frac{E}{2}+\frac{4E}{\pi^2}\sum_{n=1}^{\infty}\frac{1}{n^2}\sin^2\left(\frac{n\pi}{2}\right)\cos(n\omega t).
\end{equation}

\subsubsection{周期半波余弦信号}
\srmg
\begin{equation}
    f(t)=\frac{E}{\pi}-\frac{2E}{\pi}\sum_{n=1}^{\infty}\frac{1}{n^2-1}\cos\left(\frac{n\pi}{2}\right)\cos(n\omega t).
\end{equation}

\subsubsection{周期全波余弦信号}
\srmg
\begin{equation}
    f(t)=E|cos(\omega t)|=\frac{2E}{\pi}+\frac{4E}{\pi}\sum_{n=1}^{\infty}\frac{(-1)^{n+1}}{4n^2-1}\cos(2n\omega t).
\end{equation}
