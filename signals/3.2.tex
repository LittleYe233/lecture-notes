\subsection{典型周期信号的傅里叶级数} \label{3 典型周期信号的傅里叶级数}
\subsubsection{周期矩形脉冲信号}
设 $f(t)$ 脉冲宽度 $\tau$, 相邻两个脉冲中心间隔 $T$, 脉冲幅度 $E$. 一个周期的 $f(t)$ 的解析式为
\begin{equation}
    f(t)=E\left[u\left(t+\frac{\tau}{2}\right)-u\left(t-\frac{\tau}{2}\right)\right].
\end{equation}

则 $f(t)$ 的傅里叶级数可表示为
\begin{equation}
    f(t)=\frac{E\tau}{T}+\frac{2E\tau}{T}\sum_{n=1}^{\infty}\mathrm{Sa}\left(\frac{n\omega\tau}{2}\right)\cos(n\omega t)
\end{equation}
或
\begin{equation}
    f(t)=\frac{E\tau}{T}\sum_{-\infty}^{\infty}\mathrm{Sa}\left(\frac{n\omega\tau}{2}\right)e^{\mathrm{j}n\omega t}.
\end{equation}

在允许失真的条件下, 可以认为信号的频带范围为 $0\leq \omega\leq \dfrac{2\pi}{\tau}$.
\begin{equation} \label{eq:3.2 periodic rect signal Bf}
    B_f=\frac{1}{\tau}.
\end{equation}

\begin{figure}[H]
    \centering
    \begin{minipage}{.433\textwidth}
        \centering
        \includegraphics[width=\textwidth]{./images/signals_periodic_rect_signal.pdf}
        \caption{周期矩形脉冲信号的波形}
    \end{minipage}
    \begin{minipage}{.547\textwidth}
        \centering
        \includegraphics[width=\textwidth]{./images/signals_periodic_rect_signal_freq.pdf}
        \caption{周期矩形脉冲信号的频谱}
    \end{minipage}
\end{figure}

\subsubsection{对称方波信号}
对称方波信号相当于脉冲宽度取 $\tau=\dfrac{T}{2}$, 脉冲幅度取 $\dfrac{E}{2}$ 的周期矩形脉冲信号. 一个周期的 $f(t)$ 的解析式为
\begin{equation}
    \begin{alignedat}{2}
        f(t) & = &  & -\frac{E}{2}\left[u\left(t+\frac{T}{2}\right)-u\left(t+\frac{T}{4}\right)\right]                                                                        \\
        &   &  & +\frac{E}{2}\left[u\left(t+\frac{T}{4}\right)-u\left(t-\frac{T}{4}\right)\right]                                                                        \\
        &   &  & -\frac{E}{2}\left[u\left(t+\frac{T}{4}\right)-u\left(t+\frac{T}{2}\right)\right]                                                                        \\
        & = &  & -\frac{E}{2}\left[u\left(t+\frac{T}{2}\right)-u\left(t-\frac{T}{2}\right)\right]+E\left[u\left(t+\frac{T}{4}\right)-u\left(t-\frac{T}{4}\right)\right].
    \end{alignedat}
\end{equation}

$f(t)$ 的傅里叶级数为
\begin{equation}
    f(t)=\frac{2E}{\pi}\sum_{n=1}^{\infty}\frac{1}{n}\sin\left(\frac{n\pi}{2}\right)\cos(n\omega t).
\end{equation}

\begin{figure}[H]
    \centering
    \begin{minipage}{.448\textwidth}
        \centering
        \includegraphics[width=\textwidth]{./images/signals_symmetrical_square_wave.pdf}
        \caption{对称方波信号的波形}
    \end{minipage}
    \begin{minipage}{.532\textwidth}
        \centering
        \includegraphics[width=\textwidth]{./images/signals_symmetrical_square_wave_freq.pdf}
        \caption{对称方波信号的频谱}
    \end{minipage}
\end{figure}

\subsubsection{周期锯齿脉冲信号}
\srmg
\begin{equation}
    f(t)=\frac{E}{\pi}\sum_{n=1}^{\infty}\frac{(-1)^{n+1}}{n}\sin(n\omega t).
\end{equation}

\subsubsection{周期三角脉冲信号}
\srmg
\begin{equation}
    f(t)=\frac{E}{2}+\frac{4E}{\pi^2}\sum_{n=1}^{\infty}\frac{1}{n^2}\sin^2\left(\frac{n\pi}{2}\right)\cos(n\omega t).
\end{equation}

\begin{figure}[H]
    \centering
    \begin{minipage}{.527\textwidth}
        \centering
        \includegraphics[width=\textwidth]{./images/signals_periodic_jagged_pulse_wave.pdf}
        \caption{周期锯齿脉冲信号的波形}
    \end{minipage}
    \begin{minipage}{.453\textwidth}
        \centering
        \includegraphics[width=\textwidth]{./images/signals_periodic_triangle_pulse_wave.pdf}
        \caption{周期三角脉冲信号的波形}
    \end{minipage}
\end{figure}

\subsubsection{周期半波余弦信号}
\srmg
\begin{equation}
    f(t)=\frac{E}{\pi}-\frac{2E}{\pi}\sum_{n=1}^{\infty}\frac{1}{n^2-1}\cos\left(\frac{n\pi}{2}\right)\cos(n\omega t).
\end{equation}

\subsubsection{周期全波余弦信号}
\srmg
\begin{equation}
    f(t)=E|cos(\omega t)|=\frac{2E}{\pi}+\frac{4E}{\pi}\sum_{n=1}^{\infty}\frac{(-1)^{n+1}}{4n^2-1}\cos(2n\omega t).
\end{equation}

\begin{figure}[H]
    \centering
    \begin{minipage}{.49\textwidth}
        \centering
        \includegraphics[width=\textwidth]{./images/signals_periodic_half_cosine.pdf}
        \caption{周期半波余弦信号的波形}
    \end{minipage}
    \begin{minipage}{.49\textwidth}
        \centering
        \includegraphics[width=\textwidth]{./images/signals_periodic_full_cosine.pdf}
        \caption{周期全波余弦信号的波形}
    \end{minipage}
\end{figure}
