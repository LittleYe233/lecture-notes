\subsection{周期信号的傅里叶变换} \label{3 周期信号的傅里叶变换}
利用 (\ref{eq:3.5 F-1 delta}) 及 (\ref{eq:3.6 freq move}), 得到
\begin{equation}
    \mathcal{F}[e^{\mathrm{j}\omega_0 t}]=2\pi\delta(\omega-\omega_0).
\end{equation}
考虑到周期信号的傅里叶级数 (\ref{eq:3.1 fourier exponential F}) 包含 $e^{\mathrm{j}n\omega_0 t}$. 因此
\begin{equation} \label{eq:3.8 periodic fourier}
    \mathcal{F}[f(t)]=\mathcal{F}\left[\sum_{n=-\infty}^{+\infty}F_n e^{\mathrm{j}n\omega_0 t}\right]={\color{red} 2\pi}\sum_{n=-\infty}^{+\infty}F_n\delta(\omega-n\omega_0).
\end{equation}
周期信号傅里叶变换的各次谐波系数等于傅里叶级数对应项系数的 $2\pi$ 倍.

若记周期脉冲信号的单个脉冲的傅里叶变换 (\ref{eq:3.3 fourier}) 为
\begin{equation}
    F_0(\omega)=\int_{0}^{T_0}f(t)e^{-\mathrm{j}\omega_0 t}\mathrm{d}t,
\end{equation}
同时傅里叶级数的系数为 (\ref{eq:3.1 F n}), 比对可知
\begin{equation}
    F_n=\left.{\color{red} \frac{1}{T_0}}F_0(\omega)\right|_{\omega=n\omega_0}.
\end{equation}

根据 (\ref{eq:3.8 periodic fourier}), 容易得到正余弦函数的傅里叶变换.

对于 $\cos(\omega_0 t)$, 容易得到
\begin{equation*}
    \begin{cases}
        a_1=1,           \\
        b_1=0,           \\
        F_1=\frac{1}{2}, \\
        F_{-1}=\frac{1}{2}.
    \end{cases}
\end{equation*}
所以
\begin{equation}
    \mathcal{F}[\cos(\omega_0 t)]=\pi[\delta(\omega+\omega_0)+\delta(\omega-\omega_0)].
\end{equation}

对于 $\sin(\omega_0 t)$, 容易得到
\begin{equation*}
    \begin{cases}
        a_1=0,                      \\
        b_1=1,                      \\
        F_1=-\frac{1}{2}\mathrm{j}, \\
        F_{-1}=\frac{1}{2}\mathrm{j}.
    \end{cases}
\end{equation*}
所以
\begin{equation}
    \mathcal{F}[\sin(\omega_0 t)]=\mathrm{j}\pi[\delta(\omega+\omega_0)-\delta(\omega-\omega_0)].
\end{equation}
