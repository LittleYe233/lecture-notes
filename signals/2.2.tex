\subsection{起始点的跳变} \label{起始点的跳变}

一般 $n$ 阶系统需要 $n$ 个独立条件给定初始状态, 可以是系统响应的各阶导数值.

\textbf{起始状态 ($0_{-}$ 状态)}
\begin{equation}
    r^{(k)}(0_{-})=\left[r(0_{-}),\cdots,\frac{\mathrm{d}^{n-1}r(0_{-})}{\mathrm{d}t^{n-1}}\right].
\end{equation}

\textbf{初始状态 ($0_{+}$ 状态)}
\begin{equation}
    r^{(k)}(0_{+})=\left[r(0_{+}),\cdots,\frac{\mathrm{d}^{n-1}r(0_{+})}{\mathrm{d}t^{n-1}}\right].
\end{equation}

\textbf{换路定律}\quad 在没有冲激电流 (或阶跃电压) 强迫作用于电容的条件下, 电容两端电压 $v_C(t)$ 不发生跳变; 在没有冲激电压 (或阶跃电流) 强迫作用于电感的条件下, 电感两端电流 $i_L(t)$ 不发生跳变.
\begin{gather}
    v_C(0_{+})=v_C(0_{-}), \label{eq:2.2 transformation theorem v_C} \\
    i_L(0_{+})=i_L(0_{-}). \label{eq:2.2 transformation theorem i_L}
\end{gather}

\textbf{冲激函数匹配法}\quad 跳变前后状态 $\delta(t)$ 及其各阶导数的系数应分别相等. 具体的, 设响应的最高阶导数为 $\delta(t)$ 及其各阶导数的单项式和, 最高阶的 $\delta(t)$ 项与自由项的对应项相同, 其余各项的系数均待定. 依次求出各阶响应的表达式并代回原微分方程, 比较系数时忽略 $\Delta u(t)$ 项即可.

\begin{exampleprob}
    $\ddd{1}{t}r(t)+3r(t)=3\delta'(t)$, 且 $r(0_-)=1$, 求 $r(0_+)$.

    \begin{solution}
        设 $\ddd{1}{t}r(t)=3\delta'(t)+a\delta(t)$, 则 $r(t)=3\delta(t)+a\Delta u(t)$. 代入题设等式, 得到 $3\delta'(t)+(a+9)\delta(t)+3a\Delta u(t)=3\delta'(t)$. {\color{red} 仅关注 $\delta(t)$ 项}, 得到 $a=-9$, 则 $r(t)=3\delta(t)-9\Delta u(t)$. 因此 $r(0_+)=r(0_-)-9=-8$.

        可以看出题解等式并不能平衡, 但因为题设求 $r(0_+)$, 也即 $r(t)$ 的 $\Delta u(t)$ 项的系数, 从题设微分方程可以看出这个系数仅与 $\ddd{1}{t}r(t)$ 中 $\delta(t)$ 的系数和 $r(t)$ 中 $\Delta u(t)$ 的系数有关, 因此只需考虑相关项的系数即可. 事实上, 若设 $\ddd{1}{t}r(t)=3\delta'(t)+a\delta(t)+b\Delta u(t)$, 可以使题设等式平衡.
    \end{solution}
\end{exampleprob}

\begin{exampleprob}
    $r''(t)+3r'(t)+2r(t)=2\delta(t)$. 已知 $r(0_-)$ 和 $r'(0_-)$, 求 $r(0_+)$ 和 $r'(0_+)$.

    \begin{solution}
        本题无需待定系数. 考虑到 $r''(t)$ 必然含 $2\delta(t)$ 项, 则两边积分后 $r'(t)$ 必然{\color{red} 仅}包含 $2\Delta u(t)$ 项, 同时 $r(t)=0$. 因此 $r(0_+)=r(0_-)$, $r'(0_+)=r'(0_-)+2$.
    \end{solution}
\end{exampleprob}
