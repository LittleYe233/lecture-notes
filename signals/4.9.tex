\subsection{全通函数与最小相移函数的零, 极点分布} \label{4 全通函数与最小相移函数的零, 极点分布}
\subsubsection{全通函数的零, 极点分布}
\textbf{全通}\quad 幅频特性为常数, 对全部频率的正弦信号都施以相同的幅度传输系数.

根据系统函数 $H(\mathrm{j}\omega)$ 的表达式 (\ref{eq:4.7 H j omega}), 可以得到全通系统函数的零, 极点分布满足:
\begin{itemize}
    \item 零点和极点以虚轴为镜像对称分布;
    \item 极点位于虚轴左侧 (考虑到极点实部产生指数因子 (见表 \ref{tab:4.6 H(s) 典型一阶和二阶极点与 h(t) 波形特征的对应关系 (定性)}), 可以\textit{保证系统的稳定性}).
\end{itemize}

\subsubsection{最小相移函数的零, 极点分布}
\textbf{最小相移系统}\quad 系统函数的零点均分布在 s 平面的左半平面或虚轴上.

任何一个非最小相移系统均可以表示为一个\textit{全通系统}和一个\textit{最小相移系统}的级联 (系统函数相乘), 也即
\begin{equation}
    H(s)=H_{AP}(s)H_{min}(s).
\end{equation}

其数学原理是全通系统函数 $H_{AP}(s)$ 的极点和最小相移系统函数 $H_{min}(s)$ 的零点相消 (如此操作可以破坏全通系统函数的零极点对称特性). 若能同时消除最小相移系统函数的所有位于虚轴左侧的零点, 则此时即可得到一个非最小相移系统的函数.

\begin{exampleprob}
    将系统函数 $H(s)=\dfrac{(s-\alpha_2)^2+\Omega_2^2}{(s+\alpha_0)[(s+\alpha_1)^2+\Omega_1^2]}$ 表示为一个全通系统与一个最小相移系统的级联.

    \begin{solution}
        显然
        \begin{gather*}
            H_{AP}(s)=\frac{(s-\alpha_2)^2+\Omega_2^2}{\color{red} (s+\alpha_2)^2+\Omega_2^2}, \\
            H_{min}(s)=\frac{\color{red} (s+\alpha_2)^2+\Omega_2^2}{(s+\alpha_0)[(s+\alpha_1)^2+\Omega_1^2]}.
        \end{gather*}
    \end{solution}
\end{exampleprob}
