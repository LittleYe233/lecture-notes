\subsection{傅里叶变换} \label{3 傅里叶变换}
\rmg
\begin{gather}
    F(\omega)=\mathcal{F}[f(t)]=\int_{-\infty}^{\infty}f(t)e^{-\mathrm{j}\omega t}\mathrm{d}t. \label{eq:3.3 fourier} \\
    f(t)=\mathcal{F}^{-1}[F(\omega)]=\frac{1}{2\pi}\int_{-\infty}^{\infty}F(\omega)e^{\mathrm{j}\omega t}\mathrm{d}\omega.
\end{gather}

\textbf{频谱函数} $F(\omega)=|F(\omega)|e^{\mathrm{j}\varphi(\omega)}$.

$|F(\omega)|$ 表示信号中频率为 $\omega$ 的分量的\textit{相对}大小, $\varphi(\omega)$ 表示信号中频率为 $\omega$ 的分量的相位.

\textbf{傅里叶逆变换的三角形式}
\begin{equation}
    \begin{aligned}
        f(t)= & \frac{1}{2\pi}\int_{-\infty}^{\infty}|F(\omega)|\cos[\omega t+\varphi(\omega)]\mathrm{d}\omega           \\
        +     & \frac{\mathrm{j}}{2\pi}\int_{-\infty}^{\infty}|F(\omega)|\sin[\omega t+\varphi(\omega)]\mathrm{d}\omega.
    \end{aligned}
\end{equation}

若 $f(t)$ 是实函数,
\begin{equation}
    f(t)=\frac{1}{\pi}\int_{0}^{\infty}|F(\omega)|\cos[\omega t+\varphi(\omega)]\mathrm{d}\omega.
\end{equation}

可以看出, 非周期信号可以分解为从零到无穷高频率的正余弦分量.

\textbf{傅里叶变换存在的充分条件}

信号在无限区间内满足绝对可积条件, 也即
\begin{equation}
    \int_{-\infty}^{\infty}|f(t)|\mathrm{d}t<\infty.
\end{equation}

借助奇异信号, 可以使不满足该条件的信号存在傅里叶变换.
