\subsection{拉普拉斯变换的基本性质} \label{4 拉普拉斯变换的基本性质}
\subsubsection{线性}
若记 $\mathcal{L}[f_1(t)]=F_1(s)$, $\mathcal{L}[f_2]=F_2(s)$, 则
\begin{equation}
    \mathcal{L}[af_1(t)+bf_2(t)]=aF_1(s)+bF_2(s).
\end{equation}

\subsubsection{时域微分性质}
若记 $\mathcal{L}[f(t)]=F(s)$, 则
\begin{equation} \label{eq:4.2 diff t}
    \mathcal{L}[f^{(n)}(t)]=s^nF(s)-\sum_{r=0}^{n-1}s^{n-r-1}f^{(r)}({\color{red} 0_-}).
\end{equation}
特别的,
\begin{gather}
    \mathcal{L}[f'(t)]=sF(s)-f({\color{red} 0_-}), \\
    \mathcal{L}[f''(t)]=s^2F(s)-sf({\color{red} 0_-})-f'({\color{red} 0_-}).
\end{gather}

\subsubsection{时域积分性质}
若记 $\mathcal{L}[f(t)]=F(s)$, 则
\begin{equation}
    \mathcal{L}\left[\int_{-\infty}^{t}f(\tau)\mathrm{d}\tau\right]=\frac{F(s)}{s}+\frac{f^{(-1)}({\color{red} 0_-})}{s},
\end{equation}
其中
\begin{equation*}
    f^{(-1)}(0_-)=\int_{-\infty}^{0_-}f(\tau)\mathrm{d}\tau.
\end{equation*}

\subsubsection{复频域微分性质}
若记 $\mathcal{L}[f(t)]=F(s)$, 则
\begin{equation}
    \mathcal{L}[tf(t)]=-F'(s).
\end{equation}

\subsubsection{复频域积分性质}
若记 $\mathcal{L}[f(t)]=F(s)$, 则
\begin{equation}
    \mathcal{L}\left[\frac{f(t)}{t}\right]=\int_{s}^{\infty}F(\lambda)\mathrm{d}\lambda.
\end{equation}

\subsubsection{延时性质} \label{4.2 延时性质}
若记 $\mathcal{L}[f(t)]=F(s)$, 则
\begin{equation}
    \mathcal{L}[f(t-t_0)u(t-t_0)]=e^{-st_0}F(s).\qquad ({\color{red} t_0\geq 0})
\end{equation}
考虑到信号的因果性, 一般规定 $t_0\geq 0$.

\subsubsection{频移性质}
若记 $\mathcal{L}[f(t)]=F(s)$, 则
\begin{equation}
    \mathcal{L}[f(t)e^{-at}]=F(s+a).
\end{equation}

\subsubsection{尺度变换}
若记 $\mathcal{L}[f(t)]=F(s)$, 则
\begin{equation}
    \mathcal{L}[f(at)]=\frac{1}{a}F\left(\frac{s}{a}\right).\qquad ({\color{red} a>0})
\end{equation}

若同时进行延时和尺度变换, 例如 $\mathcal{L}[f(at-b)u(at-b)]$, 可以先延时再尺度变换, 反之亦然. 以前者为例:
\begin{gather}
    \nonumber \mathcal{L}[f(t)u(t)]=F(s). \\
    \nonumber \mathcal{L}[f(t-b)u(t-b)]=F(s)e^{-bs}. \\
    \mathcal{L}[f(at-b)u(at-b)]=\frac{1}{a}F\left(\frac{s}{a}\right)e^{-\frac{b}{a}s}.\qquad (a>0,b>0)
\end{gather}

\subsubsection{初值定理}
若记 $\mathcal{L}[f(t)]=F(s)$, $f'(t)$ 也可以进行拉普拉斯变换, 则
\begin{equation}
    \lim_{t\rightarrow 0_+}f(t)=f(0_+)=\lim_{s\rightarrow\infty}sF(s).
\end{equation}
若 $f(t)$ 包含冲激函数分量 $k\delta(t)$, 则
\begin{equation}
    \lim_{t\rightarrow 0_+}f(t)=\lim_{s\rightarrow\infty}[sF(s)-ks].
\end{equation}

\subsubsection{终值定理}
若记 $\mathcal{L}[f(t)]=F(s)$, $f'(t)$ 也可以进行拉普拉斯变换, 则
\begin{equation}
    \lim_{t\rightarrow\infty}f(t)=\lim_{s\rightarrow 0}sF(s).
\end{equation}

\subsubsection{卷积定理}
若记 $\mathcal{L}[f_1(t)]=F_1(s)$, $\mathcal{L}[f_2(t)]=F_2(s)$, 则
\begin{gather}
    \mathcal{L}[f_1(t)*f_2(t)]=F_1(s)F_2(s), \label{eq:4.2 conv t} \\
    \mathcal{L}[f_1(t)f_2(t)]=\frac{1}{2\pi\mathrm{j}}[F_1(s)*F_2(s)]. \label{eq:4.2 conv s}
\end{gather}
