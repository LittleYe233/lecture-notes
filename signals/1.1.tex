\section{绪论} \label{绪论}
\subsection{信号的描述, 分类和典型示例}
\subsubsection{常见分类}

\begin{itemize}
    \item 确定性信号\ 与\ 随机信号 (伪随机信号可以确定.)
    \item 周期信号\ 与\ 非周期信号 (伪随机信号周期相对较长.)
    \item 连续时间信号 (模拟信号)\ 与\ 离散时间信号
          \begin{itemize}
              \item \textbf{抽样信号}\quad 时间不连续, 幅度连续 (\textit{例: $\sin(0.4n)$.})
              \item \textbf{数字信号}\quad 时间不连续, 幅度不连续 (\textit{例: $\cos(n\pi)$.})
          \end{itemize}
    \item 一维信号\ 与\ 多维信号 (图像信号一般是二维信号.)
    \item 能量信号\ 与\ 功率信号
\end{itemize}

\subsubsection{指数信号}

\rmg
\begin{equation}
    f(t)=Ke^{at}.
\end{equation}

\textbf{指数衰减信号}
\begin{equation}
    f(t)=\begin{cases}
        0,\quad t<0, \\
        \exp\left(-\dfrac{t}{\tau}\right),\quad t\geq 0.
    \end{cases}
\end{equation}

\subsubsection{正弦信号}

\rmg
\begin{equation}
    f(t)=K\sin(\omega t+\theta).
\end{equation}

\textbf{复指数形式}
\begin{equation}
    \begin{gathered}
        \sin\omega t=\frac{e^{\mathrm{j}\omega t}-e^{-\mathrm{j}\omega t}}{2\mathrm{j}}, \\
        \cos\omega t=\frac{e^{\mathrm{j}\omega t}+e^{\mathrm{j}\omega t}}{2}.
    \end{gathered}
\end{equation}

\textbf{指数衰减正弦信号}
\begin{equation}
    f(t)=\begin{cases}
        0,\quad t<0, \\
        Ke^{-at}\sin\omega t,\quad t\geq 0.
    \end{cases}
\end{equation}

\subsubsection{复指数信号}

\rmg
\begin{equation}
    f(t)=Ke^{(\sigma+\mathrm{j}\omega)t}=Ke^{\sigma t}\cos\omega t+\mathrm{j}Ke^{\sigma t}\sin\omega t.
\end{equation}

\begin{figure}[ht]
    \centering
    \begin{minipage}{.48\textwidth}
        \centering
        \includegraphics[width=\textwidth]{images/exponentially_decaying_sinusoidal_signal.pdf}
        \caption{指数衰减正弦信号}
    \end{minipage}
    \begin{minipage}{.48\textwidth}
        \centering
        \includegraphics[width=\textwidth]{images/sampling_signal.pdf}
        \caption{抽样信号}
    \end{minipage}
\end{figure}

\subsubsection{抽样信号}

\rmg
\begin{equation}
    \mathrm{Sa}(t)=\frac{\sin t}{t}.
\end{equation}

\subsubsection{钟形信号 (高斯信号)}

\rmg
\begin{equation}
    f(t)=E\exp\left(-\left(\frac{t}{\tau}\right)^2\right).
\end{equation}

$f\left(\dfrac{\tau}{2}\right)\approx 0.78E$.
