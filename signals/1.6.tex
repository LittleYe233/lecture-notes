\subsection{线性时不变系统 (Linear Time-invariant, LTI)}

\subsubsection{叠加性与齐次性}

\textbf{叠加性}\quad $\begin{cases}
        x_1(t)\rightarrow y_1(t), \\
        x_2(t)\rightarrow y_2(t)
    \end{cases}\Rightarrow x_1(t)+x_2(t)\rightarrow y_1(t)+y_2(t)$.

\textbf{齐次性}\quad $x_i(t)\rightarrow y_i(t)\Rightarrow ax_i(t)\rightarrow ay_i(t)$.

线性系统\textit{零输入}时得到\textit{零输出}.

\subsubsection{时不变性}

$x(t)\rightarrow y(t)\Rightarrow x(t-t_0)\rightarrow y(t-t_0)$.

($-t_0$ 表示对信号的\textit{移位}, 例如 $x(2t)$ 移位 $-t_0$ 后为 $x(2t-t_0)$.)

\begin{exampleprob}
    判断 $y_1(t)=tx(t)$, $y_2(t)=ax(t)+b$, $y_3(t)=x(2t)$ 是否具有叠加性, 齐次性, 时不变性.

    \begin{solution}
        因为
        \srmg
        \begin{gather*}
            t(x_a(t)+x_b(t))=tx_a(t)+tx_b(t)=y_{1a}(t)+y_{1b}(t), \\
            t(ax(t))=a(tx(t))=ay_1(t), \\
            tx(t-t_0)\neq y_1(t-t_0)=(t-t_0)x(t-t_0),
        \end{gather*}

        所以 $y_1(t)$ 具有叠加性, 齐次性, 不具有时不变性.

        因为
        \srmg
        \begin{gather*}
            a(x_a(t)+x_b(t))_b\neq y_{2a}(t)+y_{2b}(t)=(ax_a(t)+b)+(ax_b(t)+b), \\
            a(kx(t))+b\neq ky_2(t)=k(ax(t)+b), \\
            ax(t-t_0)+b=y_2(t-t_0),
        \end{gather*}

        所以 $y_2(t)$ 具有时不变性, 不具有叠加性, 齐次性.

        因为
        \srmg
        \begin{gather*}
            x_a(2t)+x_b(2t)=y_{3a}(t)+y_{3b}(t), \\
            ax(2t)=ay_3(t), \\
            x(2t-t_0)\neq y_3(t-t_0)=x(2(t-t_0)),
        \end{gather*}

        所以 $y_3(t)$ 具有叠加性, 齐次性, 不具有时不变性.
    \end{solution}
\end{exampleprob}

\subsubsection{微分特性与积分特性}

\textbf{微分特性}\quad $x(t)\rightarrow y(t)\Rightarrow x'(t)\rightarrow y'(t)$.

\textbf{积分特性}\quad $x(t)\rightarrow y(t)\Rightarrow \displaystyle\int_{-\infty}^{t}x(t)\mathrm{d}t\rightarrow\int_{-\infty}^{t}y(t)\mathrm{d}t$.

\subsubsection{因果性}

\textit{因果系统}在 $t=t_0$ 时刻的响应仅与 $t\leq t_0$ 的激励有关.
