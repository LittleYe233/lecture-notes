\subsection{冲激响应与阶跃响应}

\subsubsection{单位冲激响应}

以单位冲激信号 $\delta(t)$ 作激励, 系统产生的\textit{零状态响应}.

$h(t)$ 的形式与齐次解 (\ref{eq:2.1 lti model r_h}) 相同, 使用\textit{冲激函数匹配法}求出初始条件后, 即可求得各项系数. 因为冲激信号引入的能量存储可以等效为零输入响应的初始条件, 求解冲激响应 (齐次解) \textit{绕过了求解初始条件的过程}.

$h(t)$ 是否包含 $\delta(t)$ 与响应阶数 $n$ 和激励阶数 $m$ 有关:
\begin{itemize}
    \item $n>m$: $h(t)$ 不包含 $\delta(t)$;
    \item $n=m$: $h(t)$ 包含 $\delta(t)$;
    \item $n<m$: $h(t)$ 包含 $\delta(t)$ 及其各阶导数.
\end{itemize}

\subsubsection{单位阶跃响应}

$g(t)$. 以单位阶跃信号 $u(t)$ 作激励, 系统产生的\textit{零状态响应}.
\begin{gather}
    \frac{\mathrm{d}h(t)}{\mathrm{d}t}=g(t). \\
    \begin{aligned}
        g(t) & =\int_{-\infty}^{t}h(\tau)\mathrm{d}\tau.                                \\
             & =\int_{0_-}^{t}h(\tau)\mathrm{d}\tau.\qquad (t=0_-\ \textrm{加入激励的因果系统}.)
    \end{aligned}
\end{gather}

\begin{exampleprob}
    求 $\ddd{2}{t}r(t)+4\ddd{1}{t}r(t)+3r(t)=\ddd{1}{t}e(t)+2e(t)$ 的冲激响应 $h(t)$.

    \begin{solution}[1]
        容易得到 $h(t)$ 的形式为 $h(t)=(A_1e^{-t}+A_2e^{-3t})u(t)$. 现求解初始条件, 使用冲激函数匹配法, 设 $h''(t)=\delta'(t)+a\delta(t)$, 则 $h'(t)=\delta(t)+a\Delta u(t)$, $h(t)=\Delta u(t)$. 此时可以得到 $h(0_+)-h(0_-)=1$. 代入题设微分方程, 解得 $a=-2$, 所以 $h'(0_+)-h'(0_-)=-2$. 由此可以解得 $h(t)=\dfrac{1}{2}(e^{-t}+e^{-3t})u(t)$.
    \end{solution}

    \begin{solution}[2]
        已知 $h(t)=(A_1e^{-t}+A_2e^{-3t})u(t)$, 利用冲激函数的筛选性质 $f(t)\delta(t)=f(0)\delta(t)$, 可以得到
        \begin{align*}
            h'(t)  & =(A_1e^{-t}+A_2e^{-3t})\delta(t)+(-A_1e^{-t}-3A_2e^{-3t})u(t)                       \\
                   & =(A_1+A_2)\delta(t)+(-A_1e^{-t}-3A_2e^{-3t})u(t).                                   \\
            h''(t) & = (A_1+A_2)\delta'(t)+(-A_1e^{-t}-3A_2e^{-3t})\delta(t)+(A_1e^{-t}+9A_2e^{-3t})u(t) \\
                   & =(A_1+A_2)\delta'(t)+(-A_1-3A_2)\delta(t)+(A_1e^{-t}+9A_2e^{-3t})u(t).
        \end{align*}

        忽略 $u(t)$ 项后, 代回题设微分方程, 比较系数后, 可以得到相同的结果.
    \end{solution}
\end{exampleprob}
