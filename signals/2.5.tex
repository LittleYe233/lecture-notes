\subsection{卷积} \label{2 卷积}

\textbf{卷积的数学定义}
\begin{equation} \label{eq:2.5 conv}
    f_1(t)*f_2(t)=\int_{-\infty}^{\infty}f_1(\tau)f_2(t-\tau)\mathrm{d}\tau.
\end{equation}

注意 $f_2(t-\tau)$ 相当于先将 $f_2(\tau)$ 做反褶变换后, 再向\textit{右}移动 $t$ 个单位; 或先向\textit{左}移动 $t$ 个单位, 再做反褶变换.

\textbf{求解系统的零状态响应}
\begin{equation}
    r(t)=e(t)*h(t)=\int_{-\infty}^{\infty}e(\tau)h(t-\tau)\mathrm{d}\tau.
\end{equation}

\textbf{梯形脉冲 (三角形脉冲)} \label{2.5 梯形脉冲 (三角形脉冲)}

两个关于 $t=0$ 对称的矩形脉冲卷积可以得到梯形脉冲或三角形脉冲. 具体的, 对于函数 $f_1(t)=E_1G_{\tau_1}(t)$, $f_2(t)=E_2G_{\tau_2}(t)$, 则 $[f_1*f_2](t)$ 的图象呈关于 $t=0$ 对称的梯形, 上底宽度为 $|\tau_2-\tau_1|$, 下底宽度为 $\tau_1+\tau_2$, 高度为 $E_1E_2\min\{\tau_1, \tau_2\}$. 特别的, 当 $\tau_1=\tau_2$ 时, 梯形信号退化为三角形信号.

\begin{exampleprob}
    已知 $f_1(t)=u(t)$, $f_2(t)=e^{-t}u(t)$, 求 $f_1(t)*f_2(t)$.

    \begin{solution}[1]
        容易得到 $f_1(\tau)$ 在 $\tau>0$ 时不等于零; $f_2(\tau)$ 在 $\tau\geq 0$ 时不等于零, 则 $f_2(t-\tau)$ 在 $\tau\leq t$ 时不等于零. 由此可以分类讨论 $t$.

        当 $t\leq 0$ 时, 显然 $f_1(t)*f_2(t)=0$.

        当 $t>0$ 时, $f_1(t)*f_2(t)=\displaystyle\int_{0}^{t}f_1(\tau)f_2(t-\tau)\mathrm{d}\tau=1-e^{-t}$.

        所以 $f_1(t)*f_2(t)=(1-e^{-t})u(t)$.
    \end{solution}

    \begin{solution}[2]
        由卷积的数学定义, $f_1(t)*f_2(t)=\displaystyle\int_{0}^{t}u(\tau)e^{t-\tau}u(t-\tau)\mathrm{d}\tau$. 若要使被积函数不为零, 则 $u(\tau)u(t-\tau)\neq 0$, 也即 $\begin{cases}
                \tau>0, \\
                t-\tau>0
            \end{cases}$, 即 $0<\tau<t$. 由此可以推得卷积的上下限.
    \end{solution}
\end{exampleprob}

\begin{exampleprob}
    某 LTI 系统的激励为 $e(t)=e^{-t}u(t)$, 冲激响应为 $h(t)=(2e^{-t}-e^{-2t})u(t)$. 求该激励下的零状态响应 $r_{zs}(t)$.

    \begin{solution}
        \rmg
        \begin{align*}
            r_{zs}(t) & =e(t)*h(t)                                                                                          \\
                      & =\int_{-\infty}^{\infty}e^{-\tau}u(\tau)\cdot[2e^{-(t-\tau)}-e^{-2(t-\tau)}]u(t-\tau)\mathrm{d}\tau \\
                      & = \int_{0}^{t}e^{-\tau}[2e^{-(t-\tau)}-e^{-2(t-\tau)}]\mathrm{d}\tau                                \\
                      & =(2t-1)e^{-t}+e^{-2t}.
        \end{align*}

        结果与解微分方程的结果相同.
    \end{solution}
\end{exampleprob}

\textbf{卷积积分的上下限}

若函数 $f_1(t)$ 当且仅当 $t_1<t<t_2$ 时有非零值, $f_2(t)$ 当且仅当 $t_3<t<t_4$ 时有非零值, 则卷积积分的上下限为 $t_1+t_3<t<t_2+t_4$.
