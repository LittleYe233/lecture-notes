\section{课前预习思考题} \label{课前预习思考题}
\subsection{2023-2-28}

\subsubsection*{什么是信号处理? 请举出 2 个信号处理的应用场景.}

% See: https://zh.wikipedia.org/wiki/%E4%BF%A1%E5%8F%B7%E5%A4%84%E7%90%86
对信号表示, 变换, 运算等进行处理的过程. 音频的倍速播放, 中频信号提取人声.

\subsubsection*{伪随机信号是非周期信号吗? 它在通信中有什么作用?}

不是. 伪随机信号周期较长, 可以用确定的信号处理方式生成. 信息加密.

\subsubsection*{请分别举出信号的延时, 反褶, 压缩在现实生活中的一个例子.}

音频延迟播放. 卷积运算. 音频倍速播放.

\subsubsection*{单位阶跃函数在电路中的物理意义是什么?}

在 $t=0$ 时刻, 电路中开关闭合, 电路通电.

\subsection{2023-3-2}

\subsubsection*{将信号分解为冲激信号的叠加有什么益处?}

% See: https://baike.baidu.com/item/%E5%86%B2%E6%BF%80/5922172
便于使用卷积对信号时域分析.

\subsubsection*{相同的数学模型可表征不同的系统, 这种说法是否正确? 为什么?}

% Chapter 1.5
正确. 根据电路的对偶原理, 激励源为电压源的 RLC 串联回路求解电流\ 与\ 合适参数的激励源为电流源的 GLC 并联回路求解端电压\ 可以具有相同的微分方程.

\subsubsection*{分别举出一个因果系统和一个非因果系统的例子, 并说明理由.}

\textbf{因果系统}\quad 通常由电阻器, 电感线圈, 电容器构成的实际物理系统. 因为下一时刻各个元器件通过的电流, 端电压等物理量可以并仅由先前时刻的各物理量和电路元器件参数得到.

\textbf{非因果系统}\quad 信号处理时, 信号的压缩, 求解统计学参数等. 因为这些操作可能需要提前获取到整个时域的信号, 对某一时刻进行这些操作时会使用到后续时刻的信息.

\subsubsection*{一个高铁通信系统是时不变系统吗? 为什么?}

根据给定条件, 待研究问题及对象的不同, 无法判断. 例如, 高铁非匀速运动时, 若研究某问题时涉及通信系统发射信号的频率, 根据多普勒效应, 对接收者而言接收到的频率可能会与时间相关联, 则考虑该问题时高铁通信系统的某一系统参数发生改变, 则其不是时不变系统.
