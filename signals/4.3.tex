\subsection{拉普拉斯逆变换} \label{4 拉普拉斯逆变换}
若某拉普拉斯变换式为一关于 $s$ 的有理分式
\begin{equation}
    F(s)=\frac{A(s)}{B(s)}=\frac{a_ms^m+\cdots+a_0}{b_ns^n+\cdots+b_0}.
\end{equation}
其因式分解结果为
\begin{equation}
    F(s)=\frac{a_m(s-z_1)(s-z_2)\cdots(s-z_m)}{b_n(s-p_1)(s-p_2)\cdots(s-p_n)}.
\end{equation}
则此时可以使用\textit{部分分式分解}的方法求出其拉普拉斯逆变换. 另外对于有理分式与 $e^{-st}$ 相乘的形式, 则可以先利用延时性质 (见 \ref{4.2 延时性质} 小节), 再沿用部分分式分解的方法.

根据极点 $b_i$ 的不同情况, 部分分式分解法有几种不同应用方式.

\textbf{极点为各不相等实数}

若 $m<n$, 可将 $F(s)$ 分解为 $n$ 个 $s$ 的一阶分式的形式
\begin{equation}
    F(s)=\frac{k_1}{s-p_1}+\frac{k_2}{s-p_2}+\cdots+\frac{k_n}{s-p_n},
\end{equation}
其中各系数 $k_i$ 可以通过通分比对系数求解方程组求得, 也可以利用
\begin{equation} \label{eq:4.3 partial fraction k}
    \begin{cases}
        k_1=(s-p_1)F(s)|_{s=p_1}, \\
        k_2=(s-p_2)F(s)|_{s=p_2}, \\
        \cdots                    \\
        k_i=(s-p_i)F(s)|_{s=p_i}.
    \end{cases}
\end{equation}
则可以求得 $F(s)$ 的拉普拉斯逆变换为
\begin{equation}
    f(t)=(k_1e^{p_1t}+k_2e^{p_2t}+\cdots+k_ne^{p_nt})u(t).
\end{equation}

若 $m\geq n$, 可利用\textit{长除法}将 $F(s)$ 再分解出一个关于 $s$ 的多项式. 对该多项式求拉普拉斯逆变换, 合并剩余的部分分式对应的拉普拉斯逆变换式, 即可得到最终的结果.

\begin{exampleprob}
    求 $F(s)=\dfrac{s^3+5s^2+9s+7}{(s+1)(s+2)}$ 的拉普拉斯逆变换.

    \begin{solution}
        用长除法得到
        \begin{equation*}
            F(s)=s+2+\frac{s+3}{(s+1)(s+2)}.
        \end{equation*}

        对后项部分分式分解. 设 $\dfrac{s+3}{(s+1)(s+2)}=\dfrac{k_1}{s+1}+\dfrac{k_2}{s+2}$. 则
        \begin{gather*}
            k_1=\left.\frac{s+3}{s+2}\right|_{s=-1}=2, \\
            k_2=\left.\frac{s+3}{s+1}\right|_{s=-2}=-1.
        \end{gather*}

        因此
        \begin{align*}
                        & F(s)=s+2+\frac{2}{s+1}-\frac{1}{s+2}              \\
            \Rightarrow & f(t)=(\delta'(t)+2\delta(t)+2e^{-t}-e^{-2t})u(t).
        \end{align*}
    \end{solution}
\end{exampleprob}

\textbf{极点存在共轭复数}

假设 $B(s)=D(s)[(s+\alpha)^2+\beta^2]$, 即存在一对共轭复极点 $p_{1,2}=-\alpha\pm\mathrm{j}\beta$. 对其做类似的部分分式分解, 取其中的复极点相关的部分
\begin{equation}
    F_C(s)=\frac{k_C}{s+\alpha-\mathrm{j}\beta}+\frac{k_C^*}{s+\alpha+\mathrm{j}\beta}.
\end{equation}

不妨设 $k_C=A+\mathrm{j}B$, 显然 $k_C^*=A-\mathrm{j}B$, 则
\begin{equation} \label{eq:4.3 partial fraction complex f C}
    f_C(t)=e^{-\alpha t}(k_Ce^{\mathrm{j}\beta t}+k_C^* e^{-\mathrm{j}\beta t})=2e^{-\alpha t}[A\cos(\beta t)-B\sin(\beta t)].
\end{equation}

实际运算中, 一般先对其中一项 (例如 $\dfrac{k_C}{s+\alpha-\mathrm{j}\beta}$) 使用类似式 (\ref{eq:4.3 partial fraction k}) 求出 $k_C$, 也即 $A$ 和 $B$, 再直接代入式 (\ref{eq:4.3 partial fraction complex f C}).

\begin{exampleprob}
    求 $F(s)=\dfrac{s^2+3}{(s^2+2s+5)(s+2)}$ 的拉普拉斯逆变换.

    \begin{solution}
        对其部分分式分解,
        \begin{equation*}
            F(s)=\frac{k_0}{s+2}+\frac{k_1}{s+1-\mathrm{j}2}+\frac{k_1^*}{s+1+\mathrm{j}2}.
        \end{equation*}

        求解相关系数:
        \begin{gather*}
            k_0=(s+2)F(s)|_{s=-2}=\frac{7}{5}, \\
            k_1=(s+1-\mathrm{j}2)F(s)|_{s=-1+\mathrm{j}2}=\frac{-1+\mathrm{j}2}{5}.
        \end{gather*}
        也即 $A=-\dfrac{1}{5}$, $B=\dfrac{2}{5}$. 因此
        \begin{equation*}
            f(t)=\frac{7}{5}e^{-2t}+2e^{-t}\left[-\frac{1}{5}\cos(2t)-\frac{2}{5}\sin(2t)\right]. \qquad (t\geq 0)
        \end{equation*}
    \end{solution}
\end{exampleprob}

若 $D(s)=0$, 不必用部分分式分解, 可以利用表 \ref{tab:4.1 laplace} 中三角函数的拉普拉斯变换式比对得到相关系数. 见例题 \ref{eprob:4.3 D(s)=0}.

\begin{exampleprob} \label{eprob:4.3 D(s)=0}
    求 $F(s)=\dfrac{s+\gamma}{(s+\alpha)^2+\beta^2}$ 的拉普拉斯逆变换.

    \begin{solution}
        观察得到
        \begin{equation*}
            F(s)=\frac{s+\alpha}{(s+\alpha)^2+\beta^2}+\frac{\gamma-\alpha}{\beta}\frac{\beta}{(s+\alpha)^2+\beta^2}.
        \end{equation*}
        所以
        \begin{equation*}
            f(t)=\left[e^{-\alpha t}\cos(\beta t)+\frac{\gamma-\alpha}{\beta}\sin(\beta t)\right]u(t).
        \end{equation*}
    \end{solution}
\end{exampleprob}

\textbf{极点有重根}

假设 $B(s)=D(s)(s-p_1)^k$, 即存在 $k$ 重极点 $p_1$. 对其做类似的部分分式分解, 取其中的重根相关的部分
\begin{equation}
    F_R(s)=\frac{k_{11}}{(s-p_1)^k}+\frac{k_{12}}{(s-p_1)^{k-1}}+\cdots+\frac{k_{1k}}{s-p_1},
\end{equation}
其中
\begin{equation}
    \begin{cases}
        k_{11}=\left.(s-p_1)^kF(s)\right|_{s=p_1},                                              \\
        k_{12}=\left.\frac{\mathrm{d}}{\mathrm{d}s}(s-p_1)^kF(s)\right|_{s=p_1},                \\
        k_{13}=\left.\frac{1}{2}\frac{\mathrm{d}^2}{\mathrm{d}s^2}(s-p_1)^kF(s)\right|_{s=p_1}, \\
        \cdots                                                                                  \\
        k_{1i}=\left.\frac{1}{(i-1)!}\frac{\mathrm{d}^{i-1}}{\mathrm{d}s^{i-1}}(s-p_1)^kF(s)\right|_{s=p_1}.
    \end{cases}
\end{equation}

\begin{exampleprob}
    求 $F(s)=\dfrac{s-2}{s(s+1)^3}$ 的拉普拉斯逆变换.

    \begin{solution}
        对其部分分式展开:
        \begin{equation*}
            F(s)=\frac{k_{11}}{(s+1)^3}+\frac{k_{12}}{(s+1)^2}+\frac{k_{13}}{s+1}+\frac{k_2}{s}.
        \end{equation*}

        容易得到
        \begin{gather*}
            k_{11}=(s+1)^3F(s)|_{s=-1}=3, \\
            k_{12}=\left.\frac{\mathrm{d}(s+1)^3F(s)}{\mathrm{d}s}\right|_{s=-1}=2, \\
            k_{13}=\left.\frac{\mathrm{d}^2(s+1)^3F(s)}{\mathrm{d}s^2}\right|_{s=-1}=2, \\
            k_2=sF(s)|_{s=0}=-2.
        \end{gather*}

        因此
        \begin{align*}
                        & F(s)=\frac{3}{(s+1)^3}+\frac{2}{(s+1)^2}+\frac{2}{s+1}-\frac{2}{s} \\
            \Rightarrow & f(t)=\left(\frac{3}{2}t^2e^{-t}+2te^{-t}+2e^{-t}-2\right)u(t).
        \end{align*}
    \end{solution}
\end{exampleprob}
