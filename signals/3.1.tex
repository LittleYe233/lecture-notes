\section{傅里叶变换} \label{傅里叶变换}
\subsection{周期信号的傅里叶级数分析} \label{3 周期信号的傅里叶级数分析}
\subsubsection{三角函数形式的傅里叶级数}
以 $T$ 为周期, $\omega=\dfrac{2\pi}{T}$ 为角频率的函数 $f(t)$ 满足如下的狄利克雷条件
\begin{itemize}
    \item 一周期内, 连续或仅存在有限数目的第一类间断点;
    \item 一周期内, 极大值和极小值的数目有限;
    \item 一周期内, 信号绝对可积,
\end{itemize}
则该函数可展开为如下形式:
\begin{equation} \label{eq:3.1 fourier}
    f(t)=a_0+\sum_{n=1}^{\infty}[a_n\cos(n\omega t)+b_n\sin(n\omega t)],
\end{equation}
其中
\begin{gather}
    a_0=\frac{1}{T}\int_{0}^{T}f(t)\mathrm{d}t, \\
    a_n=\frac{2}{T}\int_{0}^{T}f(t)\cos(n\omega t)\mathrm{d}t, \\
    b_n=\frac{2}{T}\int_{0}^{T}f(t)\sin(n\omega t)\mathrm{d}t.
\end{gather}
积分区间也可以取 $-\dfrac{T}{2}\sim\dfrac{T}{2}$.

式 (\ref{eq:3.1 fourier}) 等价于
\begin{equation}
    f(t)=c_0+\sum_{n=1}^{\infty}c_n\cos(n\omega t+\varphi_n)=c_0+\sum_{n=1}^{\infty}c_n\sin(n\omega t+\theta_n),
\end{equation}
其中
\begin{gather}
    c_0=a_0, \\
    c_n=\sqrt{a_n^2+b_n^2}, \\
    \tan\varphi_n=-\frac{b_n}{a_n}, \\
    \tan\theta_n=\frac{a_n}{b_n}.
\end{gather}

\subsubsection{指数形式的傅里叶级数}
\rmg
\begin{equation} \label{eq:3.1 fourier exponential}
    f(t)=a_0+\sum_{n=1}^{\infty}\left(\frac{a_n-\mathrm{j}b_n}{2}e^{\mathrm{j}n\omega t}+\frac{a_n+\mathrm{j}b_n}{2}e^{-\mathrm{j}n\omega t}\right).
\end{equation}

令
\begin{equation} \label{eq:3.1 F n}
    \begin{aligned}
        F_n=F(n\omega) & =\begin{cases}
                              a_0,\quad                            & n=0 \\
                              \dfrac{1}{2}(a_n-\mathrm{j}b_n)\quad & n>0
                          \end{cases}                                      \\
                       & =\frac{1}{T}\int_{0}^{T}f(t)e^{-\mathrm{j}n\omega t}\mathrm{d}t,\qquad (n\geq 0)
    \end{aligned}
\end{equation}
则
\begin{equation} \label{eq:3.1 fourier exponential F}
    f(t)=\sum_{-\infty}^{\infty}F(n\omega)e^{\mathrm{j}n\omega t}.
\end{equation}

$F_n$ 与上述其他系数有如下关系:
\begin{gather}
    F_0=c_0=a_0, \\
    F_n=|F_n|e^{\mathrm{j}\varphi_n}, \\
    F_{-n}=|F_{-n}|e^{-\mathrm{j}\varphi_n}, \\
    |F_n|=|F_{-n}|=\frac{1}{2}c_n=\frac{1}{2}\sqrt{a_n^2+b_n^2}.
\end{gather}

\textbf{周期信号的功率特性}
\begin{equation}
    P=\overline{f^2(t)}=a_0^2+\frac{1}{2}\sum_{n=1}^{\infty}(a_n^2+b_n^2)=c_0^2+\frac{1}{2}\sum_{n=1}^{\infty}c_n^2=\sum_{-\infty}^{\infty}|F_n|^2.
\end{equation}

\subsubsection{函数的对称性与傅里叶级数的关系}
\textbf{偶函数}\quad 只包含直流或余弦分量.
\begin{gather}
    a_n=\frac{4}{T}\int_{0}^{\frac{T}{2}}f(t)\cos(n\omega t)\mathrm{d}t, \\
    b_n=0, \\
    c_n=a_n=2F_n=2F_{-n}, \\
    \varphi_n=0, \\
    \theta_n=\frac{\pi}{2}.
\end{gather}

\textbf{奇函数}\quad 只包含正弦分量.
\srmg
\begin{gather}
    a_0=a_n=0, \\
    b_n=\frac{4}{T}\int_{0}^{\frac{T}{2}}f(t)\sin(n\omega t)\mathrm{d}t, \\
    c_n=b_n=2\mathrm{j}F_n=-2\mathrm{j}F_{-n}, \\
    \varphi_n=-\frac{\pi}{2}, \\
    \theta_n=0.
\end{gather}

\textbf{奇谐函数}\quad $f\left(t\pm\dfrac{T}{2}\right)=-f(t)$. 只包含奇次谐波分量.
\srmg
\begin{gather}
    a_0=a_n=b_n=0,\qquad (2\mid n) \\
    a_n=\frac{4}{T}\int_{0}^{\frac{T}{2}}f(t)\cos(n\omega t)\mathrm{d}t,\qquad (2\nmid n) \\
    b_n=\frac{4}{T}\int_{0}^{\frac{T}{2}}f(t)\sin(n\omega t)\mathrm{d}t.\qquad (2\nmid n)
\end{gather}

\textbf{偶谐函数}\quad $f\left(t\pm\dfrac{T}{2}\right)=f(t)$. 只包含偶次谐波分量.

\subsubsection{傅里叶有限级数与最小方均误差}
取傅里叶级数 $f(t)$ 的前 $(2N+1)$ 项 $S_N(t)$
\begin{equation}
    S_N(t)=a_0+\sum_{n=1}^{N}[a_n\cos(n\omega t)+b_n\sin(n\omega t)].
\end{equation}

则 $S_N(t)$ 逼近 $f(t)$ 引起的误差函数为
\begin{equation}
    \epsilon_N(t)=f(t)-S_N(t).
\end{equation}
方均误差为
\begin{equation}
    \begin{aligned}
        E_N & =\overline{\epsilon_N^2(t)}=\frac{1}{T}\int_{0}^{T}\epsilon_N^2(t)\mathrm{d}t \\
            & =\overline{f^2(t)}-\left[a_0^2+\frac{1}{2}\sum_{n=1}^{N}(a_n^2+b_n^2)\right].
    \end{aligned}
\end{equation}

\subsubsection{吉布斯现象}
高频分量主要影响脉冲的跳变沿, 低频分量主要影响脉冲的顶部.

$f(t)$ 波形变化越剧烈, 所包含的高频分量越丰富; 变化越缓慢, 所包含的低频分量越丰富.
