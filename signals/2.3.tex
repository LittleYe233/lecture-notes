\subsection{零输入响应与零状态响应}

\textbf{零输入响应}\quad $r_{zi}(t)$. 没有外加激励信号的作用, 只由起始状态 (起始时刻系统储能) 所产生的响应.
\begin{gather}
    \sum_{i=0}^{n}C_i\dd{n-i}{t}r_{zi}(t)=0.\qquad (\textrm{起始状态约束}\ r^{(k)}_{zi}(0_{+}).) \\
    r^{(k)}_{zi}(0_{+})=r^{(k)}_{zi}(0_{-})=r^{(k)}(0_{-}).\qquad (\textrm{无激励作用, 系统内部结构不变.}) \label{eq:2.3 r_zi 0+}
\end{gather}

$r_{zi}(t)$ 是齐次解 (\ref{eq:2.1 lti model r_h}) 的一部分:
\begin{equation}
    r_{zi}(t)=\sum_{k=1}^{n}A_{zik}e^{\alpha_k t}.
\end{equation}

\textbf{零状态响应}\quad $r_{zs}(t)$. 不考虑起始时刻系统储能的作用 (起始状态等于零), 由系统外加激励信号所产生的响应.
\begin{gather}
    \sum_{i=0}^{n}C_i\dd{n-i}{t}r_{zs}(t)=\sum_{i=0}^{n}E_i\frac{\mathrm{d}^{n-i}e(t)}{\mathrm{d}t^{n-i}}. \qquad (\textrm{起始状态约束}\ r^{(k)}_{zs}(0_{+}).) \\
    r^{(k)}_{zs}(0_{+})=r^{(k)}(0_{+})-r^{(k)}(0_{-}). \label{eq:2.3 r_zs 0+}
\end{gather}

$r_{zs}(t)$ 包含齐次解 (\ref{eq:2.1 lti model r_h}) 的另一部分和特解:
\begin{equation}
    r_{zs(t)}=\sum_{k=1}^{n}A_{zsk}e^{\alpha_k t}+B(t).
\end{equation}

对于激励中含冲激函数及其导数的部分, 考虑到通常求解区间为 $0_+\leq t\leq +\infty$, 可以将其直接去除; 相应地, $u(t)$ 应替换为 $1$ (见例题 \ref{eprob:2.3 1}).

\begin{exampleprob} \label{eprob:2.3 1}
    求 $r''(t)+3r'(t)+2r(t)=2\delta(t)+6u(t)$ 在 $r(0_-)=2$, $r'(0_-)=0$ 下的零输入响应和零状态响应.

    \begin{solution}
        对于零输入响应 $r_{zi}(t)$, 其满足 $r''+3r'+2r=0$. 这是齐次微分方程, 且满足初始条件 $r_{zi}(0_+)=r(0_-)=2$, $r_{zi}'(0_+)=r'(0_-)=0$. 解得 $r_{zi}(t)=4e^{-t}-2e^{-2t}$.

        使用冲激函数匹配法, 可以求得 $t=0$ 时 $r$ 的跳变为 $0$, $r'$ 的跳变为 $2$. 对于零状态响应 $r_{zs}(t)$, 在\ {\color{red} $t>0$} 时其满足 $r''+3r'+2r=6$. 这是非齐次微分方程, 且满足初始条件 $r_{zs}(0_+)=r(0_+)-r(0_-)=0$, $r_{zs}'(0_+)=r'(0_+)-r'(0_-)=2$. 解得 $r_{zs}(t)=-4e^{-t}+e^{-2t}+3$.
    \end{solution}
\end{exampleprob}

\begin{exampleprob}
    一线性时不变系统, 在相同初始条件下, 当激励为 $e(t)$ 时, 全响应为 $r_1(t)=[2e^{-3t}+\sin(2t)]u(t)$; 当激励为 $2e(t)$ 时, 全响应为 $r_2(t)=[e^{-3t}+2\sin(2t)]u(t)$. 求
    \begin{enumerate}
        \item 初始条件不变, 当激励为 $e(t-t_0)$ 时的全响应 $r_3(t)$;
        \item 初始条件增加 1 倍, 当激励为 $0.5e(t)$ 时的全响应 $r_4(t)$.
    \end{enumerate}

    \begin{solution}
        设相同初始条件下, 当激励为 $e(t)$ 时, 零输入响应和零状态响应分别为 $r_{zi}(t)$ 和 $r_{zs}(t)$. 则
        \begin{equation*}
            \begin{cases}
                r_{zi}+r_{zs}(t)=r_1(t), \\
                r_{zi}+2r_{zs}(t)=r_2(t).
            \end{cases}
        \end{equation*}
        解得 $\begin{cases}
                r_{zi}(t)=2r_1(t)-r_2(t)=3e^{-3t}u(t), \\
                r_{zs}(t)=r_2(t)-r_1(t)=[-e^{-3t}+\sin(2t)]u(t).
            \end{cases}$

        \begin{enumerate}
            \item $r_3(t)=r_{zi}(t)+r_{zs}(t-t_0)=3e^{-3t}u(t)+[-e^{-3(t-t_0)}+\sin(2(t-t_0))]u(t-t_0)$.
            \item $r_4(t)=2r_{zi}(t)+0.5r_{zs}(t)=[5.5e^{-3t}+0.5\sin(2t)]u(t)$.
        \end{enumerate}
    \end{solution}
\end{exampleprob}

\begin{exampleprob}
    已知某 LTI 系统的微分方程为 $r''(t)+3r'(t)+2r(t)=e'(t)+3e(t)$, 其中 $e(t)=e^{-t}u(t)$ 为激励, $r(t)$ 为响应. 系统全响应为 $r(t)=[(2t+3)e^{-t}-2e^{-2t}]u(t)$.
    \begin{enumerate}
        \item 直接区分全响应中的强迫分量和自由分量, 并说明理由;
        \item 根据冲激函数匹配法求系统起始状态 $r(0_-)$ 和 $r'(0_-)$;
        \item 求系统的零输入响应和零状态响应.
    \end{enumerate}

    \begin{solution}
        \begin{enumerate}
            \item 题设微分方程的齐次解 (自由分量) 形式为 $r_h(t)=(A_1e^{-t}+A_2e^{-2t})u(t)$. 对照全响应可知 $r_h(t)=(3e^{-t}-2e^{-2t})u(t)$, 则强迫分量 $r_p(t)=2te^{-t}$.
            \item 代入 $e(t)=e^{-t}u(t)$, 得
                  \srmg
                  \begin{align*}
                      r''(t)+3r'(t)+2r(t) & =[e^{-t}(\delta(t)-u(t))]+3e^{-t}u(t)                                     \\
                                          & ={\color{red} e^{-t}}\delta(t)+2e^{-t}u(t)                                \\
                                          & =\delta(t)+2e^{-t}u(t).\qquad (\textrm{因为}\ \left.e^{-t}\right|_{t=0}=1.)
                  \end{align*}
                  当 $t=0$ 时, $r''(t)+3r'(t)+2r(t)=\delta(t)+2\Delta u(t)$. 显然 $r''(t)=\delta(t)$, $r'(t)=\Delta u(t)$, $r(t)=0$, 所以 $r'(0_+)-r'(0_-)=1$, $r(0_+)-r(0_-)=0$.

                  又因为 $r'(t)=[-(2t+1)e^{-t}+4e^{-2t}]u(t)$, 所以 $r(0_+)=1$, $r'(0_+)=3$. 因此, $r(0_-)=1$, $r'(0_-)=2$.
            \item 先求零输入响应. 设 $r_{zi}=(A_3e^{-t}+A_4e^{-2t})u(t)$, 且满足起始条件 $r(0_-)=1$, $r'(0_-)=2$. 解得 $r_{zi}=(4e^{-t}-3e^{-2t})u(t)$. 因此零状态响应为 $r_{zs}(t)=r(t)-r_{zi}(t)=[(2t-1)e^{-t}+e^{-2t}]u(t)$.

                  若先求零状态响应, 因为已求出强迫分量, 仅需再求出其中的齐次解即可. 设 $r_{zs}(t)=(A_5e^{-t}+A_6e^{-2t})+2te^{-t}$, 且满足边界条件 $r_{zs}(0_+)=r(0_+)-r(0_-)=0$, $r_{zs}'(0_+)=r'(0_+)-r'(0_-)=1$. 结果与上述相同.
        \end{enumerate}
    \end{solution}
\end{exampleprob}
