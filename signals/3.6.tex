\subsection{傅里叶变换的基本性质} \label{3 傅里叶变换的基本性质}
\subsubsection{对称性}
\rmg
\begin{equation} \label{eq:3.6 sym}
    \mathcal{F}[\mathcal{F}[f(t)]]=2\pi f(-\omega).
\end{equation}

\begin{exampleprob}
    已知 $\mathcal{F}[\mathrm{sgn}(t)]=\dfrac{2}{\mathrm{j}\omega}$, 求 $\mathcal{F}^{-1}[\mathrm{j}\pi\mathrm{sgn}(\omega)]$.

    \begin{solution}
        由 (\ref{eq:3.6 sym}), $\mathcal{F}[\mathcal{F}[\mathrm{sgn}(t)]]=2\pi\mathrm{sgn}(-\omega)=-2\pi\mathrm{sgn}(\omega)$, 所以 $\mathrm{j}\pi\mathrm{sgn}(\omega)=-\dfrac{\mathrm{j}}{2}\mathcal{F}[\mathcal{F}[\mathrm{sgn}(t)]]$, 所以\newline $\mathcal{F}^{-1}[\mathrm{j}\pi\mathrm{sgn}(\omega)]=-\dfrac{\mathrm{j}}{2}\mathcal{F}[\mathrm{sgn}(t)]=-\dfrac{1}{t}$.
    \end{solution}
\end{exampleprob}

\subsubsection{线性 (叠加性)}
\srmg
\begin{equation}
    \mathcal{F}\left[\sum_{i}a_if_i(t)\right]=\sum_{i}a_i\mathcal{F}[f_i(t)].
\end{equation}

\subsubsection{奇偶虚实性}
将傅里叶变换式 $\mathcal{F}[f(t)]=F(\omega)$ 拆分为实部 $R(\omega)$ 和虚部 $X(\omega)$.

\textbf{$f(t)$ 为实函数}
\begin{align}
    \nonumber   & \begin{aligned}
                      F(\omega) & =\int_{-\infty}^{+\infty}f(t)e^{-\mathrm{j}\omega t}\mathrm{d}t                                                        \\
                                & =\int_{-\infty}^{+\infty}f(t)\cos(\omega t)\mathrm{d}t-\mathrm{j}\int_{-\infty}^{+\infty}f(t)\sin(\omega t)\mathrm{d}t
                  \end{aligned} \\
    \Rightarrow & \begin{cases}
                      R(\omega)=\int_{-\infty}^{+\infty}f(t)\cos(\omega t)\mathrm{d}t, \\
                      X(\omega)=-\int_{-\infty}^{+\infty}f(t)\sin(\omega t)\mathrm{d}t.
                  \end{cases}
\end{align}

此时 $R(\omega)$ 为偶函数, $X(\omega)$ 为奇函数.

\textbf{$f(t)$ 为实偶函数}
\begin{equation}
    \begin{cases}
        X(\omega)=0, \\
        F(\omega)=R(\omega)=2\int_{0}^{+\infty}f(t)\cos(\omega t)\mathrm{d}t.
    \end{cases}
\end{equation}

\textbf{$f(t)$ 为实奇函数}
\begin{equation}
    \begin{cases}
        R(\omega)=0, \\
        F(\omega)=\mathrm{j}X(\omega)=-2\mathrm{j}\int_{0}^{+\infty}f(t)\sin(\omega t)\mathrm{d}t.
    \end{cases}
\end{equation}

\textbf{$f(t)$ 为虚函数}

不放假设 $f(t)=\mathrm{j}g(t)$. 则
\begin{equation}
    \begin{cases}
        R(\omega)=\int_{-\infty}^{+\infty}g(t)\sin(\omega t)\mathrm{d}t, \\
        X(\omega)=\int_{-\infty}^{+\infty}g(t)\cos(\omega t)\mathrm{d}t.
    \end{cases}
\end{equation}

此时 $R(\omega)$ 为奇函数, $X(\omega)$ 为偶函数.

\subsubsection{尺度变换特性}
\rmg
\begin{equation}
    \mathcal{F}[f(at)]=\frac{1}{|a|}\mathcal{F}[f]\left(\frac{\omega}{a}\right).\qquad (a\neq 0)
\end{equation}
特别的,
\begin{equation*}
    \mathcal{F}[f(-t)]=\mathcal{F}[f](-\omega).
\end{equation*}

信号在时域中压缩 (扩展) 等效于在频域中扩展 (压缩). 信号在时域中沿纵轴反褶等效于在频域中频谱沿纵轴反褶.

\subsubsection{时移特性}
\rmg
\begin{equation}
    \mathcal{F}[f(t-t_0)]=\mathcal{F}[f(t)]e^{-\mathrm{j}\omega t_0}.\qquad (t\in\mathbb{R})
\end{equation}

\subsubsection{频移特性}
\rmg
\begin{equation} \label{eq:3.6 freq move}
    \mathcal{F}[f(t)e^{\mathrm{j}\omega_0 t}]=\mathcal{F}[f](\omega-\omega_0).\qquad (\omega_0\in\mathbb{R})
\end{equation}

\textbf{频谱搬移}

将信号 $f(t)$ 乘以载频信号 $\cos(\omega_0 t)$ 或 $\sin(\omega_0 t)$. 因为
\begin{gather*}
    \cos(\omega_0 t)=\frac{1}{2}(e^{\mathrm{j}\omega_0 t}+e^{-\mathrm{j}\omega_0 t}), \\
    \sin(\omega_0 t)=\frac{1}{2\mathrm{j}}(e^{\mathrm{j}\omega_0 t}-e^{-\mathrm{j}\omega_0 t}),
\end{gather*}
所以
\begin{gather}
    \mathcal{F}[f(t)\cos(\omega_0 t)]=\frac{1}{2}[\mathcal{F}[f](\omega+\omega_0)+\mathcal{F}[f](\omega-\omega_0)], \label{eq:3.6 freq move cos} \\
    \mathcal{F}[f(t)\sin(\omega_0 t)]=\frac{\mathrm{j}}{2}[\mathcal{F}[f](\omega+\omega_0)-\mathcal{F}[f](\omega-\omega_0)]. \label{eq:3.6 freq move sin}
\end{gather}

\subsubsection{时域微分特性}
\rmg
\begin{equation}
    \mathcal{F}[f^{(n)}(t)]=(\mathrm{j}\omega)^n\mathcal{F}[f(t)].
\end{equation}

\subsubsection{频域微分特性}
\rmg
\begin{equation}
    \mathcal{F}[(-\mathrm{j}t)^nf(t)]=\frac{\mathrm{d}^n\mathcal{F}[f(t)]}{\mathrm{d}\omega^n}.
\end{equation}

\subsubsection{时域积分特性}
\rmg
\begin{equation}
    \mathcal{F}\left[\int_{-\infty}^{t}f(\tau)\mathrm{d}\tau\right]=\frac{\mathcal{F}[f(t)]}{\mathrm{j}\omega}+\pi\mathcal{F}[f(t)](0)\delta(\omega).
\end{equation}
