\section{拉普拉斯变换, 连续时间系统的 s 域分析} \label{拉普拉斯变换, 连续时间系统的 s 域分析}
\subsection{拉普拉斯变换的定义, 收敛域} \label{4 拉普拉斯变换的定义, 收敛域}
\subsubsection{(单边) 拉普拉斯变换}
拉普拉斯变换可以解决不满足绝对可积条件的函数通常不存在傅里叶变换的问题. 对原函数 $f(t)$ 乘以某一衰减因子 $e^{-\sigma t}$, 若其满足绝对可积条件, 则可以对其傅里叶变换.

考虑 $f(t)$ 为一因果信号, 初始状态为零, 则 $f(t<0)=0$ (由此求得的称为\textit{单边拉普拉斯变换}). $f(t)e^{-\sigma t}$ 的傅里叶变换为
\begin{equation}
    F_1(\omega)=\int_{0}^{+\infty}[f(t)e^{-\sigma t}]e^{-\mathrm{j}\omega t}\mathrm{d}t=\int_{0}^{+\infty}f(t)e^{-st}\mathrm{d}t.
\end{equation}
其中 $s=\sigma+\mathrm{j}\omega$. 改变自变量, 由此得到

\textbf{拉普拉斯正变换}
\begin{equation} \label{4.1 laplace}
    F(s)=\mathcal{L}[f(t)]=\int_{0}^{+\infty}f(t)e^{-st}\mathrm{d}t.
\end{equation}

若考虑 $t=0$ 时刻函数值跳变, 定积分下限取 $0_-$.

类似地, 利用傅里叶逆变换, 可以推得

\textbf{拉普拉斯逆变换}
\begin{equation} \label{4.1 laplace inverse}
    f(t)=\mathcal{L}^{-1}[F(s)]=\frac{1}{2\pi\mathrm{j}}\int_{\sigma-\mathrm{j}\infty}^{\sigma+\mathrm{j}\infty}F(s)e^{st}\mathrm{d}s.
\end{equation}

\textbf{因果性}\quad 单边拉普拉斯变换的结果与 $t<0$ 的函数值无关. 单边拉普拉斯逆变换只能给出 $t\geq 0$ 的结果, $t<0$ 的响应由接入激励前系统状态决定.

\subsubsection{拉普拉斯变换的收敛域}
若对于函数 $f(t)$, 存在 $\sigma_0$ 使得
\begin{equation}
    \lim_{t\rightarrow\infty}f(t)e^{-\sigma_0 t}=0,
\end{equation}
则 $f(t)$ 在 $\sigma>\sigma_0$ 内收敛, 也即可以进行拉普拉斯变换.

\begin{itemize}
    \item 有界非周期信号的拉普拉斯变换在 $\sigma\in\mathbb{R}$ 上存在;
    \item 周期信号的拉普拉斯变换在 $\sigma\in\mathbb{R}^*$ 上存在;
    \item 与 $t^n$ 成比例增长的信号的拉普拉斯变换在 $\sigma\in\mathbb{R}^*$ 上存在;
    \item 与 $e^{at}$ 成比例增长的信号的拉普拉斯变换在 $\sigma>a$ 上存在;
    \item 一些比指数函数增长更快的函数限定在\textit{有限时间}范围内可以进行拉普拉斯变换.
\end{itemize}

\subsubsection{一些常用函数的拉普拉斯变换}
% See: https://tex.stackexchange.com/a/35517/290833
\renewcommand{\arraystretch}{1.5}
% See: https://www.overleaf.com/latex/examples/a-longtable-example/xxwzfxkxxjmc
\begin{longtable}{@{\extracolsep{\fill}}cc@{}}
    \caption{一些常用函数的拉普拉斯变换} \label{tab:4.1 laplace}                       \\

    \hline \textbf{原函数 $f(t)$} & \textbf{象函数 $F(s)$}                      \\ \hline
    \endfirsthead

    \multicolumn{2}{r}{\em \footnotesize cont'd}                          \\
    \hline \textbf{原函数 $f(t)$} & \textbf{象函数 $F(s)$}                      \\ \hline
    \endhead

    \hline
    \endlastfoot

    $\delta(t)$                & 1                                        \\
    $u(t)$                     & $\dfrac{1}{s}$                           \\
    $e^{-at}$                  & $\dfrac{1}{a+s}$                         \\
    $t^n$\quad ($n\in N^*$)    & $\dfrac{n!}{s^{n+1}}$                    \\
    $\sin(\omega t)$           & $\dfrac{\omega}{s^2+\omega^2}$           \\
    $\cos(\omega t)$           & $\dfrac{s}{s^2+\omega^2}$                \\
    $t\sin(\omega t)$          & $\dfrac{2\omega s}{(s^2+\omega^2)^2}$    \\
    $t\cos(\omega t)$          & $\dfrac{s^2-\omega^2}{(s^2+\omega^2)^2}$ \\
    $\sinh(at)$                & $\dfrac{a}{s^2-a^2}$                     \\
    $\cosh(at)$                & $\dfrac{s}{s^2-a^2}$                     \\
\end{longtable}
\renewcommand{\arraystretch}{1}
