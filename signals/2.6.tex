\subsection{卷积的性质} \label{2 卷积的性质}

\subsubsection{卷积代数}

\textbf{交换律}
\begin{equation}
    f_1*f_2=f_2*f_1.
\end{equation}

\textbf{结合律}
\begin{equation}
    (f_1*f_2)*f_3=f_1*(f_2*f_3).
\end{equation}

串联系统的冲激响应, 等于组成串联系统的各子系统的冲激响应的卷积.

\textbf{分配律}
\begin{equation}
    f_1*(f_2+f_3)=f_1*f_2+f_1*f_3.
\end{equation}

并联系统的冲激响应, 等于组成并联系统的各子系统的冲激响应之和.

\subsubsection{卷积的微分和积分性质}
\rmg
\begin{gather}
    (f_1*f_2)'=f_1*f_2'=f_1'*f_2. \\
    \int_{-\infty}^{t}(f_1*f_2)=f_1*\int_{-\infty}^{t}f_2=f_2\int_{-\infty}^{t}f_1. \\
    f_1*f_2=f_1'*\int_{-\infty}^{t}f_2=f_2'*\int_{-\infty}^{t}f_1. \label{eq:2.6 conv calc 3}
\end{gather}

对于式 (\ref{eq:2.6 conv calc 3}), $f_1$ 和 $f_2$ 应满足时间受限条件, 也即 $t\rightarrow-\infty$ 时函数值为 0.

\subsubsection{与各阶冲激函数的卷积}
\rmg
\begin{equation}
    f(t)*\delta^{(k)}(t-t_0)=f^{(k)}(t-t_0).
\end{equation}

其中 $k$ 表示 $\delta(t)$ 的阶数, 正整数表示导数阶数, 负整数表示积分次数. 特别的,
\begin{gather}
    f(t)*\delta(t)=f(t). \label{eq:2.6 f(t)*delta(t)} \\
    f(t)*\delta'(t)=f'(t). \\
    f(t)*u(t)=\int_{-\infty}^{t}f(\tau)\mathrm{d}\tau.
\end{gather}
