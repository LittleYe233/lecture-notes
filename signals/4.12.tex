\subsection{拉普拉斯变换与傅里叶变换的关系} \label{4 拉普拉斯变换与傅里叶变换的关系}
以下讨论针对单边拉普拉斯变换和因果信号. 设信号 $f(t)u(t)$ 的收敛域为 $\sigma>\sigma_0$, 则
\begin{itemize}
    \item \underline{$\sigma_0<0$}: 傅里叶变换\textit{可以}直接将单边拉普拉斯变换中 $s$ 替换为 $\mathrm{j}\omega$;
    \item \underline{$\sigma_0=0$}: 傅里叶变换\textit{不能}直接将单边拉普拉斯变换中 $s$ 替换为 $\mathrm{j}\omega$;
    \item \underline{$\sigma_0>0$}: 傅里叶变换\textit{不存在}.
\end{itemize}

其中, 对于 $\sigma_0=0$ 的情况, 设某信号 $f(t)$ 的单边拉普拉斯变换式为
\begin{equation}
    F(s)=F_a(s)+\sum_{n=1}^{N}\frac{K_n}{s-\mathrm{j}\omega_n},
\end{equation}
其中 $F_a(s)$ 表示位于虚轴左侧的极点的部分分式之和, 后项表示位于虚轴上的 $N$ 个极点的部分分式之和, $K_n$ 表示对应的 $N$ 个极点的部分分式系数, 则该信号的傅里叶变换为
\begin{equation}
    \mathcal{F}[f(t)]=F(s)|_{s=\mathrm{j}\omega}+\sum_{n=1}^{N}K_n\pi\delta(\omega-\omega_n).
\end{equation}
也即, 收敛边界位于虚轴的信号的傅里叶变换, 一部分为其单边拉普拉斯变换将 $s$ 替换为 $\mathrm{j}\omega$ 的结果, 另一部分是 $N$ 个冲激函数之和.

进一步, 当该信号在虚轴上存在 $k$ 重极点 $\mathrm{j}\Omega_0$, 部分分式系数为 $A_0$, 则其傅里叶变换中对应项应改为 $\pi\dfrac{A_0\mathrm{j}^{k-1}}{(k-1)!}\delta^{(k-1)}(\omega-\Omega_0)$.

\begin{exampleprob}
    已知 $tu(t)$ 的单边拉普拉斯变换为 $\dfrac{1}{s^2}$. 求其傅里叶变换.

    \begin{solution}
        $tu(t)$ 的单边拉普拉斯变换的收敛域为 $\sigma>0$, 其虚轴上有一个二阶极点 $s=0$, 因此 $tu(t)$ 的傅里叶变换为
        \begin{equation*}
            \mathcal{F}[tu(t)]=\frac{1}{(\mathrm{j}\omega)^2}+\pi\frac{1\mathrm{j}}{1!}\delta'(\omega-0)=-\frac{1}{\omega^2}+\mathrm{j}\pi\delta'(\omega).
        \end{equation*}
    \end{solution}
\end{exampleprob}
