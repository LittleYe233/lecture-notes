\section{附录} \label{附录}
\subsection{图片索引} \label{图片索引}
\makeatletter
\@starttoc{lof}
\makeatother

\subsection{表格索引} \label{表格索引}
\makeatletter
\@starttoc{lot}
\makeatother

\subsection{例题索引} \label{例题索引}
\listofexampleprobs

\subsection{疑问索引} \label{疑问索引}
\listofquestions

\subsection{更新日志} \label{更新日志}
\subsubsection*{1.3.0 (2023-3-23)}
\begin{itemize}
    \item 增加 \ref{3 傅里叶变换的基本性质} 到 \ref{3 抽样定理} 的部分, 以及其他细微更改;
    \item 增加 \hyperref[proofs 符号函数的傅里叶变换]{符号函数的傅里叶变换} 部分的推导过程及说明;
    \item 增加 2 道例题;
    \item 增加 8 张图片;
    \item 将正余弦函数的傅里叶变换推导移至 \ref{3 周期信号的傅里叶变换}.
\end{itemize}

\subsubsection*{1.2.0 (2023-3-23)}
\begin{itemize}
    \item 增加 \ref{3 周期信号的傅里叶级数分析} 到 \ref{3 冲激函数和阶跃函数的傅里叶变换} 的部分, 以及其他细微更改.
\end{itemize}

\subsubsection*{1.1.0 (2023-3-14)}
\begin{itemize}
    \item 增加 \ref{连续时间系统的时域分析} 的部分, 以及其他细微更改;
    \item 增加 \hyperref[部分推导过程及说明]{部分推导过程及说明} 的部分;
    \item 增加 9 道例题;
    \item 增加 1 张外源图片;
    \item 删除所有的课前预习思考题;
    \item 修改部分图片的大小和位置;
    \item 将所有的外源图片用 \LaTeX\ TikZ/CircuiTikZ 生成.
\end{itemize}

\subsubsection*{1.0.0 (2023-3-3)}
\begin{itemize}
    \item 增加 \hyperref[课程信息]{课程信息}, \ref{绪论} 的部分;
    \item 增加 \hyperref[附录]{附录} 的部分;
    \item 增加 8 道课前预习思考题;
    \item 增加 1 道例题;
    \item 增加 2 张外源图片.
\end{itemize}
