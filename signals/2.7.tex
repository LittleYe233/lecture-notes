\subsection{用算子符号表示微分方程} \label{2 用算子符号表示微分方程}

\subsubsection{算子符号的基本规则}
\rmg
\begin{gather}
    p^n(\cdot)=\frac{\mathrm{d}^n(\cdot)}{\mathrm{d}t^n}. \\
    \frac{1}{p}(\cdot)=\int_{-\infty}^{t}(\cdot)\mathrm{d}\tau. \\
    D(p)=\sum_{i=0}^{n}C_ip^{n-i}. \\
    N(p)=\sum_{i=0}^{m}E_ip^{m-i}.
\end{gather}

\textbf{与代数方程的异同}
\begin{itemize}
    \item 算子多项式可以进行类似于代数运算的因式分解或因式相乘展开;
    \item 算子多项式等式两端的公共因式不能随意相消;
    \item 算子多项式的算子乘除顺序不可随意颠倒;
          \begin{gather}
              p\frac{1}{p}x(t)=\frac{\mathrm{d}}{\mathrm{d}t}\int_{-\infty}^{t}x(t)\mathrm{d}t=x(t). \\
              \frac{1}{p}px(t)=\int_{-\infty}^{t}\frac{\mathrm{d}}{\mathrm{d}t}x(t)\mathrm{d}t=x(t)-x(-\infty).
          \end{gather}
\end{itemize}

\subsubsection{传输算子 \texorpdfstring{$H(p)$}{H(p)}}
\rmg
\begin{gather}
    H(p)=\frac{N(p)}{D(p)}. \\
    r(t)=H(p)[e(t)].
\end{gather}
