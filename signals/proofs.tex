\section{部分推导过程及说明} \label{部分推导过程及说明}

\subsection{矩形脉冲卷积形成梯形脉冲 (三角形脉冲)} \label{proofs 矩形脉冲卷积形成梯形脉冲 (三角形脉冲)}

由 \hyperref[2.5 梯形脉冲 (三角形脉冲)]{梯形脉冲 (三角形脉冲)} 小节的说明, 设 $f_1(t)=E_1G_{\tau_1}(t)$, $f_2(t)=E_2G_{\tau_2}(t)$, 并不妨设 $\tau_1>\tau_2$. 显然, $f_1$ 和 $f_2$ 分别表示高度为 $E_1$, 宽度为 $\tau_1$ 和高度为 $E_2$, 宽度为 $\tau_2$ 的矩形脉冲. 根据卷积的数学定义 (\ref{eq:2.5 conv}),
\begin{equation}
    [f_1*f_2](t)=\int_{-\infty}^{\infty}E_1E_2G_{\tau_1}(\tau)G_{\tau_2}(t-\tau)\mathrm{d}\tau.
\end{equation}

先确定 $f_1(\tau)f_2(t-\tau)\neq 0$ 时 $t$ 的取值范围. 显然 $f_1(\tau)$ 和 $f_2(t-\tau)$ 取值非零的充分必要条件分别为
\begin{gather*}
    -\frac{\tau_1}{2}<\tau<\frac{\tau_1}{2}, \\
    -\frac{\tau_2}{2}<t-\tau<\frac{\tau_2}{2}.
\end{gather*}
也即
\begin{gather}
    -\frac{\tau_1}{2}<\tau<\frac{\tau_1}{2}, \label{eq:proofs conv trapezoid tau 1} \\
    t-\frac{\tau_2}{2}<\tau<t+\frac{\tau_2}{2}. \label{eq:proofs conv trapezoid tau 2}
\end{gather}

若要 $f_1(\tau)f_2(t-\tau)\neq 0$, 则需上述两个条件求得的 $t$ 的解集存在交集. 现考虑其反面, 若要使该交集为空集, 仅需满足
\begin{equation}
    t-\frac{\tau_2}{2}\geq\frac{\tau_1}{2}
\end{equation}
或
\begin{equation}
    t+\frac{\tau_2}{2}\leq-\frac{\tau_1}{2}.
\end{equation}
则该交集为
\begin{equation*}
    \left\{t\left|t-\frac{\tau_2}{2}<\frac{\tau_1}{2}, t+\frac{\tau_2}{2}>-\frac{\tau_1}{2}\right.\right\}.
\end{equation*}
也即, $f_1(\tau)f_2(t-\tau)\neq 0$ 时 $t$ 的取值范围为
\begin{equation}
    -\frac{\tau_1+\tau_2}{2}<t<\frac{\tau_1+\tau_2}{2}.
\end{equation}
这也表示所形成的梯形脉冲 (三角形脉冲) 的下底范围为 $-\dfrac{\tau_1+\tau_2}{2}<t<\dfrac{\tau_1+\tau_2}{2}$, 宽度恰为 $\tau_1+\tau_2$.

再确定卷积积分的上下限. 因为 $\tau_1>\tau_2$, 所以联立 (\ref{eq:proofs conv trapezoid tau 1}) 和 (\ref{eq:proofs conv trapezoid tau 2}) 后 $\tau$ 满足的集合符合如下的分类讨论:

\underline{Case \#1:} 当
\begin{equation*}
    \begin{cases}
        t+\dfrac{\tau_2}{2}>-\dfrac{\tau_1}{2}, \\
        t-\dfrac{\tau_2}{2}<-\dfrac{\tau_1}{2},
    \end{cases}
\end{equation*}
也即
\begin{equation}
    -\frac{\tau_1+\tau_2}{2}<t<-\frac{\tau_1-\tau_2}{2}
\end{equation}
时,
\begin{equation}
    -\frac{\tau_1}{2}<\tau<t+\frac{\tau_2}{2}.
\end{equation}

\underline{Case \#2:} 当
\begin{equation*}
    \begin{cases}
        t+\dfrac{\tau_2}{2}<\dfrac{\tau_1}{2}, \\
        t-\dfrac{\tau_2}{2}>-\dfrac{\tau_1}{2},
    \end{cases}
\end{equation*}
也即
\begin{equation}
    -\frac{\tau_1-\tau_2}{2}<t<\frac{\tau_1-\tau_2}{2}
\end{equation}
时,
\begin{equation}
    t-\frac{\tau_2}{2}<\tau<t+\frac{\tau_2}{2}.
\end{equation}

\underline{Case \#3:} 当
\begin{equation*}
    \begin{cases}
        t+\dfrac{\tau_2}{2}>\dfrac{\tau_1}{2}, \\
        t-\dfrac{\tau_2}{2}<\dfrac{\tau_1}{2},
    \end{cases}
\end{equation*}
也即
\begin{equation}
    \frac{\tau_1-\tau_2}{2}<t<\frac{\tau_1+\tau_2}{2}
\end{equation}
时,
\begin{equation}
    t-\frac{\tau_2}{2}<\tau<\frac{\tau_1}{2}.
\end{equation}

因此, 卷积积分的值, 也即所形成的脉冲的表达式为
\begin{align}
    \nonumber [f_1*f_2](t) & =\int_{\tau_\mathrm{min}}^{\tau_\mathrm{max}}E_1E_2\mathrm{d}\tau                                                      \\
    \nonumber              & =E_1E_2(\tau_\mathrm{max}-\tau_\mathrm{min})                                                                           \\
                           & =\begin{cases}
                                  \displaystyle
                                  E_1E_2\left(t+\frac{\tau_1+\tau_2}{2}\right),\quad & \displaystyle -\frac{\tau_1-\tau_2}{2}<t<\frac{\tau_1-\tau_2}{2} \\
                                  \displaystyle
                                  E_1E_2\tau_2,\quad                                 & \displaystyle -\frac{\tau_1-\tau_2}{2}<t<\frac{\tau_1-\tau_2}{2} \\
                                  \displaystyle
                                  E_1E_2\left(\frac{\tau_1+\tau_2}{2}-t\right).\quad & \displaystyle \frac{\tau_1-\tau_2}{2}<t<\frac{\tau_1+\tau_2}{2}
                              \end{cases}
\end{align}

显然, 表达式上所形成的脉冲符合上文所述的特征.

\subsection{符号函数的傅里叶变换} \label{proofs 符号函数的傅里叶变换}
符号函数本身不满足绝对可积条件, 却存在傅里叶变换. 借助双边指数衰减函数 $f(t)=e^{-a|t|}\ (a>0)$, 令 $f_1(t)=f(t)\mathrm{sgn}(t)$. 则
\begin{equation}
    \begin{aligned}
        F_1(\omega)=\mathcal{F}[f_1(t)] & =\int_{-\infty}^{0}-e^{at}\cdot e^{-\mathrm{j}\omega t}\mathrm{d}t+\int_{0}^{+\infty}e^{-at}\cdot e^{-\mathrm{j}\omega t}\mathrm{d}t                                    \\
                                        & =\left.-\frac{1}{a-\mathrm{j}\omega}e^{(a-\mathrm{j}\omega)t}\right|_{-\infty}^{0}+\left.\frac{1}{-(a+\mathrm{j}\omega)}e^{-(a+\mathrm{j}\omega)t}\right|_{0}^{+\infty} \\
                                        & =-\frac{1}{a-\mathrm{j}\omega}+\frac{1}{a+\mathrm{j}\omega}                                                                                                             \\
                                        & =-\frac{2\mathrm{j}\omega}{a^2+\omega^2}.
    \end{aligned}
\end{equation}

因此, 符号函数 $\mathrm{sgn}(t)$ 的频谱 $F(\omega)$ 为
\begin{equation}
    F(\omega)=\lim_{a\rightarrow 0}F_1(\omega)=\lim_{a\rightarrow 0}\left(-\frac{2\mathrm{j}\omega}{a^2+\omega^2}\right)=\frac{2}{\mathrm{j}\omega}.
\end{equation}

\subsection{正弦信号激励下系统的稳态响应} \label{proofs 正弦信号激励下系统的稳态响应}
若已知某系统的系统函数为 $H(s)=|H(s)|e^{\mathrm{j}\varphi(\omega)}$. 设正弦信号激励的表达式为 $e(t)=E_m\sin(\omega t)$. 显然激励的拉普拉斯变换为
\begin{equation}
    E(s)=\frac{E_m\omega}{s^2+\omega^2}.
\end{equation}

一般系统函数的拉普拉斯变换为有理分式. 不妨设
\begin{equation}
    H(s)=\frac{KA(s)}{(s-p_1)(s-p_2)\cdots(s-p_n)}.
\end{equation}
也即 $H(s)$ 的幅度为 $K$, 有 $n$ 个极点 $p_1,p_2,\cdots,p_n$.

此时, 系统的全响应的拉普拉斯变换为
\begin{equation}
    R(s)=E(s)H(s)=\frac{E_mK\omega A(s)}{(s^2+\omega^2)(s-p_1)(s-p_2)\cdots(s-p_n)}.
\end{equation}
可以看出, 全响应有 $(n+2)$ 个极点, 多出的两个极点 $p_{a,b}=\pm\mathrm{j}\omega$ 来自激励.

对于来自系统函数的 $n$ 个极点代表的分量, 经过拉普拉斯逆变换后其应具有如下的形式
\begin{equation}
    r_{H}(t)=k_1e^{p_1t}+k_2e^{p_2t}+\cdots+k_ne^{p_nt}.
\end{equation}
仅考虑系统稳定的情形. 此时显然 $p_1,p_2,\cdots,p_n<0$, 所以 $r_H(t)$ 是指数衰减振荡, 不属于稳态响应.

对于 $p_{a,b}$, 显然其是共轭复极点. 使用部分分式法, 求其部分分式系数为
\begin{equation}
    k_C=(s-\mathrm{j}\omega)R(s)|_{s=\mathrm{j}\omega}=\frac{E_mKA(\mathrm{j}\omega)}{2\mathrm{j}(\mathrm{j}\omega-p_1)(\mathrm{j}\omega-p_2)\cdots(\mathrm{j}\omega-p_n)}=\frac{E_mH(\mathrm{j}\omega)}{2\mathrm{j}}.
\end{equation}
不妨设 $H(\mathrm{j}\omega)=P+\mathrm{j}Q$, 则
\begin{equation}
    k_C=\frac{E_m(P+\mathrm{j}Q)}{2\mathrm{j}}=\frac{E_m}{2}(Q-\mathrm{j}P).
\end{equation}
设 $k_C=A+\mathrm{j}B$, 显然
\begin{equation}
    \begin{cases}
        A=\frac{E_mB_0}{2}, \\
        B=\frac{E_mA_0}{2}.
    \end{cases}
\end{equation}

因此, 激励所产生的响应在时域上应为
\begin{equation}
    \begin{aligned}
        r_E(t) & =E_m[Q\cos(\omega t)+P\sin(\omega t)]                                                \\
               & =E_m\sqrt{P^2+Q^2}\sin(\omega t+\varphi) \qquad \left(\tan\varphi=\frac{Q}{P}\right) \\
               & =E_m|H(\mathrm{j}\omega)|\sin[\omega t+\varphi(\omega)].
    \end{aligned}
\end{equation}
显然这是稳态响应. 此即式 (\ref{eq:4.7 r ss}).
