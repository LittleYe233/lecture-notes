\section{电磁场问题的求解} \label{3}
\subsection{静电场}
\subsubsection{静电场的基本方程和边界条件}
静电场问题的基本方程为
\begin{equation} \label{eq:3.1 1}
    \begin{cases}
        \nabla\cdot\bm{D}=\rho, \\
        \nabla\times\bm{E}=0.
    \end{cases} \qquad
    \begin{cases}
        \oint_S\bm{D}\cdot\mathrm{d}\bm{S}=Q, \\
        \oint_C\bm{E}\cdot\mathrm{d}\bm{l}=0.
    \end{cases}
\end{equation}
其中存在本构关系 $\bm{D}=\epsilon\bm{E}$.

静电场问题的边界条件 (无需考虑初始条件) 为
\begin{equation} \label{eq:3.1 0}
    \begin{cases}
        \bm{e_n}\cdot(\bm{D_1}-\bm{D_2})=\rho_S, \qquad \text{\color{blue} (\question[为什么要如此表示?])} \\
        \bm{e_n}\times(\bm{E_1}-\bm{E_2})=0.
    \end{cases}
\end{equation}

容易得到 (\ref{eq:3.1 1}) 可以化为
\begin{equation} \label{eq:3.1 2}
    \bm{E}=-\nabla\varphi(\bm{r}). \qquad \text{(因为 $\nabla\times(\nabla u)\equiv 0$.)}
\end{equation}

注意到 (\ref{eq:3.1 1}) 和本构关系式可推出
\begin{equation}
    \nabla\cdot(\epsilon\bm{E})=\epsilon(\bm{r})\nabla\cdot\bm{E}(\bm{r})+\bm{E}(\bm{r})\nabla\cdot\epsilon(\bm{r})=\rho.
\end{equation}
假设在各向同性均匀线性介质中, 则有
\begin{equation} \label{eq:3.1 3}
    \epsilon\nabla\cdot\bm{E}=\rho.
\end{equation}

将 (\ref{eq:3.1 2}) 代入 (\ref{eq:3.1 3}), 得到
\begin{equation} \label{eq:3.1 4}
    \nabla^2\varphi(\bm{r})=-\frac{\rho}{\epsilon}.
\end{equation}
上式 (\ref{eq:3.1 4}) 是一个关于电位 $\varphi(\bm{r})$ 的泊松方程. 给定边界条件之后, 即可得到唯一解. 若空间中无电荷, 则
\begin{equation}
    \nabla^2\varphi(\bm{r})=0.
\end{equation}
这是一个拉普拉斯方程.

\subsubsection{静电位的边界条件} \label{3.1.2}
\textbf{介质分界面}

设 $P_1,P_2$ 是介质分界面两侧紧贴界面的相邻两点, 则
\begin{align*}
                & \varphi_1-\varphi_2  =\lim_{\Delta l\rightarrow 0}\int_{P_1}^{P_2}\bm{E}\mathrm{d}\bm{l}=0 \\
    \Rightarrow & \varphi_1            =\varphi_2.
\end{align*}

同时, 由 (\ref{eq:3.1 0}),
\begin{equation} \label{eq:3.1 5}
    \bm{e_n}(-\epsilon_1\nabla\varphi_1+\epsilon_2\nabla\varphi_2)=\epsilon_2\frac{\partial\varphi_2}{\partial n}-\epsilon_1\frac{\partial\varphi_1}{\partial n}=\rho_S.
\end{equation}
若分界面上无自由电荷 ($\rho_S=0$), 则
\begin{equation}
    \epsilon_2\frac{\partial\varphi_2}{\partial n}=\epsilon_1\frac{\partial\varphi_1}{\partial n}.
\end{equation}

\textbf{导体分界面}

设导体的介电常数为 $\epsilon_2$. 因为导体内部场强为零, 所以
\begin{equation}
    \nabla\varphi=0.
\end{equation}

由式 (\ref{eq:3.1 5}),
\begin{equation}
    \epsilon_1\frac{\partial\varphi_1}{\partial n}=-\rho_S. \qquad \text{\color{blue} (\question[为什么消去了 $\epsilon_2$?] \question[为什么 $\varphi_1$ 的偏微分不为零?])}
\end{equation}

\begin{exampleprob}
    真空中, 两块无限大接地导体平板分别置于 $x=0$ 和 $x=a$ 处, 在两板之间的 $x=b$ 处有一面密度 $\rho_{S0}$ 的均匀电荷分布的板. 求两导体平板之间的电位和电场.

    \begin{solution}
        设 $0<x<b$ 和 $b<x<a$ 之间的电位分别为 $\varphi_1(x),\varphi_2(x)$. 除 $x=b$ 外其余空间无电荷分布, 故电位函数满足一维拉普拉斯方程:
        \begin{equation*}
            \begin{cases}
                \frac{\mathrm{d}^2\varphi_1(x)}{\mathrm{d}x^2}=0, & \quad 0<x<b; \\
                \frac{\mathrm{d}^2\varphi_2(x)}{\mathrm{d}x^2}=0, & \quad b<x<a.
            \end{cases}
        \end{equation*}

        由数理方程知识, 方程组的解为
        \begin{equation*}
            \begin{cases}
                \varphi_1(x)=C_1x+D_1, \\
                \varphi_2(x)=C_2x+D_2.
            \end{cases}
        \end{equation*}

        考虑边界条件,
        \begin{equation*}
            \begin{cases}
                \varphi_1(0)=0,                                                                                                                                                                        \\
                \varphi_2(a)=0,                                                                                                                                                                        \\
                \varphi_1(b)=\varphi_2(b),                                                                                                                           & \quad \text{(介质分界面)}            \\
                \left.\left(\frac{\mathrm{d}\varphi_2(x)}{\mathrm{d}x}-\frac{\mathrm{d}\varphi_1(x)}{\mathrm{d}x}\right)\right|_{x=b}=-\frac{\rho_{S0}}{\epsilon_0}. & \quad \text{(式 \ref{eq:3.1 5})}
            \end{cases}
        \end{equation*}

        综上, 解得
        \begin{equation*}
            \begin{cases}
                C_1=-\frac{\rho_{S0}(b-a)}{\epsilon_0a}, \\
                D_1=0,                                   \\
                C_2=-\frac{\rho_{S0}b}{\epsilon_0a},     \\
                D_2=\frac{\rho_{S0}b}{\epsilon_0}.
            \end{cases}
        \end{equation*}

        因此,
        \begin{equation*}
            \begin{cases}
                \varphi_1(x)=\frac{\rho_{S0}(a-b)}{\epsilon_0a}x, & \quad 0<x<b; \\
                \varphi_2(x)=\frac{\rho_{S0}b}{\epsilon_0a}(a-x), & \quad b<x<a.
            \end{cases}
        \end{equation*}
        并且
        \begin{equation*}
            \begin{cases}
                \bm{E_1}(x)=-\frac{\rho_{S0}(a-b)}{\epsilon_0a}\bm{e_x}, & \quad 0<x<b; \\
                \bm{E_2}(x)=\frac{\rho_{S0}b}{\epsilon_0a}\bm{e_x},      & \quad b<x<a.
            \end{cases}
        \end{equation*}
    \end{solution}
\end{exampleprob}
