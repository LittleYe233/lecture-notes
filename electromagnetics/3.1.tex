\section{电磁场问题的求解}
\subsection{静电场}
静电场问题的基本方程为
\begin{equation} \label{eq:3.1 1}
    \nabla\times\bm{E}=0.
\end{equation}

容易得到 (\ref{eq:3.1 1}) 可以化为
\begin{equation} \label{eq:3.1 2}
    \bm{E}=-\nabla\varphi(\bm{r}).
\end{equation}

注意到在电介质中
\begin{equation}
    \nabla\cdot\bm{D}=\rho
\end{equation}
也即
\begin{equation} \label{eq:3.1 3}
    \nabla\cdot\epsilon\bm{E}=\rho.
\end{equation}

将 (\ref{eq:3.1 2}) 代入 (\ref{eq:3.1 3}), 得到
\begin{equation} \label{eq:3.1 4}
    \nabla^2\varphi(\bm{r})=-\frac{\rho}{\epsilon}.
\end{equation}

上式 (\ref{eq:3.1 4}) 是一个泊松方程. 给定边界条件之后, 即可得到唯一解.

\subsection{稳磁场}
稳磁场问题的基本方程为
\begin{equation}
    \nabla\cdot\bm{B}=0.
\end{equation}

容易得到 $\bm{B}$ 可以表示为
\begin{equation}
    \bm{B}=\nabla\times\bm{A}.
\end{equation}

而
\begin{equation}
    \nabla\times\bm{H}=\frac{1}{\mu}[\nabla(\nabla\cdot\bm{A})-\nabla^2\bm{A}].
\end{equation}
令 $\nabla\cdot\bm{A}=0$, 则
\begin{equation}
    \nabla\times\bm{H}=-\frac{1}{\mu}\nabla^2\bm{A}=\bm{J}.
\end{equation}

因此
\begin{equation}
    \nabla^2\bm{A}=-\mu_0\bm{J}.
\end{equation}

这是三个方向上的泊松方程. 给定边界条件之后, 即可得到唯一解.

\subsection{时变磁场}
时变磁场条件下, 基本方程如下
\begin{equation}
    \begin{cases}
        \nabla^2\varphi=\frac{1}{\mu\epsilon}\frac{\partial^2\varphi}{\partial t^2}-\frac{\rho}{\epsilon}; \\
        \nabla^2\bm{A}=\frac{1}{\mu\epsilon}\frac{\partial^2 A}{\partial t^2}-\mu\bm{J}.
    \end{cases}
\end{equation}
