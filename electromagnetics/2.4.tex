\subsection{法拉第电磁感应定律}
运动的电流切割磁感线产生的\textbf{感应电动势}满足
\begin{equation}
    \mathcal{E}=-\frac{\mathrm{d}}{\mathrm{d}t}\int\bm{B}\cdot\mathrm{d}\bm{S}.
\end{equation}

导体回路中存在感应电流, 则存在\textbf{感应电场}:
\begin{equation}
    \mathcal{E}=\oint_C\bm{E}\cdot\mathrm{d}\bm{l}.
\end{equation}
即, 对空间中的任意回路, 都满足
\begin{equation}
    \oint_C\bm{E}\cdot\mathrm{d}\bm{l}=-\frac{\mathrm{d}}{\mathrm{d}t}\int\bm{B}\cdot\mathrm{d}\bm{S}.
\end{equation}
因为静电荷产生的电场是无旋场, 因此其对 $C$ 的线积分为 0, 所以 $\bm{E}$ 可表示感应电场和静电场的\textit{总和}.

\subsubsection{引起回路磁通量变化的分类}
\textbf{回路静止, 磁场时变}

此时
\begin{equation}
    \mathcal{E}=-\int\frac{\partial\bm{B}}{\partial t}\cdot\mathrm{d}\bm{S}.
\end{equation}
也即
\begin{equation}
    \oint_C\bm{E}\cdot\mathrm{d}\bm{l}=-\int\frac{\partial\bm{B}}{\partial t}\mathrm{d}\bm{S}.
\end{equation}

根据斯托克斯定理 (\ref{eq:1.4 stokes}), 可以推得
\begin{equation}
    \nabla\times\bm{E}=-\frac{\partial\bm{B}}{\partial t}.
\end{equation}

\textbf{回路在恒定磁场中运动}

磁通量的微分形式可以表为
\begin{equation}
    \bm{B}\cdot\mathrm{d}\bm{S}=-(\bm{v}\times\bm{B})\cdot\mathrm{d}\bm{l}\mathrm{d}t.
\end{equation}
则
\begin{equation}
    \mathcal{E}=\oint_C(\bm{v}\times\bm{B})\cdot\mathrm{d}\bm{l}.
\end{equation}

\textbf{回路在时变磁场中运动}

感应电动势为
\begin{equation}
    \mathcal{E}=\oint_C\bm{E}\cdot\mathrm{d}\bm{l}=\oint_C(\bm{v}\times\bm{B})\cdot\mathrm{d}\bm{l}-\int_S\frac{\partial\bm{B}}{\partial t}\mathrm{d}\bm{S}.
\end{equation}
其微分形式为
\begin{equation}
    \nabla\times\bm{E}=\nabla\times(\bm{v}\times\bm{B})-\frac{\partial\bm{B}}{\partial t}.
\end{equation}
