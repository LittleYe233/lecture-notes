\subsection{电位}
电位的引入位于式 (\ref{eq:3.1 2}). 此处介绍电位的基本概念, 而将边界问题置于第 \ref{3.1.2} 章.

\subsubsection{电位的定义}
静电场可以用一个标量函数的梯度表示,
\begin{equation}
    \bm{E}(\bm{r})=-\nabla\varphi(\bm{r}).
\end{equation}
定义\textit{标量函数} $\varphi$ 为静电场的\textbf{电位}. 电位函数不是唯一的.

\subsubsection{电位差}
将单位点电荷从 P 点移动到 Q 点电场力做功为
\begin{equation}
    W_{PQ}=\int_{P}^{Q}q\bm{E}\cdot\mathrm{d}\bm{l}=\varphi(P)-\varphi(Q).
\end{equation}

定义 P, Q 两点之间的电位差为
\begin{equation}
    U_{PQ}\stackrel{def}{=}\varphi(P)-\varphi(Q)=W_{PQ}.
\end{equation}

电位差是一定值, 只与首尾两点位置有关, 与积分路径无关.

一般设无穷远处电位为零, 则 P 点的电位表示为
\begin{equation}
    \varphi(P)=\int_{P}^{\infty}\bm{E}\cdot\mathrm{d}\bm{l}.
\end{equation}

\textbf{点电荷周围的电位}
\begin{equation}
    \varphi(\bm{r})=\frac{q}{4\pi\epsilon_0|\bm{r}|}+C.
\end{equation}
