\section{绪论} \label{绪论}
\subsection{无线电信号传输原理}
\subsubsection{传输信号的基本方法}
\begin{figure}[H]
    \centering
    \begin{tikzpicture}
        \node (1) [normal] {信号源};
        \node (2) [normal, right=of 1] {发送设备};
        \node (3) [normal, right=of 2] {传输信道};
        \node (4) [normal, right=of 3] {接收设备};
        \node (5) [normal, right=of 4] {收信装置};

        \draw [-latex] (1) edge (2)
        (2) edge (3)
        (3) edge (4)
        (4) edge (5);
    \end{tikzpicture}
    \caption{通信系统组成框图}
\end{figure}

\begin{itemize}
    \item \textbf{发送设备}\quad 转换为适合传输的信号.
    \item \textbf{收信装置}\quad 扬声器, 等.
\end{itemize}

\subsubsection{无线电信号的产生与发射}
\begin{figure}[H]
    \centering
    \begin{tikzpicture}
        \node (1) [normal] {高频振荡};
        \node (2) [normal, right=of 1] {倍频};
        \node (3) [normal, right=of 2] {高频放大};
        \node (4) [normal, right=of 3] {调制};
        \node (5) [normal, below=of 2] {音频源};
        \node (6) [normal, right=of 5] {音频放大};
        \node (7) [right=1.5cm of 4] {};

        \draw [-latex] (1) -- (2) node [midway, above] {缓冲};
        \draw [-latex] (2) edge (3)
        (3) edge (4)
        (5) edge (6)
        (6) -| (4);
        \draw [-latex] (4) -- (7) node [midway, above] {传输线};
    \end{tikzpicture}
    \caption{无线电发射机方框图}
\end{figure}

\textbf{调制}\quad 将声音电流加在高频电流上.
\begin{itemize}
    \item \textbf{分类}\quad 调幅 (AM, 高频振幅变化所形成的包络波形即为原信号波形), 调频 (FM), 调相 (PM).
    \item \textbf{原因}
          \begin{itemize}
              \item 传输低频信号需要更庞大的天线;
              \item 减少干扰, 提高信道复用率 (将信号放大到不同的频段).
          \end{itemize}
\end{itemize}

\subsubsection{无线电信号的接收}
\textbf{接收设备}要求尽可能减少失真.

\begin{figure}[H]
    \centering
    \begin{tikzpicture}
        \node (1) [] {};
        \node (2) [normal, right=1.5cm of 1] {选择性电路};
        \node (3) [normal, right=of 2] {检波};
        \node (4) [normal, right=of 3] {发声装置};

        \draw [-latex] (1) -- (2) node [midway, above] {传输线};
        \draw [-latex] (2) edge (3)
        (3) edge (4);
    \end{tikzpicture}
    \caption{最简单的接收机方框图}
\end{figure}

\begin{figure}[H]
    \centering
    \begin{tikzpicture}
        \node (1) [] {};
        \node (2) [normal, right=1.5cm of 1] {高频放大};
        \node (3) [normal, right=of 2] {混频};
        \node (4) [normal, right=of 3] {中频放大};
        \node (5) [normal, right=of 4] {检波};
        \node (6) [normal, right=of 5] {低频放大};
        \node (7) [normal, below=of 3] {本地振荡};

        \draw [-latex] (1) -- (2) node [midway, above] {传输线};
        \draw [-latex] (2) edge (3)
        (3) edge (4)
        (4) edge (5)
        (5) edge (6)
        (7) edge (3);
    \end{tikzpicture}
    \caption{超外差式接收机方框图}
\end{figure}

\subsection{通信的传输媒质}
\begin{itemize}
    \item 有线通信
          \begin{itemize}
              \item 双线对电缆 (双绞线; 频率低)
              \item 同轴电缆
              \item 光纤 (频率高, 衰减小)
          \end{itemize}
    \item 无线通信
          \begin{itemize}
              \item 地波
                    \begin{itemize}
                        \item 地面波 (电磁波沿地面绕射传输; 传播距离近, 频率低)
                        \item 空间波 (由直射波, 反射波组成; 视距范围内)
                    \end{itemize}
              \item 天波 (通过空中电离层的折射反射传播, 穿透电离层, 更高频率会携带水蒸气)
          \end{itemize}
\end{itemize}
