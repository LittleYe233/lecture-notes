\section{部分推导过程及说明} \label{部分推导过程及说明}

\subsection{耦合振荡回路的谐振}

\subsubsection{部分谐振}

部分谐振下的条件 (\ref{eq:2.4 primary resonance cond}), (\ref{eq:2.4 secondary resonance cond}) 分别可表为
\begin{gather}
    X_{11}=\frac{X_M^2X_{22}}{R_{22}^2+X_{22}^2}, \label{eq:proofs primary resonance cond} \\
    X_{22}=\frac{X_M^2X_{11}}{R_{11}^2+X_{11}^2}. \label{eq:proofs secondary resonance cond}
\end{gather}

初级等效回路谐振时, 初级回路电流 (\ref{eq:2.4 I_1}) 达到最大值; 因为次级回路参数不变, 次级回路电流 (\ref{eq:2.4 I_2}) 与初级回路电流呈正比.
\begin{gather}
    I_{1\mathrm{max}}=\frac{V_1}{R_{11}+\dfrac{X_M^2R_{22}}{R_{22}^2+X_{22}^2}}, \\
    I_{2\mathrm{max}}=\frac{X_MV_1}{|Z_{22}|R_{11}+\dfrac{X_M^2R_{22}}{|Z_{22}|}}. \label{eq:proofs primary resonance I_2max}
\end{gather}

次级等效回路谐振时, 次级回路电流达到最大值; 初级等效电路不作考虑.
\begin{equation}
    I_{2\mathrm{max}}=\frac{X_MV_1}{|Z_{11}|R_{22}+\dfrac{X_M^2R_{11}}{|Z_{11}|}}.
\end{equation}

\subsubsection{复谐振}

不妨认为产生复谐振的一个充分条件为\textit{初级等效回路部分谐振} (\ref{eq:2.4 primary resonance cond}) 且\textit{对初级等效回路的反射阻抗满足最大功率传输条件} (\ref{eq:2.4 complex resonance power cond}). 前者即 (\ref{eq:proofs primary resonance cond}), 后者可表为
\begin{equation} \label{eq:proofs complex resonance power cond}
    R_{11}=\frac{X_M^2R_{22}}{R_{22}^2+X_{22}^2}.
\end{equation}

不妨假设 $X_{11}X_{22}R_{11}R_{22}\neq 0$. (\ref{eq:proofs complex resonance power cond}) 除以 (\ref{eq:proofs primary resonance cond}), 得
\begin{equation} \label{eq:proofs R_11/X_11}
    \frac{R_{11}}{X_{11}}=\frac{R_{22}}{X_{22}}.
\end{equation}

此时在 (\ref{eq:proofs primary resonance cond}) 两边同乘 $X_{22}$, 并将 (\ref{eq:proofs R_11/X_11}) 代入之, 则
% See: https://tex.stackexchange.com/a/475846/290833
\begin{alignat*}{3}
                &  & X_{11}X_{22} & =\frac{X_M^2X_{22}^2}{R_{22}^2+X_{22}^2} &  & =\frac{X_M^2X_{11}^2}{R_{11}^2+X_{11}^2} \\
    \Rightarrow &  & X_{22}       &                                          &  & =\frac{X_M^2X_{11}}{R_{11}^2+X_{11}^2}.
\end{alignat*}

\noindent 此即 (\ref{eq:proofs secondary resonance cond}). 同理将 (\ref{eq:proofs R_11/X_11}) 代入 (\ref{eq:proofs secondary resonance cond}) 可得到
\begin{equation} \label{eq:proofs complex resonance secondary power cond}
    R_{22}=\frac{X_M^2R_{11}}{R_{11}^2+X_{11}^2}.
\end{equation}

一些资料中, 将 (\ref{eq:proofs primary resonance cond}) 和 (\ref{eq:proofs complex resonance power cond}) 视作 ``初级回路复谐振'' 的充分必要条件, 而将 (\ref{eq:proofs secondary resonance cond}) 和 (\ref{eq:proofs complex resonance secondary power cond}) 视作 ``次级回路复谐振'' 的充分必要条件. 从上述推导中, 由其中一对条件即可推出另一对条件.

在初级等效回路谐振的条件下, 若同时满足最大功率传输条件, 次级回路电流将达到最大值. 将 (\ref{eq:proofs complex resonance power cond}) 表为
\begin{equation} \label{eq:proofs X_M/|Z_22|}
    \frac{X_M}{|Z_{22}|}=\sqrt{\dfrac{R_{11}}{R_{22}}}.
\end{equation}

将 (\ref{eq:proofs complex resonance power cond}), (\ref{eq:proofs X_M/|Z_22|}) 代入 (\ref{eq:proofs primary resonance I_2max}), 得到
\begin{equation}
    I_{2\mathrm{max}}=\frac{X_MV_1}{2|Z_{22}|R_{11}}=\frac{V_1}{|R_{11}|}\sqrt\dfrac{R_{11}}{R_{22}}=\frac{V_1}{2\sqrt{R_{11}R_{22}}}=I_{2\mathrm{max,max}}.
\end{equation}

\noindent 此即 (\ref{eq:2.4 complex resonance I_2}).

\subsubsection{全谐振}

注意到全谐振时, $X_{11}=X_{22}=0$, 易知 $X_{f1}=X_{f2}=0$, 此时形式上也满足部分谐振的条件 (\ref{eq:2.4 primary resonance cond}), (\ref{eq:2.4 secondary resonance cond}). 若仅仅满足这两个条件, 且 $X_{11}X_{22}R_{11}R_{22}\neq 0$, 则两式相除, 整理之后即可推得 (\ref{eq:proofs R_11/X_11}), 继而反推得到 (\ref{eq:proofs complex resonance power cond}) 和 (\ref{eq:proofs complex resonance secondary power cond}), 也即与复谐振等价. 但显然在 $X_{11}=X_{22}=0$ 时, 以上推导不能成立, 也即 $R_{11}=R_{f1}$, $R_{22}=R_{f2}$ 不一定成立.

\subsubsection{最佳全谐振}

由 (\ref{eq:2.4 complete resonance cond}), (\ref{eq:proofs complex resonance power cond}), 推得
\begin{alignat*}{2}
                &  & R_{11} & =\frac{(\omega M_c)^2R_{22}}{R_{22}^2}=\frac{(\omega M_c)^2}{R_{22}} \\
    \Rightarrow &  & M_c    & =\frac{\sqrt{R_{11}R_{22}}}{\omega}.
\end{alignat*}

\noindent 此即 (\ref{eq:2.4 M_c}).

\subsection{耦合振荡回路的相对抑制比}

以互感耦合回路为例. 根据 (\ref{eq:2.4 basic equations and currents}), 在给定的耦合回路条件下,
\begin{gather} \label{eq:proofs basic equations}
    \dot{V}_1=\dot{I}_1(R+\mathrm{j}X)-\dot{I}_2(\mathrm{j}\omega M), \\
    0=\dot{I}_2(R+\mathrm{j}X)-\dot{I}_1(\mathrm{j}\omega M).
\end{gather}

根据 \hyperref[eq:2.1 xi]{广义失谐} 和 (\ref{eq:2.4 eta}),
\begin{gather}
    \xi=\frac{X_{11}}{R_{11}}=\frac{X_{22}}{R_{22}}=\frac{X}{R}, \label{eq:proofs coupled circuit xi} \\
    \eta=\frac{X_M}{\sqrt{R_{11}R_{22}}}=\frac{X_M}{R}. \label{eq:proofs coupled circuit eta}
\end{gather}

将 (\ref{eq:proofs coupled circuit xi}) 代入 (\ref{eq:proofs basic equations}), 得
\begin{gather}
    \dot{V}_1=\dot{I}_1R(1+\mathrm{j}\xi)-\mathrm{j}X_M\dot{I}_2, \\
    0=\dot{I}_2R(1+\mathrm{j}\xi)-\mathrm{j}X_M\dot{I}_1.
\end{gather}

解得
\begin{equation}
    \dot{I}_2=\frac{\mathrm{j}X_M\dot{V}_1}{R^2(1+\mathrm{j}\xi)^2+X_M^2}.
\end{equation}

将 (\ref{eq:proofs coupled circuit eta}) 代入上式,
\begin{equation}
    \dot{I}_2=\frac{\mathrm{j}\dfrac{X_M}{R}\dot{V}_1}{R\left[(1+\mathrm{j}\xi)^2+\dfrac{X_M^2}{R^2}\right]}=\frac{\mathrm{j}\eta\dot{V}_1}{R[(1+\mathrm{j}\xi)^2+\eta^2]}=\frac{\mathrm{j}\eta\dot{V}_1}{R[(1+\eta^2-\xi^2)+\mathrm{j}(2\eta)]}.
\end{equation}

取模, 得
\begin{equation}
    I_2=I_2(\xi,\eta)=\frac{\eta V_1}{R\sqrt{(1+\eta^2-\xi^2)^2+4\eta^2}}.
\end{equation}

$I_2$ 可看作关于 $\xi$ 和 $\eta$ 的二元函数. 在\textit{工作频率在谐振频率附近小范围变化}时, 可以将 $\xi$ 和 $\eta$ 看作两个独立自变量. 分别对其求偏导,
\begin{gather}
    \frac{\partial I_2}{\partial\xi}=\frac{V_{1} \eta \left[2 \xi \left(\eta^{2} - \xi^{2} + 1\right) - 4 \xi\right]}{R \left[4 \xi^{2} + \left(\eta^{2} - \xi^{2} + 1\right)^{2}\right]^{\frac{3}{2}}}. \\
    \frac{\partial I_2}{\partial\eta}=- \frac{2 V_{1} \eta^{2} \left(\eta^{2} - \xi^{2} + 1\right)}{R \left[4 \xi^{2} + \left(\eta^{2} - \xi^{2} + 1\right)^{2}\right]^{\frac{3}{2}}} + \frac{V_{1}}{R \sqrt{4 \xi^{2} + \left(\eta^{2} - \xi^{2} + 1\right)^{2}}}.
\end{gather}

令上式值为零, 解得 $\eta=1$, $\xi=0$, 此时 $I_2$ 达到最大值 $\dfrac{V_1}{2R}$, 即 (\ref{eq:2.4 complex resonance I_2}).

\subsection{混合 \texorpdfstring{$\pi$}{pi} 等效电路参数与形式等效电路 y 参数的转换} \label{proofs 混合 pi 等效电路参数与形式等效电路 y 参数的转换}

参照图 \ref{fig:3.2 transistor mixed pi simplified}. 分别短路输出端和输入端, 可以求得混合 $\pi$ 等效模型对应的 y 参数.

\subsubsection{输入导纳 \texorpdfstring{$y_{ie}$}{y ie}}
直接求忽略信号源的阻容网络的导纳:
\begin{equation}
    y_{ie}=\frac{1}{r_{bb'}+\dfrac{1}{\color{red} \mathrm{j}\omega(C_{b'e}+C_{b'c})}}=\frac{1}{r_{bb'}+\dfrac{1}{\color{red} Y_{b'e}}}=\frac{Y_{b'e}}{1+r_{bb'}Y_{b'e}}.
\end{equation}
其中 $Y_{b'e}=\mathrm{j}\omega(C_{b'e}+C_{b'c})$ 为从 b'e 向右看去的导纳.

\subsubsection{正向传输导纳 \texorpdfstring{$y_{fe}$}{y fe}}
短路 $\dot{V}_2$, 将 $\dot{I}_2$ 用 $\dot{V}_1$ 表示:
\srmg
\begin{gather}
    \dot{I}_2=g_m\dot{V}_{b'e}+\dot{I}. \\
    \dot{V}_{b'e}=\frac{\dot{V}_1}{r_{bb'}+\dfrac{1}{Y_{b'e}}}\cdot\frac{1}{Y_{b'e}}=\frac{\dot{V}_1}{1+r_{bb'}Y_{b'e}}.
\end{gather}

因为 $C_{b'c}$ 很小, $\dot{I}$ 相较于 $g_m\dot{V}_{b'e}$ 可以忽略, 所以
\begin{equation}
    y_{fe}=\frac{\dot{I}_2}{\dot{V}_1}\approx\frac{g_m\dot{V}_{b'e}}{\dot{V}_1}=\frac{g_m}{1+r_{bb'}Y_{b'e}}.
\end{equation}
其中 $Y_{b'e}=\mathrm{j}\omega(C_{b'e}+C_{b'c})$ 为从 b'e 向右看去的导纳.

\subsubsection{反向传输导纳 \texorpdfstring{$y_{ie}$}{y ie}}
短路 $\dot{V}_1$, 将 $\dot{I}_1$ 用 $\dot{V}_2$ 表示:
\begin{gather}
    \dot{I}_1=-\frac{\dot{V}_{b'e}}{r_{bb'}}. \\
    V_{b'e}=\frac{\dot{I}}{\dfrac{1}{r_{bb'}}+{\color{red} \mathrm{j}\omega C_{b'e}}}=\frac{\dot{I}r_{bb'}}{1+{\color{red} Y_{b'e}}r_{bb'}}. \label{eq:proofs transistor mixed pi y ie V b'e} \\
    \dot{I}=\frac{\dot{V}_2}{\dfrac{1}{\mathrm{j}\omega C_{b'e}}+\dfrac{1}{\dfrac{1}{r_{bb'}}+Y_{b'e}}}\approx \mathrm{j}\omega C_{b'c}\dot{V}_2. \label{eq:proofs transistor mixed pi y ie I} \\
    y_{ie}=\frac{\dot{I}_2}{\dot{V}_1}=-\frac{\mathrm{j}\omega C_{b'c}}{1+r_{bb'}Y_{b'e}}.
\end{gather}
其中 $Y_{b'e}=\mathrm{j}\omega C_{b'e}$ 为从 b'e 向左看去的导纳.

\subsubsection{输出导纳 \texorpdfstring{$y_o$}{y o}}
利用 (\ref{eq:proofs transistor mixed pi y ie V b'e}) 和 (\ref{eq:proofs transistor mixed pi y ie I}), 短路 $\dot{V}_1$, 将 $\dot{I}_2$ 用 $\dot{V}_2$ 表示:
\begin{equation}
    y_o=\frac{\dot{I}_2}{\dot{V}_2}=\frac{\dot{I}+g_m\dot{V}_{b'e}}{\dot{V}_2}=\mathrm{j}\omega C_{b'c}\left(1+\frac{g_mr_{bb'}}{1+r_{bb'}Y_{b'e}}\right).
\end{equation}
其中 $Y_{b'e}=\mathrm{j}\omega C_{b'e}$ 为从 b'e 向左看去的导纳.

\subsection{高频小信号放大器含源含负载回路的 y 参数} \label{proofs:高频小信号放大器含源含负载回路的 y 参数}
具体图略. 可以参考图 \ref{fig:3.2 transistor y param}, 在两端分别并联含阻信号源和负载即可, 图其他部分包括电流标识不变.

显然式 (\ref{eq:3.2 y param}) 仍然成立. 考虑到在共射极电路中, 在 y 参数上均加入脚标 e. 同时负载 $Y_L$ 两端电压即晶体管 ce 侧电压 $\dot{V}_2$, 则
\begin{equation}
    \dot{I}_2=-Y_L\dot{V}_2.
\end{equation}

联立以上各式, 解得
\begin{equation}
    Y_i=\frac{\dot{I}_1}{\dot{V}_1}=y_{ie}-\frac{y_{re}y_{fe}}{y_{oe}+Y_L}.
\end{equation}

对于输出导纳, 需要将信号电流源\textit{开路}, 因而
\begin{equation}
    \dot{I}_1=-Y_s\dot{V}_1.
\end{equation}

联立以上各式, 解得
\begin{equation}
    Y_o=\frac{\dot{I}_2}{\dot{V}_2}=y_{oe}-\frac{y_{re}y_{fe}}{y_{ie}+Y_L}.
\end{equation}
