\subsection{耦合回路} \label{耦合回路}

\textbf{耦合回路}\quad 由两个或两个以上的电路形成的一个网络, 且两个电路之间必须有\textbf{公共阻抗} (纯电阻, 纯电感, 互感, 纯电容, LC 复耦合等) 存在.

\textbf{初级回路}\quad 接有激励信号源的回路.

\textbf{次级回路}\quad 与负载相接的回路.

\textbf{耦合系数}\quad 耦合回路的公共电抗 (或电阻) 绝对值与初次级回路中同性质电抗 (或电阻) 的几何中项之比.
\begin{equation}
    k=\frac{|X_{12}|}{\sqrt{X_{11}X_{22}}}\leq 1.
\end{equation}

\textbf{电容耦合}
\begin{equation}
    k=\frac{C_M}{\sqrt{(C_1+C_M)(C_2+C_M)}}\approx\frac{C_M}{C}<1.\qquad (C_1=C_2=C,C_M<<C.)
\end{equation}

\textbf{电感耦合}
\begin{equation}
    k=\frac{M}{\sqrt{L_1L_2}}=\frac{M}{L}.\qquad (L_1=L_2=L.)
\end{equation}

\subsubsection{互感耦合回路的一般性质}

\begin{figure}[H]
    \begin{center}
        \begin{minipage}{.625\textwidth}
            \centering
            \includegraphics[width=\textwidth]{images/mutual_inductance_coupled_circuit.pdf}
            \caption{互感耦合回路的一般形式}
        \end{minipage}
        \begin{minipage}{.365\textwidth}
            \centering
            \includegraphics[width=\textwidth]{images/mutual_inductance_coupled_circuit_primary.pdf}
            \caption{初级等效电路}
        \end{minipage}
    \end{center}
\end{figure}

\begin{figure}[H]
    \begin{center}
        \begin{subfigure}{.2868\textwidth}
            \centering
            \includegraphics[width=\textwidth]{images/mutual_inductance_coupled_circuit_secondary_a.pdf}
            \caption{}
        \end{subfigure}
        \begin{subfigure}{.3132\textwidth}
            \centering
            \includegraphics[width=\textwidth]{images/mutual_inductance_coupled_circuit_secondary_b.pdf}
            \caption{}
        \end{subfigure}
        \caption{次级等效电路的两种形式}
    \end{center}
\end{figure}

以下结论对电容耦合回路也适用.
\begin{gather} \label{eq:2.4 basic equations and currents}
    \dot{V}_1=\dot{I}_1{\color{red} (Z_1+\mathrm{j}\omega L_1)}-\dot{I}_2(\mathrm{j}\omega M)=\dot{I}_1{\color{red} Z_{11}}-\mathrm{j}\omega M\dot{I}_2. \\
    0=\dot{I}_2{\color{red} (Z_2+\mathrm{j}\omega L_2)}-\dot{I}_1(\mathrm{j}\omega M)=\dot{I}_2{\color{red} Z_{22}}-\mathrm{j}\omega M\dot{I}_1. \\
    \dot{I}_1=\frac{\dot{V}_1}{Z_{11}+Z_{f1}}. \label{eq:2.4 I_1}  \\
    \dot{I}_2=\frac{\mathrm{j}\omega M\dot{I}_1}{Z_{22}}=\frac{\mathrm{j}\omega M\dfrac{\dot{V}_1}{Z_{11}}}{Z_{22}+Z_{f2}}. \label{eq:2.4 I_2}
\end{gather}

\textbf{自阻抗}\quad 电路中包含单侧耦合电感的总阻抗.
\begin{gather}
    Z_{11}=Z_1+\mathrm{j}\omega L_1=R_{11}+\mathrm{j}X_{11}. \\
    Z_{22}=Z_2+\mathrm{j}\omega L_2=R_{22}+\mathrm{j}X_{22}.
\end{gather}

\textbf{反射阻抗 (耦合阻抗)}\quad 另一侧回路的电流通过互感作用, 在本侧回路中感应的电动势对本侧电流的影响, 可用等效的阻抗表示.
\begin{gather}
    Z_{f1}=\frac{(\omega M)^2}{Z_{22}}=\frac{X_M^2R_{22}}{R_{22}^2+X_{22}^2}-\mathrm{j}\frac{X_M^2X_{22}}{R_{22}^2+X_{22}^2}. \\
    Z_{f2}=\frac{(\omega M)^2}{Z_{11}}=\frac{X_M^2R_{11}}{R_{11}^2+X_{11}^2}-\mathrm{j}\frac{X_M^2X_{11}}{R_{11}^2+X_{11}^2}.
\end{gather}

反射阻抗用以形象地表示解耦后的电路中一级对另一级电路的影响, 并不实际存在.

反射阻抗的电阻始终是正值 (始终产生\textit{能量损耗}), 电抗与本侧回路电抗符号相反 (\textit{容感性相反}).

次级回路中消耗的功率等于初级回路电流流过\textit{反射阻抗}的\textit{电阻}部分 $R_{f1}$ 所消耗的功率.

\textbf{次级回路开路时初级电流感应产生的电动势}\quad $\dot{V}_2=\mathrm{j}\omega M\dfrac{\dot{V}_1}{Z_{11}}$.

\textbf{考虑反射阻抗后的回路总阻抗}
\begin{gather}
    Z_{e1}=Z_{11}+Z_{f1}=\left[R_{11}+\frac{X_M^2}{R_{22}^2+X_{22}^2}R_{22}\right]+\mathrm{j}\left[X_{11}-\frac{X_M^2}{R_{22}^2+X_{22}^2}X_{22}\right]. \\
    Z_{e2}=Z_{22}+Z_{f2}=\left[R_{22}+\frac{X_M^2}{R_{11}^2+X_{11}^2}R_{11}\right]+\mathrm{j}\left[X_{22}-\frac{X_M^2}{R_{11}^2+X_{11}^2}X_{11}\right].
\end{gather}

\subsubsection{耦合振荡回路的谐振}

调谐时可以变化的电路参量为 $L_1$, $L_2$, $M$. 电路工作频率不变. 不妨记 $X_M=\omega M$.

\textbf{初级部分谐振}\quad 仅初级等效回路谐振. \label{2.4 primary resonance}

\underline{条件}\quad 仅调整 $L_1$:
\begin{equation} \label{eq:2.4 primary resonance cond}
    X_{11}+X_{f1}=0.
\end{equation}

\underline{结果}\quad 仅调整 $L_1$ 条件下, 初级和次级回路电流均达到最大值.

\textbf{次级部分谐振}\quad 仅次级等效回路谐振. \label{2.4 secondary resonance}

\underline{条件}\quad 仅调整 $L_2$:
\begin{equation} \label{eq:2.4 secondary resonance cond}
    X_{22}+X_{f2}=0.
\end{equation}

\underline{结果}\quad 仅调整 $L_2$ 条件下, 次级回路电流达到最大值.

\textbf{复谐振}\quad \textit{部分谐振}时, 满足\textit{最大功率传输条件}. \label{2.4 complex resonance}

\question[单级回路谐振及最大功率传输条件能否保证另一级回路谐振]
\underline{条件}\quad 满足 (\ref{eq:2.4 primary resonance cond}); 仅调整 $M$:
\begin{equation} \label{eq:2.4 complex resonance power cond}
    R_{11}=R_{f1}.
\end{equation}

\underline{结果}\quad 次级回路电流达到最大值; 仅调整 $L_1$, $M$ 条件下, 初级回路电流达到最大值.
\begin{equation} \label{eq:2.4 complex resonance I_2}
    I_{2\mathrm{max,max}}=\frac{V}{2\sqrt{R_{11}R_{22}}}.
\end{equation}

\textbf{全谐振}\quad 两个回路单独达到谐振. \label{2.4 complete resonance}

\underline{条件}\quad 仅调整 $L_1$, $L_2$:
\begin{equation} \label{eq:2.4 complete resonance cond}
    X_{11}=X_{22}=0.
\end{equation}

此时不保证 $R_{11}=R_{f1}$, $R_{22}=R_{f2}$.

\textbf{最佳全谐振}\quad \textit{全谐振}时, 满足\textit{最大功率传输条件}. \label{2.4 best complete resonance}

\underline{条件}\quad 满足 (\ref{eq:2.4 complete resonance cond}); 仅调整 $M$, 满足 (\ref{eq:2.4 complex resonance power cond}).
\begin{equation} \label{eq:2.4 M_c}
    M_c=\frac{\sqrt{R_{11}R_{22}}}{\omega}.
\end{equation}

\underline{结果}\quad 次级回路电流达到最大值, 即 (\ref{eq:2.4 complex resonance I_2}).

\textbf{临界耦合系数}\quad 最佳全谐振时的耦合系数.
\begin{equation}
    k_c=\frac{M_c}{\sqrt{L_1L_2}}=\sqrt{\dfrac{R_{11}R_{22}}{\omega^2L_{11}L_{22}}}\approx\frac{1}{\sqrt{Q_1Q_2}}.
\end{equation}

\textbf{耦合因数}\quad \textit{双谐振回路}中, 表示耦合相对强弱的参量.
\begin{equation} \label{eq:2.4 eta}
    \eta=\frac{k}{k_c}=k\sqrt{Q_1Q_2}=\frac{X_M}{\sqrt{R_{11}R_{22}}}.
\end{equation}

\subsubsection{耦合振荡回路的频率特性}

\textbf{频率特性}\quad 回路各参数不变, 改变工作频率 $\omega$, 次级回路电流 $I_2$ 的特性.

假设两级互感耦合回路的参数和工作状态相同 ($R_1=R_2=R$, $L_1=L_2=L$, $C_1=C_2=C$, $Q_1=Q_2=Q$, $\omega_{01}=\omega_{02}=\omega_0$, $\xi_1=\xi_2=\xi$).

\textbf{相对抑制比}\quad 耦合谐振回路谐振曲线的通用表示式. (对\textit{任何单一电抗耦合形式}, \textit{任何形式的调谐方法}都适用; 信号频率只能在\textit{谐振频率}附近小范围变化.)
\begin{equation}
    \alpha=\frac{I_2}{I_{2\mathrm{max,max}}}=\frac{2\eta}{\sqrt{(1-\xi^2+\eta^2)^2+4\xi^2}}.
\end{equation}

\begin{figure}[H]
    \centering
    \includegraphics[width=.7\textwidth]{images/mutual_coupled_circuit_resonance_curve.pdf}
    \caption{耦合谐振回路通用形式的谐振曲线}
\end{figure}

\textbf{欠耦合 (弱耦合)}\quad $\eta<1$, 单峰 $\alpha_\mathrm{max}=\dfrac{2\eta}{1+\eta^2}$, $\dfrac{\mathrm{d}\alpha}{\mathrm{d}\eta}>0$.

\textbf{临界耦合}\quad 最佳全谐振时的耦合. $\eta=1$, $\alpha_\mathrm{max}=1$, $\alpha=\dfrac{2}{\sqrt{4+\xi^2}}$.

通频带边缘 $\alpha=\dfrac{1}{\sqrt{2}}$, $\xi=\pm\sqrt{2}$. 通频带 $B=\sqrt{2}\dfrac{f_0}{Q}$. (双谐振回路通频带是单谐振回路的\ {\color{red} $\sqrt{2}$} 倍.)

\textbf{过耦合 (强耦合)}\quad $\eta>1$, 双峰 $\alpha_\mathrm{max}=1$ (复谐振), 此时 $\xi=\pm\sqrt{\eta^2-1}$; 谷值 $\delta=\dfrac{2\eta}{1+\eta^2}$.

当\ {\color{red} $\delta\geq\dfrac{1}{\sqrt{2}}$} 时, 通频带边缘 $\xi=\pm\sqrt{\eta^2+2\eta-1}$, 通频带 $B=\sqrt{\eta^2+2\eta-1}\dfrac{f_0}{Q}$.

当 $\delta=\dfrac{1}{\sqrt{2}}$ 时, $\eta\approx 2.41$, $\xi\approx 3.1$ (双谐振回路通频带是单谐振回路的\ {\color{red} 3.1} 倍.)

实际应用中, 为使\textit{通频带内放大均匀}, \textit{带外衰减较大}, 可取 $\eta=1.5$.
