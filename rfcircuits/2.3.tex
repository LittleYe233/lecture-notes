\subsection{串, 并联阻抗的等效互换与回路抽头时的阻抗变换} \label{串, 并联阻抗的等效互换与回路抽头时的阻抗变换}
\subsubsection{串, 并联阻抗的等效互换}
\begin{figure}[H]
    \centering
    \begin{minipage}{.53\textwidth}
        \centering
        \begin{subfigure}{.18\textwidth}
            \centering
            \includegraphics[width=\textwidth]{images/series_resistance.pdf}
        \end{subfigure}
        \begin{subfigure}{.28\textwidth}
            \centering
            \includegraphics[width=\textwidth]{images/parallel_resistance.pdf}
        \end{subfigure}
        \caption{串, 并联阻抗的等效互换}
    \end{minipage}
    \begin{minipage}{.40\textwidth}
        \centering
        \includegraphics[width=.55\textwidth]{images/parallel_circuit_general.pdf}
        \caption{并联电路的广义形式}
    \end{minipage}
\end{figure}

\textbf{并联阻抗变换为串联阻抗}
\begin{gather}
    \nonumber R_s+\mathrm{j}X_s=\frac{R_p(\mathrm{j}X_p)}{R_p+\mathrm{j}X_p}. \\
    R_s=\frac{X_p^2}{R_p^2+X_p^2}R_p=\frac{X_p^2}{Z_p^2}R_p. \\
    X_s=\frac{R_p^2}{R_p^2+X_p^2}X_p=\frac{R_p^2}{Z_p^2}X_p.
\end{gather}

\textbf{串联阻抗变换为并联阻抗}
\begin{gather}
    \nonumber \frac{1}{R_s+\mathrm{j}X_s}=\frac{1}{R_p}+\frac{1}{\mathrm{j}X_p}. \\
    R_p=\frac{R_s^2+X_s^2}{R_s}=\frac{Z_s^2}{R_s}. \\
    X_p=\frac{R_s^2+X_s^2}{X_s}=\frac{Z_s^2}{X_s}.
\end{gather}

可以推得, $Q_s=\dfrac{X_s}{R_s}$, $Q_p=\dfrac{R_p}{X_p}$, 且 $Q_s=Q_p$ (记作 $Q_L$). 因此
\begin{gather}
    R_s=\frac{R_p}{1+Q_L^2}. \\
    X_s=\frac{Q_L^2}{1+Q_L^2}X_p. \\
    R_p=(1+Q_L^2)R_s. \\
    X_p=\frac{1+Q_L^2}{Q_L^2}X_s.
\end{gather}

当 $Q_L$ \textit{较高} (大于 10) 时, $X_p$ 与 $X_s$ 的性质相同, 大小近似相等; $R_p$ 近似为 $R_s$ 的 $Q_L^2$ 倍.

\subsubsection{并联谐振回路的其他形式}

假设 $X_1>>R_1$, $X_2>>R_2$.
\begin{equation}
    Z_p=Z_1\parallel Z_2=\frac{(R_1+\mathrm{j}X_1)(R_2+\mathrm{j}X_2)}{R_1+R_2}\approx-\frac{X_1X_2}{R_1+R_2}.
\end{equation}

利用并联谐振条件
\begin{equation} \label{eq:2.3 并联谐振条件}
    \frac{1}{R_1+\mathrm{j}X_1}+\frac{1}{R_2+\mathrm{j}X_2}=0 \xRightarrow{X_1>>R_1,X_2>>R_2} X_1+X_2=0
\end{equation}

\noindent 推得
\begin{equation}
    Z_p=\frac{X_1^2}{R_1+R_2}=\frac{X_2^2}{R_1+R_2}.
\end{equation}

\subsubsection{抽头式并联电路的阻抗变换}

\begin{wrapfigure}{r}{.38\textwidth}
    \centering
    \includegraphics[width=.3\textwidth]{images/inductor_tapped_parallel_resonance_circuit.pdf}
    \caption{电感抽头式并联谐振回路}
\end{wrapfigure}

以下结论适用于谐振和非谐振回路.

\textbf{接入系数}\quad 两个抽头间电压之比.

\textbf{电感抽头}

考虑低抽头 ($L_1$) 比高抽头 ($L=L_1+L_2$), 因为串联电感两端电压与电感值成正比,
\begin{equation}
    p=\frac{L_1}{L_1+L_2}<1. \qquad (|Z_{ab}|\geq\omega_pL_1)
\end{equation}

\textbf{阻抗折合}
\begin{equation}
    Z_{ab}=\frac{X_1^2}{R_1+R_2}=\frac{(\omega_p L_1)^2}{R_1+R_2}\xlongequal[{\color{red} L=L_1+L_2}]{{\color{red} p=L_1/L}}\frac{(\omega_p L)^2}{R_1+R_2}p^2.
\end{equation}

\textit{考虑互感}时, $p=\dfrac{L_1\pm M}{L_1+L_2+2M}$, $L_1$ 与 $L_2$ 线圈绕组方向一致时取正号; 反之取负号.

\textbf{导纳折合}
\begin{equation}
    \frac{Y_{ab}}{Y_{db}}=n^2.
\end{equation}

\textbf{电压源折合}
\begin{equation}
    V_{ab}^2G_{ab}=V_{db}^2G_{db} \Rightarrow \frac{V_{ab}}{V_{db}}=\frac{L_1}{L_1+L_2}=p.\qquad (\textrm{\color{red} 等效电路功率不变})
\end{equation}

\textbf{电流源折合}

\begin{figure}[H]
    \centering
    \begin{subfigure}{.3\textwidth}
        \centering
        \includegraphics[width=\textwidth]{images/current_source_conversion_circuit_a.pdf}
    \end{subfigure}
    \begin{subfigure}{.3\textwidth}
        \centering
        \includegraphics[width=\textwidth]{images/current_source_conversion_circuit_b.pdf}
    \end{subfigure}
    \caption{电流源折合电路}
\end{figure}

\rmg
\begin{gather}
    R_i'=\frac{1}{p^2}R_i=n^2R_i. \\
    IV_{bc}=I'V_{ac} \Rightarrow \frac{I}{I'}=\frac{1}{p}=n.
\end{gather}

\textbf{负载阻容网络折合}

\begin{figure}[H]
    \centering
    \begin{subfigure}{.3\textwidth}
        \centering
        \includegraphics[width=.9\textwidth]{images/load_resistance_conversion_circuit_a.pdf}
    \end{subfigure}
    \begin{subfigure}{.3\textwidth}
        \centering
        \includegraphics[width=.9\textwidth]{images/load_resistance_conversion_circuit_b.pdf}
    \end{subfigure}
    \caption{负载阻容网络折合电路}
\end{figure}

\rmg
\begin{gather}
    R_L'=\frac{1}{p^2}R_L=n^2R_L. \\
    \frac{1}{\omega C_L'}=\frac{1}{p^2}\frac{1}{\omega C_L} \Rightarrow C_L'=p^2C_L.
\end{gather}

\textbf{谐振频率}
\begin{equation}
    \omega_p=\frac{1}{\sqrt{(L_1+L_2)C}}.\qquad (\textrm{利用\ (\ref{eq:2.3 并联谐振条件})})
\end{equation}

回路谐振时, 由回路的\textit{任何}两点看去, 回路都谐振于同一频率, 且呈纯电阻性.

\textbf{电容抽头}

考虑电容抽头时, 电压时反比关系, 所以
\begin{equation}
    p=\frac{1}{n}=\frac{C}{C_1}=\frac{C_2}{C_1+C_2}.\qquad (|Z_{ab}|\geq\frac{1}{\omega_pC_1}.)
\end{equation}

\noindent\hrulefill

总结不同抽头 (电感抽头, 电容抽头) 对不同元件参数的影响. 以下均假设从低抽头转换到高抽头, 变比均为高抽头参数与低抽头参数之比:
\begin{itemize}
    \item 不同抽头:
          \begin{itemize}
              \item 电感抽头: $p=\dfrac{L_1}{L}$
              \item 电容抽头: $p=\dfrac{C}{C_1}$
          \end{itemize}
    \item 不同元件:
          \begin{itemize}
              \item 阻抗: $\dfrac{1}{p^2}$
              \item 导纳: $p^2$
              \item 电压源: $\dfrac{1}{p}$
              \item 电流源: $p$
          \end{itemize}
\end{itemize}

\begin{exampleprob}
    如图 \ref{fig: 2.3 例题 紧耦合抽头电路} 为紧耦合抽头电路, $f_p=465$kHz, $R_s=27\mathrm{k}\Omega$, $R_p=172\mathrm{k}\Omega$, $R_L=1.36\mathrm{k}\Omega$, $Q_0=100$, $P_1=0.28$, $P_2=0.063$, $I_s=1$mA. 求回路通频带 $B$ 和等效电流源幅值 $I_s'$.

    \begin{figure}[H]
        \centering
        \begin{subfigure}{.45\textwidth}
            \centering
            \includegraphics[width=.9\textwidth]{images/exampleprob_2_3_tight_tapped_circuit_a.pdf}
            \caption{}
            \label{fig: 2.3 例题 紧耦合抽头电路}
        \end{subfigure}
        \begin{subfigure}{.45\textwidth}
            \centering
            \includegraphics[width=.9\textwidth]{images/exampleprob_2_3_tight_tapped_circuit_b.pdf}
            \caption{}
            \label{fig: 2.3 例题 紧耦合抽头电路 解}
        \end{subfigure}
    \end{figure}

    \begin{solution}
        电路等效为图 \ref{fig: 2.3 例题 紧耦合抽头电路 解}.
        \begin{gather*}
            R_s'=\frac{1}{P_1^2}R_s\approx 344.52\ \mathrm{k}\Omega,\quad R_L'=\frac{1}{P_2^2}R_L\approx 342.65\ \mathrm{k}\Omega. \\
            Q_L=\frac{Q_0}{1+\dfrac{R_p}{R_s'}+\dfrac{R_p}{R_L'}}\approx 50.\quad B=\frac{f_0}{Q_L}\approx 9.3\ \mathrm{kHz}.\quad I_s'=P_1I_s=0.28\ \mathrm{mA}.
        \end{gather*}
    \end{solution}
\end{exampleprob}
