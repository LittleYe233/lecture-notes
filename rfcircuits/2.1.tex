\section{选频网络}

\begin{itemize}
    \item 振荡回路 (由 L, C 组成)
          \begin{itemize}
              \item 单振荡回路
              \item 耦合振荡回路
          \end{itemize}
    \item 滤波器
          \begin{itemize}
              \item LC 集中滤波器
              \item 石英晶体滤波器
              \item 陶瓷滤波器
              \item 声表面波滤波器
          \end{itemize}
\end{itemize}

\subsection{串联谐振回路}
\subsubsection{基本原理}

\begin{figure}[H]
    \centering
    \begin{minipage}{.3\textwidth}
        \centering
        \begin{tikzpicture}[circuit ee IEC, x=3cm, y=2cm, semithick, every info/.style={font=\normalsize}, large circuit symbols]
            \draw (0, 0) to [voltage source={info={$\dot{V}_s$}}] (0, 1);
            \draw (0, 1) to [inductor={info={$L$}}] (1, 1);
            \draw (1, 1) to [capacitor={info={$C$}}] (1, 0);
            \draw (1, 0) to [resistor={info={$R$}}] (0, 0);
        \end{tikzpicture}
        \caption{串联谐振回路}
    \end{minipage}
    \qquad
    \centering
    \begin{minipage}{.5\textwidth}
        \centering
        \begin{tikzpicture}
            % See: https://tex.stackexchange.com/a/569760
            \begin{axis}[
                    domain=0:6.5,
                    xmin=0, xmax=7,
                    ymin=0, ymax=7,
                    ticks=none,
                    xlabel={$\omega$},
                    ylabel={$I$}]
                \addplot [color=flatblue] {3/(sqrt(0.5^2+(\x-1/\x)^2))};
                \addplot [color=tea] {3/(sqrt(1^2+(\x-1/\x)^2))};
                \addplot [color=copper] {3/(sqrt(2^2+(\x-1/\x)^2))};
                \addlegendentry{$Q=4Q^*$};
                \addlegendentry{$Q=2Q^*$};
                \addlegendentry{$Q=Q^*$};
            \end{axis};
        \end{tikzpicture}
        \caption{外加电压为常数时, $Q$ (或 $R$) 对 $I$-$\omega$ 曲线的影响 (谐振曲线)}
    \end{minipage}
\end{figure}

\textbf{阻抗}

\begin{equation}
    Z=R+\mathrm{j}\left(\omega L-\frac{1}{\omega C}\right)=|Z|e^{\mathrm{j}\varphi}.
\end{equation}

其中

\begin{equation}
    X=\left(\omega L-\frac{1}{\omega C}\right).
\end{equation}

\begin{equation}
    |Z|=\sqrt{R^2+X^2}=\sqrt{R^2+\left(\omega L-\frac{1}{\omega C}\right)^2}.
\end{equation}

\begin{equation}
    \varphi=\arctan\frac{X}{R}=\arctan\frac{\omega L-\dfrac{1}{\omega C}}{R}.
\end{equation}

\textbf{谐振条件}\quad $\omega=\omega_0=\dfrac{1}{\sqrt{LC}}$. $f_0=\dfrac{1}{2\pi\sqrt{LC}}$.

\begin{exampleprob}
    有一串联回路在频段 535kHz\textasciitilde 1605kHz 内工作. 现有两个可变电容器 A: 12pF\textasciitilde 100pF, B: 15pF\textasciitilde 450pF. 应该选用何种电容器?

    \begin{solution}
        电容器的可调电容值范围应尽可能宽, 使得回路在指定频段内可以达到串联谐振. 因为 $f=\dfrac{1}{2\pi\sqrt{LC}}$, 所以

        \begin{equation*}
            k=\frac{C_m}{C_n}=\left(\frac{f_m}{f_n}\right)^2=9.
        \end{equation*}

        A 电容器的比值 $k\approx 8.33$, B 电容器的比值 $k=30$, 故应该选用 B 电容器.
    \end{solution}
\end{exampleprob}

\textbf{特性阻抗}\quad 谐振时的感抗或容抗. $\rho=\sqrt{\dfrac{L}{C}}$.

\textbf{非谐振特性}

当 $\omega\neq\omega_0$ 时, $|Z|>R$.

\begin{itemize}
    \item $\omega>\omega_0$, $X>0$, 呈感性; $|\dot{V}_L|<|\dot{V}_C|$; $\varphi>0$, 电流滞后电压;
    \item $\omega<\omega_0$, $X<0$, 呈容性; $|\dot{V}_L|>|\dot{V}_C|$; $\varphi<0$, 电流超前电压.
\end{itemize}

\textbf{谐振特性}

\begin{itemize}
    \item $X=0$, $Z=R$, 呈纯阻性, $|Z|$ 达到最小值;
    \item $\varphi=0$, 电流与电压同相位;
    \item L 和 C 上的电压 $\dot{V}_{L0}$ 和 $\dot{V}_{C0}$ 大小相等, 相位相差 180°, $\dot{V}_s=\dot{V}_R$;
    \item $\dot{V}_{L0}=\mathrm{j}Q\dot{V}_s$, $\dot{V}_{C0}=-\mathrm{j}Q\dot{V}_s$, 其中 $Q=\dfrac{\omega_0L}{R}=\dfrac{1}{\omega_0CR}=\dfrac{\rho}{R}=\dfrac{1}{R}\sqrt{\dfrac{L}{C}}$ 为回路的\textbf{品质因数} ($Q$ 越高, 回路选择性越好).
\end{itemize}

当 $\dot{V}_s$ 保持不变时, $|\dot{I}|$ 达到最大值.

\subsubsection{能量关系} \label{能量关系}

串联单振荡回路中, 只有 R \textit{消耗}外加电动势的能量; 电路进入稳定状态后, L 和 C 只\textit{储存和交换}能量.

\textbf{电阻}

\begin{equation}
    \overline{P}_R=\frac{1}{2}RI_{0m}^2.
\end{equation}

单个周期:

\begin{equation}
    W_R=\overline{P}_T=\frac{1}{2}\frac{RI_{0m}^2}{f_0}.
\end{equation}

\textbf{电容和电感}

设谐振时 $i=I_{0m}\sin\omega t$,

\begin{equation*}
    \begin{gathered}
        v_C=\frac{1}{C}\int i\mathrm{d}t=\frac{I_{0m}}{\omega C}\sin(\omega t-90^\circ)=-V_{Cm}\cos\omega t, \\
        v_L=L\frac{\mathrm{d}i}{\mathrm{d}t}=LI_{0m}\omega\cos\omega t=V_{Lm}\cos\omega t.\qquad\textrm{(显然}\ V_{Cm}=V_{Lm}\textrm{.)}
    \end{gathered}
\end{equation*}

\begin{equation}
    W_C=\frac{1}{2}Cv_C^2=\frac{1}{2}CV_{Cm}^2\cos^2\omega t.
\end{equation}

\begin{equation}
    W_L=\frac{1}{2}Li_L^2=\frac{1}{2}LI_{0m}^2\sin^2\omega t.
\end{equation}

因为 $Q=\dfrac{1}{R}\sqrt{\dfrac{L}{C}}$, 所以 $CV_{Cm}^2=LI_{0m}^2$.

\begin{equation}
    W=W_L+W_C=\frac{1}{2}LI_{0m}^2.
\end{equation}

\begin{equation}
    \frac{W_C+W_L}{W_R}=\frac{f_0L}{R}=\frac{Q}{2\pi}.
\end{equation}

\subsubsection{串联振荡回路的谐振曲线和通频带} \label{串联振荡回路的谐振曲线和通频带}

\begin{equation}
    \frac{\dot{I}}{\dot{I}_0}=\frac{R}{R+\mathrm{j}\left(\dfrac{\omega}{\omega_0}-\dfrac{\omega_0}{\omega}\right)}=\frac{1}{1+\mathrm{j}Q\left(\dfrac{\omega}{\omega_0}-\dfrac{\omega_0}{\omega}\right)}.
\end{equation}

\begin{equation} \label{eq:2.1 I/I_0}
    \frac{I}{I_0}=\frac{1}{\sqrt{1+\left(\dfrac{X}{R}\right)^2}}=\frac{1}{\sqrt{1+Q^2\left(\dfrac{\omega}{\omega_0}-\dfrac{\omega_0}{\omega}\right)^2}}.
\end{equation}

\textbf{失谐 (失调)}\quad $\Delta\omega=\omega-\omega_0$.

当 $\omega\approx\omega_0$ 时, 利用

\begin{equation*}
    \frac{\omega}{\omega_0}-\frac{\omega_0}{\omega}=\frac{\omega^2-\omega_0^2}{\omega_0\omega}=\left(\frac{\omega+\omega_0}{\omega}\right)\left(\frac{\omega-\omega_0}{\omega_0}\right)\approx 2\frac{\Delta\omega}{\omega_0},
\end{equation*}

\noindent 将 (\ref{eq:2.1 I/I_0}) 改写为\textit{小量失谐}情况下通用形式的\textbf{谐振特性方程式}

\begin{equation} \label{eq:2.1 I/I_0 xi}
    \frac{I}{I_0}\approx\frac{1}{\sqrt{1+\left(Q\dfrac{2\Delta\omega}{\omega_0}\right)^2}}=\frac{1}{\sqrt{1+\xi^2}}.
\end{equation}

\textbf{广义失谐 (一般失谐)}\quad $\xi=\dfrac{X}{R}=Q\dfrac{2\Delta\omega}{\omega_0}$.

\textbf{边界角频率 (频率)}\quad $\dfrac{I}{I_0}=\dfrac{1}{\sqrt{2}}$ 时的 $\omega_{1,2}$ ($f_{1,2}$).

此时回路损耗功率为谐振功率的一半 (半功率点). $\xi=\pm 1$.

\textbf{通频带}\quad $B_\omega=2\Delta\omega_{0.7}=\omega_2-\omega_1=\dfrac{\omega_0}{Q}$, $B=2\Delta f_{0.7}=f_2-f_1=\dfrac{f_0}{Q}$.

$Q$ 越高, 通频带越窄.

\textbf{有载品质因数 (考虑其他内阻)}

\begin{equation}
    Q_L=\frac{\omega_0L}{R+R_s+R_L}.
\end{equation}

若信号源变为恒流源 ($R_s$ 和 $V_s$ 可趋于无穷大), $Q_L$ 降为零.

\begin{figure}[H]
    \centering
    \begin{tikzpicture}
        % See: https://tex.stackexchange.com/a/569760
        \begin{axis}[
                domain=0:6.5,
                xmin=0, xmax=7,
                ymin=0, ymax=7,
                ticks=none,
                xlabel={$\omega$},
                ylabel={$I$}]
            \addplot [color=flatblue] {3/(sqrt(0.5^2+(\x-1/\x)^2))};
            \addplot [color=tea] {3/(sqrt(0.75^2+(\x-1/\x)^2))};
            \addplot [color=copper] {3/(sqrt(1^2+(\x-1/\x)^2))};
            \addplot [color=lavender] {1};
            \addlegendentry{$R_s=0$};
            \addlegendentry{$R_s=0.5R$};
            \addlegendentry{$R_s=R$};
            \addlegendentry{恒流源};
        \end{axis};
    \end{tikzpicture}
    \caption{信号源内阻对谐振曲线的影响}
\end{figure}

串联谐振回路通常适用于\textit{信号源内阻很小}, \textit{负载电阻也不大}的情况.

\subsubsection{串联振荡回路的相位特性曲线} \label{串联振荡回路的相位特性曲线}

回路电流的\textbf{相位特性曲线表达式} (认为 $\dot{V}_s$ 的相位为零):

\begin{equation} \label{eq:2.1 psi}
    \psi=-\varphi=-\arctan Q\left(\frac{\omega}{\omega_0}-\frac{\omega_0}{\omega}\right)\approx-\arctan Q\frac{2\Delta\omega}{\omega_0}=-\arctan\xi.
\end{equation}

\begin{figure}[H]
    % See: https://tex.stackexchange.com/a/272262/290833
    \begin{minipage}{.48\textwidth}
        \centering
        \begin{tikzpicture}
            \begin{axis}[%
                    domain=10:30,
                    xmin=10, xmax=30,
                    ymin=-2, ymax=2,
                    xlabel={$\omega$}, ylabel={$\psi$},
                    % See: https://tex.stackexchange.com/a/406697/290833
                    x tick label style={xshift=-0.6em},
                    extra x ticks={20},
                    extra x tick label={$\omega_0$},
                    xtick=\empty,
                    % See: https://tex.stackexchange.com/a/34958/290833
                    ytick={-0.7853981633974483, -1.5707963267948966, 0.7853981633974483, 1.5707963267948966},
                    yticklabels={$-\dfrac{\pi}{2}$, $-\dfrac{\pi}{4}$, $\dfrac{\pi}{4}$, $\dfrac{\pi}{2}$},
                    extra y ticks={0},
                    extra y tick label={0},
                    % See: https://tex.stackexchange.com/a/134086/290833
                    legend style={at={(1, 0.85)}}]
                \addplot[color=flatblue]{-rad(atan(10*(\x/20-20/\x)))};
                \addplot[color=copper]{-rad(atan(2.5*(\x/20-20/\x)))};
                \addplot[black, dashed]{pi/2};
                \addplot[black, dashed]{-pi/2};
                \addlegendentry{$Q_1$};
                \addlegendentry{$Q_2$}
            \end{axis};
        \end{tikzpicture}
        \caption{串联振荡回路的相位特性曲线 ($Q_1>Q_2$)}
    \end{minipage}
    \begin{minipage}{.48\textwidth}
        \centering
        \begin{tikzpicture}
            \begin{axis}[%
                    domain=-8:8,
                    xmin=-8.5, xmax=8.5,
                    ymin=-2, ymax=2,
                    xlabel={$\xi$}, ylabel={$\psi$},
                    % See: https://tex.stackexchange.com/a/34958/290833
                    ytick={-0.7853981633974483, -1.5707963267948966, 0, 0.7853981633974483, 1.5707963267948966},
                    yticklabels={$-\dfrac{\pi}{2}$, $-\dfrac{\pi}{4}$, 0, $\dfrac{\pi}{4}$, $\dfrac{\pi}{2}$}]
                \addplot[black]{-rad(atan(\x))};
                \addplot[black, dashed]{pi/2};
                \addplot[black, dashed]{-pi/2};
            \end{axis};
        \end{tikzpicture}
        \caption{串联振荡回路通用相位特性}
    \end{minipage}
\end{figure}

\begin{exampleprob}
    设给定串联谐振回路的 $f_0=1$ MHz, $Q_0=50$. 若输出电流超前信号源电压相位 $45^\circ$, 试求:

    \begin{enumerate}
        \item 信号源频率 $f$; 输出电流相对于谐振时衰减多少分贝;
        \item 现要在回路中再串联一个元件使回路谐振, 该元件的种类及其参数满足的表达式.
    \end{enumerate}

    \begin{solution}[1]
        \begin{enumerate}
            \item 根据 (\ref{eq:2.1 psi}),

                  \begin{equation*}
                      45^\circ=-\arctan Q_0\left(\frac{f}{f_0}-\frac{f_0}{f}\right),
                  \end{equation*}

                  解得 $f\approx 990$ kHz.

                  根据 (\ref{eq:2.1 I/I_0 xi}),

                  \begin{equation*}
                      {\color{red} 20}\lg\left|\frac{I}{I_0}\right|=20\lg\left|\frac{1}{\sqrt{1+(-\tan 45^\circ)^2}}\right|=-3\ \mathrm{dB}.
                  \end{equation*}

                  (求解\textit{电流衰减}时, 因为功率和电流呈二次方关系, 用分贝表示能量衰减时的系数为 10, 则相应地转化为电流衰减系数为 20.)
            \item 显然此时回路呈容性, 需要串联一电感 $L^*$ 满足

                  \begin{equation*}
                      4\pi^2 f_0(L+L^*)=\frac{1}{f_0 C}.
                  \end{equation*}
        \end{enumerate}
    \end{solution}

    \begin{solution}[2]
        此时广义失谐为 $\xi=-\arctan 45^\circ=-1$, 说明回路实际频率即为\textit{边界频率的较低值}, 也即 $f=f_0-\Delta f$. 而根据 (\ref{eq:2.1 psi}) 可以求出 $\Delta f$, 继而可以求得 $f$.
    \end{solution}
\end{exampleprob}
