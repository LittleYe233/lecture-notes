\section{附录} \label{附录}
\subsection{图片索引} \label{图片索引}
% See: https://tex.stackexchange.com/a/55088
\makeatletter
\@starttoc{lof}
\makeatother

\subsection{表格索引} \label{表格索引}
\makeatletter
\@starttoc{lot}
\makeatother

\subsection{例题索引} \label{例题索引}
\listofexampleprobs

\subsection{疑问索引} \label{疑问索引}
\listofquestions

\subsection{更新日志} \label{更新日志}
\subsubsection*{1.3.1 (2023-3-3)}
\begin{itemize}
    \item 使用 \texttt{\textbackslash rmg}, \texttt{\textbackslash srmg} 重设 \texttt{equation}, \texttt{gather} 等环境前后空白间距.
\end{itemize}

\subsubsection*{1.3.0 (2023-3-3)}
\begin{itemize}
    \item 增加 \ref{串, 并联阻抗的等效互换与回路抽头时的阻抗变换} 的部分, 及其他内容;
    \item 增加 2 道例题;
    \item 删除 2 个问题;
    \item 修改部分图片的位置;
    \item 修改更新日志的字词等其他细微改动.
\end{itemize}

\subsubsection*{1.2.1 (2023-2-27)}
\begin{itemize}
    \item 使用 \texttt{\textbackslash include} 重构文档结构;
    \item 移除目录中多余的 ``更新日志'' 条目.
\end{itemize}

\subsubsection*{1.2.0 (2023-2-26)}
\begin{itemize}
    \item 增加 \ref{能量关系}, \ref{串联振荡回路的相位特性曲线} 至 \ref{并联谐振回路} 的部分, 及其他内容;
    \item 增加 2 道例题;
    \item 修改图片的位置选项为 \texttt{H};
    \item 增加 \hyperref[附录]{附录}, 并增加 \hyperref[图片索引]{图片索引}, \hyperref[表格索引]{表格索引}, \hyperref[例题索引]{例题索引}, \hyperref[疑问索引]{疑问索引} 的部分;
    \item 其他细微改动.
\end{itemize}

\subsubsection*{1.1.0 (2023-2-23)}
\begin{itemize}
    \item 增加 \ref{绪论} 至 \ref{串联振荡回路的谐振曲线和通频带} 的部分;
    \item 增加 \hyperref[更新日志]{更新日志}.
\end{itemize}

\subsubsection*{1.0.0 (2023-2-22)}
\begin{itemize}
    \item 增加 \hyperref[课程信息]{课程信息} 的部分.
\end{itemize}
