\subsection{并联谐振回路} \label{并联谐振回路}
\subsubsection{基本原理及特性}

\begin{figure}[H]
    \centering
    \includegraphics[width=.25\textwidth]{images/parallel_resonance_circuit.pdf}
    \caption{并联谐振回路}
\end{figure}

\rmg
\begin{subequations}
    \begin{align}
                                                & Z=\frac{(R+\mathrm{j}\omega L)\dfrac{1}{\mathrm{j}\omega C}}{(R+\mathrm{j}\omega L)+\dfrac{1}{\mathrm{j}\omega C}}                                              \\
        \xRightarrow{{\color{red} \omega L>>R}} & Z\approx\frac{\dfrac{L}{C}}{R+\mathrm{j}\left(\omega L-\dfrac{1}{\omega C}\right)}=\frac{1}{\dfrac{CR}{L}+\mathrm{j}\left(\omega C-\dfrac{1}{\omega L}\right)}.
    \end{align}
\end{subequations}

\srmg
\begin{gather}
    \dot{V}=\dot{I}_sZ=\frac{\dot{I}_s}{\dfrac{CR}{L}+\mathrm{j}\left(\omega C-\dfrac{1}{\omega L}\right)}. \\
    Y=\frac{1}{Z}=G+\mathrm{j}B=\frac{CR}{L}+\mathrm{j}\left(\omega C-\frac{1}{\omega L}\right).
\end{gather}

\textbf{谐振条件}\quad $B=\omega_pC-\dfrac{1}{\omega_p L}=0$.
\begin{equation}
    \omega_p=\frac{1}{\sqrt{LC}}.
\end{equation}

不满足 $\omega L>>R$ 时, 谐振频率的精确解:
\begin{equation}
    \omega_p=\sqrt{\frac{1}{LC}-\frac{R^2}{L^2}}.
\end{equation}

\textbf{谐振特性}

\begin{itemize}
    \item $\dot{V}_0$ 与 $\dot{I}_s$ 同相, $V_0=V_m=\dfrac{I_sL}{CR}$;
    \item 谐振时, 回路阻抗为纯电阻且达到最大值;
    \item 电容支路和电感支路电流大小相等, 相位相差 180°, 各支路电流为总电流的 $Q_p$ 倍
          \begin{equation}
              \dot{I}_{Cp}=\mathrm{j}Q_p\dot{I}_s,\quad\dot{I}_{Lp}=-\mathrm{j}Q_p\dot{I}_s;
          \end{equation}
    \item 考虑 $R$ 时, 在 $Q_p$ 很大时, 可以认为 $Z_p$ 为\textit{纯电阻时恰好为最大值}; 此时 $\dot{I}_L$ 的相位落后 $\dot{V}_0$ 小于 90°.
\end{itemize}

\textbf{非谐振特性}

\begin{itemize}
    \item $\omega>\omega_p$, 呈容性;
    \item $\omega<\omega_p$, 呈感性.
\end{itemize}

\textbf{品质因数}
\begin{equation}
    Q_p=\frac{\omega_p}{L}=\frac{1}{\omega_pCR}=\frac{1}{R}\sqrt{\frac{L}{C}}.
\end{equation}

\textbf{谐振电阻}
\begin{equation}
    R_p=\frac{\dot{V}_0}{\dot{I}_s}=\frac{L}{CR}=\frac{\omega_p^2L^2}{R}=Q_p\omega_pL=Q_p\frac{1}{\omega_pC}=\frac{1}{R\omega_p^2C^2}.
\end{equation}

谐振时, 并联振荡回路的谐振电阻等于电感支路或电容支路电抗值的 $Q_p$ 倍.

\subsubsection{并联振荡回路的谐振曲线, 相位特性曲线和通频带}

类比串联振荡回路:
\begin{equation}
    \frac{\dot{V}}{\dot{V}_0}=\frac{1}{1+\mathrm{j}Q_p\left(\dfrac{\omega}{\omega_p}-\dfrac{\omega_p}{\omega}\right)}.
\end{equation}

\begin{subequations}
    \begin{equation}
        \frac{V}{V_0}=\frac{1}{\sqrt{1+Q_p^2\left(\dfrac{\omega}{\omega_p}-\dfrac{\omega_p}{\omega}\right)^2}}.
    \end{equation}
    \begin{equation}
        \frac{V}{V_0}=\frac{1}{\sqrt{1+\xi^2}}.
    \end{equation}
\end{subequations}

\begin{subequations}
    \begin{equation}
        \psi=-\arctan Q_p\left(\frac{\omega}{\omega_p}-\frac{\omega_p}{\omega}\right).
    \end{equation}
    \begin{equation}
        \psi=-\arctan\xi.
    \end{equation}
\end{subequations}

\textbf{绝对通频带}\quad $B_\omega=2\Delta\omega_{0.7}=\dfrac{\omega_p}{Q_p}$, $B=2\Delta f_{0.7}=\dfrac{f_p}{Q_p}$.

\textbf{相对通频带}\quad $\dfrac{2\Delta\omega_{0.7}}{\omega_p}=\dfrac{2\Delta f_{0.7}}{f_p}=\dfrac{1}{Q_p}$.

\subsubsection{信号源内阻和负载内阻的影响}

\begin{figure}[H]
    \centering
    \includegraphics[width=.6\textwidth]{images/parallel_resonance_circuit_rs_rl.pdf}
    \caption{考虑 $R_s$ 和 $R_L$ 后的并联振荡回路的等效电路}
    \label{fig:2.2 有载有内阻并联振荡回路}
\end{figure}

将 $R_p$ 并联在电路中, 因为谐振状态 $\dot{I}_C+\dot{I}_L=0$.

\textbf{品质因数}
\begin{subequations}
    \begin{equation}
        Q_L=\frac{1}{\omega_pL(G_p+G_s+G_L)}=\frac{1}{\omega_pL\left(\dfrac{1}{R_p}+\dfrac{1}{R_s}+\dfrac{1}{R_L}\right)}.
    \end{equation}
    \begin{equation}
        Q_L=\frac{Q_p}{1+\dfrac{R_p}{R_s}+\dfrac{R_p}{R_L}}.
    \end{equation}
\end{subequations}

并联更多的电阻, 或降低 $R_s$, $R_L$ 的阻值, 回路总电导增大, 品质因数降低, 通频带变宽, 回路选择性更差.

\begin{figure}[H]
    \centering
    \begin{subfigure}{.45\textwidth}
        \centering
        \includegraphics[width=\textwidth]{images/parallel_resonance_curve_constant_voltage.pdf}
        \caption{信号源电动势保持不变}
    \end{subfigure}\qquad
    \begin{subfigure}{.45\textwidth}
        \centering
        \includegraphics[width=\textwidth]{images/parallel_resonance_curve_constant_vc.pdf}
        \caption{调整信号源电动势使 $V_C=1\mathrm{V}$}
    \end{subfigure}
    \caption{电源内阻对并联谐振曲线的影响 ($L=1\mathrm{H},R=0.1\Omega,C=1\mathrm{F},V_s=1\mathrm{V}$)}
\end{figure}

为获得优良的选择性, 信号源内阻低时, 应采用串联振荡回路; 信号源内阻高时, 应采用并联振荡回路.

\begin{exampleprob}
    有一并联谐振回路如图 \ref{fig:2.2 有载有内阻并联振荡回路} 所示, $Q_p=80$, $R_p=25\mathrm{k}\Omega$, $f_0=30$MHz, $I_s=0.1$mA. 求:

    \begin{enumerate}
        \item $R_s=10\mathrm{k}\Omega$, $R_L=0$, $B$ 和谐振输出电压幅值 $V_0$;
        \item $R_s=6\mathrm{k}\Omega$, $R_L=2\mathrm{k}\Omega$, $B$ 和 $V_0$.
    \end{enumerate}

    \begin{solution}
        \begin{enumerate}
            \item \begin{equation*}
                      Q_L=\frac{Q_p}{1+\dfrac{R_p}{R_s}}=\frac{160}{7},\quad B=\frac{f_0}{Q_L}\approx 1.3\ \mathrm{MHz},\quad V_0=I_s\frac{R_pR_s}{R_p+R_s}\approx 0.71\ \mathrm{V}.
                  \end{equation*}
            \item \begin{equation*}
                      Q_L\approx 4.5,\quad B\approx 6.7\ \mathrm{MHz},\quad V_0\approx 0.14\ \mathrm{V}.
                  \end{equation*}
        \end{enumerate}
    \end{solution}
\end{exampleprob}

\textbf{插入损耗}\quad 回路的一部分功率被回路电导 $G_p$ 损耗, 而不能全部送给负载. $K_l$ 越大, 回路损耗越大.

\rmg
\begin{gather}
    P_L=\frac{I_s^2}{(G_s+G_L)^2}G_L, \\
    P_L'=\frac{I_s^2}{(G_s+G_L+G_p)^2}G_L. \\
    K_l=\frac{P_L}{P_L'}=\left(\frac{1}{1-\dfrac{Q_L}{Q_p}}\right)^2\geq 1. \\
    K_l(\mathrm{dB})=10\lg K_l=20\lg\frac{1}{1-\dfrac{Q_L}{Q_p}}.
\end{gather}

\subsubsection{低 Q 值 (Q<10) 的并联谐振回路}

假设工作频率保持不变, 若要得到谐振 (满足 $Z_p$ 最大且 $Z_p$ 为纯电阻):

\begin{itemize}
    \item 电阻集中在电感支路, 电容支路的电阻为零, 则改变 C;
    \item 电阻集中在电容支路, 电感支路的电阻为零, 则改变 L.
\end{itemize}
